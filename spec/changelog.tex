\subsection*{Unreleased}

Proof rules:
\begin{itemize}
  \item Addition of a section describing bitvector proofs.
  \item Bitblasting rules: \proofRule{bitblast_extract},  \proofRule{bitblast_add},
      \proofRule{bitblast_ult}.
  \item Addition of rules \proofRule{la_mult_pos} and \proofRule{la_mult_neg} to describe multiplication with a positive or negative factor.
  \item Addition of rules \proofRule{symm} and \proofRule{not_symm} to express symmetry of equality explicitly.
  \item Addition of the rule \proofRule{eq_symmetric} to express symmetry
  of equality explicitly but as an equivalence. Note that in principle this
  could be done with the rule \proofRule{symm} above, but would require a long
  and tedious use of \proofRule{subproof} for each direction of the equivalence.
  \item Addition of the rule \proofRule{weakening} to express weakening of a
        clause.
  \item Addition of the \proofRule{reordering} rule to represent reordering of
        the literals in a clause.
  \item Addition of the \proofRule{bind_let} rule.  This rule can be used to
        preprocess \inlineAlethe{let} expressions similar to the \proofRule{bind}
        rule used with ordinary quantifiers.
\end{itemize}

\noindent
Breaking changes:
\begin{itemize}
  \item Allow arbitrary extra annotations in \texttt{assume} commands.
  \item Add the sort to all variables in contexts.  Before, the context
        of a bind could be \texttt{(x S) (:= y x)}.  Now it must
        be \texttt{(x S) (:= (y S) x)}.
  \item The arguments for \proofRule{forall_inst} have been changed to
        no longer take the shape of bindings using \texttt{(:= x c)}.
        Instead, the list of instantiation terms must follow the variable
        order and cover all the respective bound variables.
  \item The rules \proofRule{and_pos}, \proofRule{or_neg}, \proofRule{and},
        \proofRule{not_or} now have one argument that indicates which subterm
        they use.
  \item Add new syntax for decimal and negative numbers and use it
        for \proofRule{la_generic}.
  \item Restrict sorting of numbers such that the sort of a constant is
        only determined by its syntactic category.
  \item Add new syntax for decimal and negative numbers and clarify that
        the set of allowed symbols is restricted to not conflict with this
        new syntax.
  \item Always use decimals for the constants in \proofRule{la_generic}.
  \item Restrict sorting of numbers such that the sort of a constant is
        only determined by its syntactic category.
\end{itemize}

\noindent
Clarifications and corrected errors:
\begin{itemize}
  \item Clarify that the \texttt{:args} annotation in \texttt{anchor} can be
        omitted if the list is empty.
  \item Clarify that \proofRule{bind} can work on existential and universal
        quantifiers.
  \item Fix mistake in proof grammar.  It now uses the \texttt{context\_annotation}
        non-terminal in the rule for the \texttt{anchor} command.
  \item Fix the example of \proofRule{onepoint}.
  \item Add the missing context to the conclusion of \proofRule{bind},
        \proofRule{sko_ex}, \proofRule{sko_forall}, \proofRule{onepoint}.
\end{itemize}

\subsection*{0.3 --- \DTMdisplaydate{2023}{02}{10}{-1}}

This release overhauls the entire document, but introduces only few
changes to the proof format itself.

The standard now specifies that \texttt{assume} commands can only be
issued at the start of the proof or right after an \texttt{anchor}
command.

Beyond many smaller clarifications and typographic improvements, the
following changes are implemented in this release.
\begin{itemize}
  \item Add an abstract proof checking procedure to clarify
  the semantics of the proof format.
  \item Add a description of the semantics of contexts based on λ-terms.
  \item List all transformations that are implicit in Alethe proofs.
  \item Change the notation used for terms from first-order style
   (e.g., $f(x,g(y))$) to higher-order style (e.g.,
   $(f\;x\;(g\;y))$).  This is only a change in notation -- the used logic
   remains many-sorted first-order logic.
  \item Eliminate the distinction between if-and-only-if and
  equality.  Instead, use equality (the symbol $≈$) with appropriate
   sorts.
  \item Add an index that lists all proof rules.
\end{itemize}

Proof rules:
\begin{itemize}
  \item The rule \proofRule{implies_simplify} is no longer allowed to
  perform the simplification $(\varphi_1\rightarrow \varphi_2)\rightarrow \varphi_2
  ⇒  \varphi_1\lor \varphi_2$.  This is now covered by \proofRule{bool_simplify}.
\end{itemize}

\subsection*{0.2 --- \DTMdisplaydate{2022}{12}{19}{-1}}

This is an intermediate release.  It collects all changes to the original
specification document before the major changes that were implemented as
part of Hans-Jörg Schurr's PhD thesis.   These changes will be reflected
in release 0.3.

This release implements major changes to the structure of the document to
clarify the difference between the \emph{language} and the \emph{rules}.
The language has a formal definition and a proof of soundness.  The syntax
describes how proofs are encoded in the text file.

The syntax was extended to allow extra annotations.  Tools consuming Alethe
proofs must be able to ignore such extra annotations.

List of rules:
\begin{itemize}
    \item Improve description of \proofRule{sko_ex}.
    \item Add {\proofRule{hole}} rule to allow holes.  A proof that contains
    steps that use this rule is not valid.
\end{itemize}
\noindent
Corrections:
\begin{itemize}
    \item Grammar: the \texttt{choice} binder can only bind one variable.
\end{itemize}

\noindent
Clarifications:
\begin{itemize}
    \item Clarify functionality of choice in introduction.
    \item Add illustrating example to introduction.
    \item Normalize printing of (variable, term) arguments in the abstract notation.
    \item Fix linear arithmetic example in introduction.
    \item Change syntax of abstract proof steps to be clearer.
\end{itemize}


\subsection*{0.1 -- \DTMdisplaydate{2021}{07}{10}{-1}}

This is the first public release of this document.  It coincides with
the seventh PxTP Workshop.
