%
% This document is licenced under Creative Commons 4.0 Attribution International
%

\documentclass{scrartcl}

\usepackage{scrlayer-scrpage}
\usepackage{unicode-math}
\usepackage{fontsetup}

\usepackage{minted}
\usemintedstyle{trac}
\renewcommand{\MintedPygmentize}{./highlight.py}
\newminted[AletheVerb]{smt-lib}{}
\newmintinline[inlineAlethe]{smt-lib}{}

\usepackage{cite}
\usepackage{url}
\usepackage[useregional]{datetime2}
\usepackage{xltabular} % Gives us stretchable, breakable tables. for proofs
\usepackage{longtable} % Gives us longtables for proof
\usepackage{environ} % Gives us NewEnviron to define environements with tables
\usepackage{fancyvrb} % Gives us \Verb which can be used in footnotes
\usepackage{amsthm}
\usepackage{thmtools}
\usepackage{ccicons} % Gives us CC icons for title page
\usepackage[nottoc,notlot,notlof]{tocbibind}

\usepackage{imakeidx}
\makeindex
\makeindex[name=rules,options= -s rule_index_style.ist]
% We provide a command that discards the argument to the index.  This
% allows us to do the sectioning of the index manually and give different
% sectin levels to the rule index and the overall index
\newcommand\indexsection[1]{}
\indexsetup{level=\indexsection,firstpagestyle=headings,noclearpage}

% Must come after imakeidx
\usepackage[hidelinks,hypertexnames=false]{hyperref}
\usepackage{breakurl}

\usepackage{tikz}
\usetikzlibrary{tikzmark,positioning}
\usetikzlibrary{shapes,arrows,fit, scopes}
\usetikzlibrary{arrows.meta}
\usetikzlibrary{svg.path}

\usepackage{bussproofs}
\EnableBpAbbreviations

\usepackage{amsthm}

\usepackage[english]{babel}
\nonfrenchspacing

\usepackage[final,tracking=true,stretch=10,shrink=10]{microtype}
\microtypecontext{spacing=nonfrench}
% Used for Aletheia in greek
%\newfontfamily\greekfont[Script=Greek, Scale=MatchUppercase, Ligatures=TeX]{CMU Serif}

\hyphenation{Isa-belle Nieuw-en-huis meta-logic multi-set multi-sets non-unit Mini-Sat}

\newcommand\smtlib{SMT-LIB}
\newcommand\cdclt{CDCL(T)}

\newcommand\tool[1]{\textsf{#1}}
\newcommand\tactic[1]{\texttt{#1}}
\newcommand\verit{\tool{veriT}}
\newcommand\cvcfive{\tool{cvc5}}
\newcommand\cvcfour{\tool{CVC4}}
\newcommand\isabelle{\tool{Isabelle/HOL}}

\newcommand\lsymb[1]{\mathbf{#1}}
\DeclareMathOperator*{\subst}{subst}
\DeclareMathOperator*{\reify}{reify}
\DeclareMathOperator*{\bvexplode}{bvexplode}
\newcommand\groundbox[1]{\boxed{#1}}

% Proofs
\newcommand\textAlethe[1]{\texttt{#1}}
\newcommand\ctxsep{$\vartriangleright$}
\newcommand\spctxsep{\multicolumn{1}{|c}{\ctxsep}}
\newcommand\spctx[1]{\multicolumn{1}{|c}{#1}}
\newcommand\spsep{\cline{2-4}}

% Empty lines with dots
\newcommand\aletheLine{ &                       & $\vdots$ & \\}
\newcommand\aletheLineB{& \multicolumn{1}{|c}{} & $\vdots$ & \\}
% with subproof line to the left
\newcommand\aletheLineS {&                       & & $\vdots$ & \\}
\newcommand\aletheLineSB{& \multicolumn{1}{|c}{} & & $\vdots$ & \\}

% Tables for proofs
\newcolumntype{Y}{>{\centering\arraybackslash}X}

\newcommand{\ruleTypeImpl}[1]{%
  \microtypesetup{tracking=false}\textsf{#1}\microtypesetup{tracking=true}%
}
\def\ruleType#1{\ruleTypeImpl{\detokenize{#1}}} % non linked rule (for examples)
\def\proofRule#1{\hyperref[rule:\detokenize{#1}]{\ruleTypeImpl{\detokenize{#1}}}} % linked rule
%\newcommand{\ruleref}[1]{\ruleType{\nameref{rule:#1}}~(\ref{rule:#1})}
\def\ruleref#1{\hyperref[rule:\detokenize{#1}]{\ruleTypeImpl{\detokenize{#1}}}~(\ref{rule:\detokenize{#1}})} % for the rule tables

\NewEnviron{Alethe}{%
\renewcommand\spsep{\cline{2-4}}
\setlength{\arrayrulewidth}{0.8pt}
\addtolength{\tabcolsep}{-4pt}
\begin{xltabular}{\linewidth}{l c Y r}
  \BODY
\end{xltabular}
\addtolength{\tabcolsep}{+4pt}
}
\NewEnviron{AletheS}{%
\renewcommand\spsep{\cline{2-5}}
\setlength{\arrayrulewidth}{0.8pt}
\addtolength{\tabcolsep}{-4pt}
\begin{xltabular}{\linewidth}{l l c Y r}
  \BODY
\end{xltabular}
\addtolength{\tabcolsep}{+4pt}
}

% These avoid the xltabular environment. While xltabular can break on pages
% it seems to interact badly with the RuleDescription enviornment.
\NewEnviron{AletheX}{%
\renewcommand\spsep{\cline{2-4}}%
\setlength{\arrayrulewidth}{0.8pt}%
\addtolength{\tabcolsep}{-4pt}%
\noindent
\begin{tabularx}{\linewidth}{l c Y r}
  \BODY
\end{tabularx}
\addtolength{\tabcolsep}{+4pt}
}
\NewEnviron{AletheXS}{%
\renewcommand\spsep{\cline{2-5}}%
\setlength{\arrayrulewidth}{0.8pt}%
\addtolength{\tabcolsep}{-4pt}%
\noindent
\begin{tabularx}{\linewidth}{l l c Y r}
  \BODY
\end{tabularx}
\addtolength{\tabcolsep}{+4pt}
}

% Environment for proof-rules
\newcounter{ProofRuleCounter}
\newcommand\currule{\textbf{ERROR}}

\declaretheoremstyle[
headfont=\sffamily\bfseries,%
notefont=\sffamily\bfseries,%
notebraces={}{},%
bodyfont=\normalfont,%
headpunct={},%
headformat={\NAME~\NUMBER:\NOTE},%
break
]{proof-rule-style}

\declaretheorem[name=Rule,style=proof-rule-style,sibling=ProofRuleCounter]{inner-rule}

\makeatletter
\newcommand{\ruleparagraph}{%
  \@startsection{paragraph}{4}%
  {\z@}{1ex \@plus 0.5ex \@minus .2ex}{-1em}%
  {\normalfont\normalsize\bfseries}%
}
\makeatother

\NewEnviron{RuleDescription}[1]{%
\renewcommand\currule{\proofRule{#1}}
\index[rules]{#1}

\begin{inner-rule}[\detokenize{#1}]\label{rule:#1}

\BODY
\end{inner-rule}
}

\theoremstyle{definition}
\newtheorem{RuleExample}{Example}[inner-rule]

% TODO: synchronize colors!
\definecolor{SmtBlue}{HTML}{00007f}
\definecolor{SmtGreen}{HTML}{3b7f31}
\definecolor{SmtStepId}{HTML}{3b7f31}

\newcommand{\grNT}[1]{\textcolor{SmtGreen}{\langle\texttt{#1}\rangle}}
\newcommand{\grT}[1]{\textcolor{SmtBlue}{\texttt{#1}}}
\newcommand{\grRule}{≔}
\newcommand{\grOr}{|}

\newtheorem{example}{Example}
\newtheorem{theorem}{Theorem}[example]
\newtheorem{lemma}{Lemma}[example]
\newtheorem{definition}{Definition}[example]
\newtheorem{seppara}{}

\newpairofpagestyles{titlestyle}{
\ofoot{Version \input{|"./version.sh"}\unskip . Typeset on {\today}.}%
\ifoot{\ccby}%
}
\renewcommand*{\titlepagestyle}{titlestyle}
\setkomafont{publishers}{\small}

\title{The Alethe Proof Format}
\subtitle{An Evolving Specification and Reference}
\author{Haniel Barbosa\textsuperscript{1}
\and Mathias Fleury\textsuperscript{2}
\and Pascal Fontaine\textsuperscript{3}
\and Hans-Jörg Schurr\textsuperscript{4}}
\date{}
\publishers{
  \textsuperscript{1} Universidade Federal de Minas Gerais, Brazil\\
  \textsuperscript{2} Albert-Ludwig-Universität Freiburg, Germany\\
  \textsuperscript{3} Université de Liège, Belgium\\
  \textsuperscript{4} The University of Iowa, Iowa City, USA\\
}


\begin{document}

\maketitle
\tableofcontents
\clearpage

\begin{abstract}
\section*{Foreword}
This document is a speculative specification and reference of a proof
format for SMT solvers.  The format consists of a language to express
proofs and a set of proof rules.  On the one side, the language is
inspired by natural-deduction and is based on the widely used SMT-LIB
format.  The language also includes a flexible mechanism to reason
about bound variables which allows fine-grained preprocessing proofs.
%
On the other side, the rules are structured around resolution and the
introduction of theory lemmas, in the same way as CDCL(T)-based SMT
solvers.

The specification is not yet
cast in stone, but it will evolve over time. It emerged from a list
of proof rules used by the SMT solver veriT collected in a document
called ``Proofonomicon''. Following the fate presupposed by its name,
it informally circulated among researchers interested in the proofs
produced by veriT after a few months.  We now polished this document
and gave it a respectable name.

Instead of aiming for theoretical purity, our approach
is pragmatic: the specification describes the format as it is in use
right now. It will develop in parallel with practical support for the
format within SMT solvers, proof checkers, and other tools.  We believe
it is not a perfect specification that fosters the adaption of a format,
but great tooling. This document will be a guide to develop
such tools.

Nevertheless, it not only serves as a norm to ensure compatibility
between tools, it also allows us to uncover the unsatisfactory aspects
that would otherwise be hidden deep within the nooks and crannies of
solver and checker implementations.
%
Every uncovered problem presents an opportunity
to improve the format. The authors of this document overlap with the
authors of those tools and we are committed to improve the tools, the
format, and ultimately the specification together.
This document is also an invitation to other researchers to join
these efforts. To read the reference and provide feedback, or to even
implement support for Alethe into their own tools. Please
get in touch!

\bigskip
\hspace*{\fill}The authors.
\end{abstract}

\section{Introduction}
This document is a reference of the
Alethe\footnote{Alethe is a genus of small birds that occur in West Africa~\cite{wp:alethe}.
The name was chosen because it
resembles the Greek word {αλήθεια} (alítheia) -- truth.} proof format.  Alethe is
designed to be a flexible format to represent unsatisfiability proofs
generated by SMT solvers.  Alethe proofs can be consumed by other systems,
such as interactive theorem provers or proof checkers.  The design
is based on natural-deduction style structure
and rules generating and operating on first-order clauses.
%
The Alethe proof format consists of two parts: the proof
language based on {\smtlib} and a collection of proof rules.
Section~\ref{sec:alethe:language} introduces the language.  First as an
abstract language, then as a concrete syntax.
Section~\ref{sec:alethe:semantics} then discusses an abstract procedure
to check Alethe proofs.  This abstract checking procedure specifies
the semantics of Alethe proofs.
%
The Alethe proof rules are discussed in two sections.
First, Section~\ref{sec:alethe:rules-generic} discusses the core
concepts behind the rules.
%
Second, Section~\ref{apx:rules} presents a list of
all proof rules currently used by {\verit}.

Alethe follows a few core design principles.  First, proofs should
be easy to understand by humans to ensure working with Alethe proofs
is easy.  Second, the language of the format should directly
correspond to the language used by the solver.  Since many solvers use the
{\smtlib} language, Alethe also uses this language.
Therefore, Alethe's base logic is the many-sorted first-order logic of {\smtlib}.
Third, the format should
be uniform for all theories used by SMT solvers.  With the exception
of clauses for propositional reasoning, there is no dedicated syntax for
any theory.

The Alethe format was originally developed for the SMT solver {\verit}. If requested
by the user, {\verit} outputs a proof if it can deduce that the input problem is
unsatisfiable. In proof production mode, {\verit} supports the theory of
uninterpreted functions, the theory of linear integer and real arithmetic, and
quantifiers.
The SMT solver {\cvcfive}~\cite{barbosa-2022} (the successor of
{\cvcfour}) supports Alethe experimentally as one of its multiple proof
output formats.
%
Alethe proofs can be reconstructed by the \tactic{smt} tactic of
the proof assistant {\isabelle}~\cite{fleury-2019,schurr-2021}.
The \tool{SMTCoq} tool can
reconstruct an older version of the format in the
proof assistant \tool{Coq}~\cite{SMTCoq}.  An effort to update the
tool to the latest version of Alethe is ongoing.
%
Furthermore, \tool{Carcara} is an experimental
high-performance \index{proof checker}proof checker written in Rust.\footnote{Available at
\url{https://github.com/ufmg-smite/carcara}.}

In addition to this reference, the proof format has been discussed in past
publications, which provide valuable background information.  The core of
the format goes back to 2011 when two publications at the PxTP workshop
outlined the fundamental ideas behind the format~\cite{besson-2011}
and proposed rules for quantifier instantiation~\cite{deharbe-2011}.
%
More recently, the format has gained support for reasoning typically used for
processing, such as skolemization, substitutions, and other manipulations of
bound variables~\cite{barbosa-2019}.

\subsection{Notations}

The notation used in this document is similar to the notation
used by the SMT-LIB standard.  The Alethe proof format uses
the SMT-LIB logic.  Since the SMT-LIB language is based on
S-expressions\index{S-expression}, SMT-LIB formulas are written using
a λ-calculus style.  That is, instead of $f(1,2)$, we write $(f\,1\,2)$.
However, connectives that are usually written using infix notation, also
use infix notation here.  That is, we write $t_1 \lor t_2$, not
$(\lor\,t_1\,t_2)$.

We use $x, y, z$ to indicate variables,
$f, g$ for functions, and $P, Q$ for predicates (functions with co-domain
sort $\lsymb{Bool}$.  To indicate terms we use $t, u$ and to indicate
formulas (terms of sort $\lsymb{Bool}$) we use $\varphi, \psi$.
To distinguish syntactic equality and the SMT-LIB equality predicate,
we write $=$ for the former, and $≈$ for the latter.
We will write pre-defined SMT-LIB symbols, such as sorts and functions,
in bold (e.g., $\lsymb{Bool}$, $\lsymb{ite}$).

We will use $θ$ to denote a substitution\index{substitution}.
The notation $[x₁ ↦ t₁, …, x_n ↦ t_n]$ denotes the substitution
that maps $x_i$ to $t_i$ for $1 ≤ i ≤ n$ and corresponds to the
identity function for all other variables.
%
If $θ$ and $η$ are two
substitutions, then $θη$ denotes the result of first applying $θ$
and then $η$ (i.e., $η(θ(.))$).  A substitution can naturally be extended
to a function that maps terms to terms by replacing the occurrences of
free variables.
%
The application of a substitution $θ$ to a term $t$ (i.e., $θ(t)$)
is capture-avoiding; bound variables in $t$ are renamed as necessary.

We write $t[u]$ for a term that contains the term $u$ as a subterm.  If $u$
is subsequently replaced by a term $v$, we write $t[v]$ for the
new term.
%
We also use this notation with multiple terms.
%
The notation $t[u_1, \dots, u_n]$ stands for a term may contain the
pairwise distinct terms $u_1, \dots, u_n$.
%
Then, $t[s_1,\dots, s_n]$ is the respective term where the
variables $u_1,\dots, u_n$ are
simultaneously replaced by $s_1,\dots, s_n$.  Usually, $u_1, \dots, u_n$ will
be variables.

Note that we will introduce the Alethe specific notation to write proof steps
in the following sections.

\section{The Alethe Language}
\label{sec:alethe:language}

This section provides an overview of the core concepts of the
\index{Alethe}Alethe language and also introduces some notation used
throughout this chapter.
The section first introduces an abstract notation to write Alethe proofs.
Then, it introduces the concrete, {\smtlib}-based syntax.  Finally,
we show how a concrete Alethe proof can be checked.

\begin{example}
The following example shows a simple Alethe proof
expressed in the abstract notation used in this document.
It uses quantifier instantiation and resolution to show a contradiction.
The paragraphs below describe the concepts necessary to
understand the proof step by step.

\begin{Alethe}
1.& \ctxsep & $\forall x.\, (P\,x)$ & $ \proofRule{assume}$ \\
2.& \ctxsep & $\neg (P\,a)        $ & $ \proofRule{assume}$ \\
3.& \ctxsep & $\neg (\forall x.\, (P\,x)) \lor (P\,a)$ & $\proofRule{forall_inst}\,[(x, a)]$ \\
4.& \ctxsep & $\neg (\forall x.\, (P\,x)), (P\,a)$ & $ (\proofRule{or}\:3)$ \\
5.& \ctxsep & $\bot             $ & $ (\proofRule{resolution}\: 1, 2, 4)$ \\
\end{Alethe}

\end{example}

\paragraph{Many-Sorted First-Order Logic.}
Alethe builds on the {\smtlib} language.
%
This includes its many-sorted first-order logic.
The available sorts depend on
the selected {\smtlib} theory/logic as well as on those defined by the user, but the
distinguished $\lsymb{Bool}$ sort is always available.
%
However, Alethe also extends this logic with Hilbert's choice
operator $\varepsilon$.
%
The term $\varepsilon x.\, \varphi[x]$ stands for a value $v$
such that $\varphi[v]$ is true if such a value exists. Any value is
possible otherwise.  Alethe requires that $\varepsilon$ is functional
with respect to logical equivalence: if for two formulas $\varphi$, $\psi$
that contain the free variable $x$, it holds that
$(\forall x.\,\varphi ≈ \psi)$,
then $(\varepsilon x.\, \varphi)\approx(\varepsilon x.\, \psi)$ must also hold.
Note that choice terms can only appear in Alethe proofs, not in {\smtlib} problems.
%\end{seppara}

\paragraph{Steps.}
A \index{proof}proof in the Alethe language is an indexed list of \index{step}steps.
To mimic the concrete syntax of Alethe proofs, proof steps in the
abstract notation have the form

\begin{AletheS}
$i$.& $c_1,\,\dots,\, c_j$ & \ctxsep &
$l_1,\dots ,l_k$ & $(\ruleType{rule}\; p_1,\,\dots,\, p_n)\,[a_1,\,\dots,\,a_m]$ \\
\end{AletheS}

\noindent
Each step has a unique index $i \in \mathbb{I}$, where $\mathbb{I}$ is a countable
infinite set of valid indices. In the concrete syntax all {\smtlib} symbols are
valid indices, but for examples we will use natural numbers.
%
Furthermore, $l_1, \dots ,l_k$ is a clause with the literals
$l_i$. It is the conclusion of the step. If a step has the empty clause
as its conclusion (i.e., $k = 0$) we write $\bot$.  While this
muddles the water a bit with regard to steps which have the unit clause
with the unit literal $\bot$ as their conclusion, it simplifies the
notation.  We will remark on the difference if it is relevant.  The rule
name \ruleType{rule} is taken from a set of possible proof
rules (see Section~\ref{apx:rules}).
Furthermore, each step has a possibly empty set of premises $\{p_1,
\dots, p_n\} \subseteq \mathbb{I}$,
and a rule-dependent and possibly
empty list of arguments $[a_1, \dots, a_m]$. The list of premises
only references earlier steps, such that the proof forms a directed
acyclic graph.  If the list of premises is empty, we will drop the
parentheses around the proof rule.
The arguments $a_i$ are either terms or tuples $(x_i,
t_i)$ where $x_i$ is a variable and $t_i$ is a term. The interpretation
of the arguments is rule specific.  The list $c_1, \dots, c_j$ is
the \index{context}{\em context} of the step.  Contexts are discussed below.
Every proof ends with a step that has the empty clause as the conclusion
and an empty context.  The list of proof rules
in Section~\ref{apx:rules}
also uses
this notation to define the proof rules.

The example above consists of five steps. Step~4 and~5 use premises.
Since step~3 introduces a tautology, it uses no premises.  However,
it uses arguments to express the substitution $[x\mapsto a]$ used to instantiate
the quantifier. Step~4 translates the disjunction into a clause.
In the example above, the contexts are all empty.

\paragraph{Assumptions.}
An \proofRule{assume} step introduces a term as an
\index{assumption}assumption. The proof starts with a number of
\proofRule{assume} steps. Each such step corresponds to an input
assertion. Within a subproof, additional assumptions can be introduced
too. In this case, each assumption must be discharged with an appropriate
step. The rule \proofRule{subproof} can be used to do so.  In the concrete
syntax, \proofRule{assume} steps have a dedicated command \inlineAlethe{assume}
to clearly distinguish them from normal steps that use the \inlineAlethe{step}
command (see~Section~\ref{sec:alethe:syntax}).

The example above uses two assumptions which are introduced in the first
two steps.


\paragraph{Subproofs and Lemmas.}
Alethe uses subproofs to prove lemmas and to create and manipulate the context.
%
To prove \index{lemma}lemmas, a \index{subproof}subproof can
introduce local assumptions. The \proofRule{subproof} {\em rule} discharges the
local assumptions.
From an
assumption $\varphi$ and a formula $\psi$ proved
from $\varphi$, the \proofRule{subproof} rule deduces the clause $\neg \varphi,
\psi$ that discharges the local assumption $\varphi$.
%
A \proofRule{subproof} step cannot use a premise from a subproof nested
within the current subproof.

Subproofs are also used to manipulate the context.
As the example below shows, the abstract notation denotes subproofs by a
frame around the steps in the subproof.  In this case the subproof
concludes with a step that does not use the \proofRule{subproof} rule,
but another rule, such as the \proofRule{bind} rule.

\begin{example}
This example shows a refutation of the
formula $(2 + 2) ≈ 5$. The proof uses a subproof to prove the
lemma $((2 + 2) ≈ 5) \Rightarrow 4 ≈ 5$.

\begin{Alethe}
1.&  \ctxsep &  $(2 + 2) ≈ 5$ &  $\proofRule{assume}$ \\
2.&  \spctxsep &  $(2 + 2) ≈ 5 $ &  $\proofRule{assume}$ \\
3.&  \spctxsep &  $(2 + 2) ≈ 4 $ &  $\proofRule{sum_simplify}$ \\
4.&  \spctxsep &  $4 ≈ 5       $ &  $(\proofRule{trans}\ 2, 3)$ \\
\spsep
5.&  \ctxsep &  $\neg((2 + 2) ≈ 5), 4 ≈ 5$ &  $\proofRule{subproof}$ \\
6.&  \ctxsep &  $(4 ≈ 5)≈ \bot$ &  $\proofRule{eq_simplify}$ \\
7.&  \ctxsep &  $\neg((4 ≈ 5)≈ \bot), \neg(4≈ 5), \bot $ &  $\proofRule{equiv_pos2}$ \\
8.&  \ctxsep &  $\bot $ &  $(\proofRule{resolution}\ 1, 5, 6, 7)$ \\
\end{Alethe}

\end{example}

\paragraph{Contexts.}
A specificity of the Alethe proofs
is the step \index{context}context.
Alethe contexts are a general mechanism to write
substitutions and to change them by attaching new elements.
A context is a possibly empty list $c_1,
\dots, c_l$, where each element is either a variable or a variable-term tuple
denoted $x_i\mapsto t_i$.
%
In the first case, we say that $c_i$ {\em fixes} the
variable.  The second case is a mapping.
Throughout this chapter, $\Gamma$ denotes
an arbitrary context.

Every context $\Gamma$ induces a capture-avoiding substitution
$\subst(\Gamma)$. If $\Gamma$ is the empty list,
$\subst(\Gamma)$ is the empty substitution, i.e, the
identity function.
The first case fixes $x_n$ and allows the context to shadow a previously defined
substitution for $x_n$:
\[
\subst([c_1,\dots, c_{n-1}, x_n])
\text{ is }\subst([c_1,\dots, c_{n-1}])\text{, but }x_n\text{ maps to }x_n.
\]

When $\Gamma$ ends in a mapping, the substitution is extended
with this mapping\label{page:ctxdef}:
\[
\subst([c_1,\dots, c_{n-1}, x_n\mapsto t_n]) =
        \subst([c_1,\dots, c_{n-1}])\circ \{x_n\mapsto t_n\}.
\]

\label{ex:alethe:substctx}The following example illustrates this idea.
\begin{align*}
    \subst([x\mapsto 7, x \mapsto g(x)]) &= \{x\mapsto g(7)\} \\
    \subst([x\mapsto 7, x, x \mapsto g(x)]) &= \{x\mapsto g(x)\} \\
\end{align*}

Contexts are used to express proofs about the preprocessing of terms.  The
conclusions of proof rules that use contexts always have the form
\begin{AletheS}
i. & $\Gamma$ & \ctxsep & $t ≈ u$ & ($\ruleType{rule}$, …) \\
\end{AletheS}

where the term $t$\, is the original term, and $u$ is the term after
preprocessing.  Tautologies with contexts correspond to judgments
$\vDash_T \subst(\Gamma)(t) ≈ u$.   Note that, some proof rules require
an empty context.  In the list of proof rules
in Section~\ref{apx:rules}
this is indicated by omitting
the $\Gamma$.

The substitution induced by $\Gamma$ is capture-avoiding.
Hence, some bound variables could be renamed in
$\subst(\Gamma)(t)$ with respect to the original term $t$.
A consequence of this is that steps that use a context must be checked
under α-equivalence.  The \proofRule{bind} rule can be used
to express renaming of bound variables explicitly.  The \proofRule{refl}
rule, on the other hand, can be exploited to directly rename bound variables
without an explicit proof.

Formally, the context can be translated to \index{abstraction!lambda}λ-abstractions and
applications.  This is discussed in Section~\ref{sec:alethe:semantics}.

\begin{example}\label{ex:ti:ctx-abstract}
This example shows a proof that uses a subproof with a context to rename a bound
variable.
\begin{AletheS}
1.& & \ctxsep & $\forall x.\,(P\,x)$ &        $\proofRule{assume}$ \\
2.& & \ctxsep & $\neg (\forall y.\,(P\,y))$ & $\proofRule{assume}$ \\
3.& \spctx{$y, x \mapsto y$} & \ctxsep & $x ≈ y$  & $\proofRule{refl}$\\
4.& \spctx{$y, x \mapsto y$} & \ctxsep & $(P\,x) ≈ (P\,y)$
     & $(\proofRule{cong}\,3) $\\
\spsep
5.& & \ctxsep & $\forall x.\,(P\,x) ≈ \forall y.\,(P\,y)$ & $\proofRule{bind}$ \\
6.& & \ctxsep &
      $\neg(\forall x.\,(P\,x) ≈ \forall y.\,(P\,y))$, $\neg(\forall x.\,(P\,x)), (\forall y.\,(P\,y))$
    & $\proofRule{equiv_pos2}$ \\
7.& &\ctxsep & $\bot$ & $(\proofRule{resolution}\,1, 2, 5, 6)$ \\
\end{AletheS}
\end{example}


\paragraph{Implicit Reordering of Equalities.}
In addition to the explicit steps, solvers might reorder equalities,
i.e., apply symmetry of the equality predicate, without generating steps.
The sole exception is the topmost equality in the conclusion of steps
with non-empty context.  The order of the arguments of this equality can
never change.
As described above, all rules that accept a non-empty context have
a conclusion of the form $t ≈ u$.  Since the context
represents a substitution applied to the left-hand side,
this equality symbol is not symmetric.

The SMT solver {\verit} currently applies this freedom in a restricted
form: equalities are reordered only
when the term below the equality changes during proof search. One such
case is the instantiation of universally quantified variables. If an
instantiated variable appears below an equality, then the equality might
have an arbitrary order after instantiation.
Nevertheless, consumers of Alethe must consider the possible
implicit reordering of equalities everywhere.


\subsection{The Syntax}\label{sec:alethe:syntax}

The concrete text representation of the Alethe proofs
is based on the {\smtlib} standard. Figure~\ref{fig:proof_ex} shows an
example proof as printed by {\verit} with light edits for readability.
%
The format broadly follows the {\smtlib} standard.
%
Input problems in the {\smtlib} format are scripts.  An {\smtlib} script
is a list of commands that manipulate the SMT solver.  For example,
\inlineAlethe{assert} introduces an assertion, \inlineAlethe{check-sat} starts
solving, and \inlineAlethe{get-proof} instructs the SMT solver to print the
proof.
%
Alethe mirrors this structure. Therefore, beside the {\smtlib}
logic and term language, it also uses commands to structure the proof.
An Alethe proof is a list of commands.

\begin{figure}[t]
    \begin{AletheVerb}
(assume h1 (not (p a)))
(assume h2 (forall ((z1 U)) (forall ((z2 U)) (p z2))))
...
(anchor :step t9 :args ((vr4 U) (:= (z2 U) vr4)))
(step t9.t1 (cl (= z2 vr4)) :rule refl)
(step t9.t2 (cl (= (p z2) (p vr4)))
         :rule cong :premises (t9.t1))
(step t9 (cl (= (forall ((z2 U)) (p z2))
                (forall ((vr4 U)) (p vr4))))
         :rule bind)
...
(step t14 (cl (forall ((vr5 U)) (p vr5)))
          :rule th_resolution :premises (t11 t12 t13))
(step t15 (cl (or (not (forall ((vr5 U)) (p vr5)))
                  (p a)))
          :rule forall_inst :args ((:= vr5 a)))
(step t16 (cl (not (forall ((vr5 U)) (p vr5))) (p a))
          :rule or :premises (t15))
(step t17 (cl) :rule resolution :premises (t16 h1 t14))
    \end{AletheVerb}
\caption{Example proof output. Assumptions are
  introduced;   a subproof renames bound variables; the proof finishes with
  instantiation and resolution steps.}
\label{fig:proof_ex}
\end{figure}


\begin{figure}[t]
\[
  \begin{array}{r c l}

\multicolumn{3}{c}{
	\text{A }\grNT{symbol}\text{ is an SMT-LIB }\grNT{symbol}\text{ that is not a}}\\
& & -\grNT{numeral}/\grNT{positive\_numeral},\\
& & -\grNT{numeral},\text{ or} -\grNT{decimal}.\\

 \grNT{proof}           &\grRule & \grNT{proof\_command}^{*} \\
 \grNT{proof\_command}  &\grRule & \textAlethe{(assume}\; \grNT{symbol}\; \grNT{proof\_term}\;\grNT{attribute}^{*}\,\textAlethe{)} \\
                        &\grOr   & \textAlethe{(step}\; \grNT{symbol}\; \grNT{clause}
                                        \; \textAlethe{:rule}\; \grNT{symbol} \\
                        &        & \quad \grNT{premises\_annotation}^{?} \\
                        &        & \quad \grNT{args\_annotation}^{?}\;\grNT{attribute}^{*}\,\textAlethe{)} \\
                        & \grOr  & \textAlethe{(anchor :step}\; \grNT{symbol}\;
                                            \\
                        &        & \quad \grNT{context\_annotation}^{?}\;\grNT{attribute}^{*}\,\textAlethe{)} \\
                        & \grOr  & \textAlethe{(define-fun}\; \grNT{function\_def}\,\textAlethe{)} \\
 \grNT{clause}          &\grRule & \textAlethe{(cl}\; \grNT{proof\_term}^{*}\,\textAlethe{)} \\
 \grNT{proof\_term}     &\grRule & \grNT{term}\text{ extended with } \\
                        &        & \textAlethe{(choice (}\, \grNT{sorted\_var}\,\textAlethe{)}\; \grNT{proof\_term}\,\textAlethe{)}  \\
                        & \grOr  & \grNT{rational} \\
                        & \grOr  & \grNT{nonpositive\_numeral} \\
                        & \grOr  & \grNT{nonpositive\_decimal} \\
 \grNT{premises\_annotation} &\grRule & \textAlethe{:premises (}\; \grNT{symbol}^{+}\textAlethe{)} \\
 \grNT{args\_annotation}     &\grRule & \textAlethe{:args}\,\textAlethe{(}\,\grNT{step\_arg}^{+}\,\textAlethe{)}  \\
 \grNT{step\_arg}            &\grRule & \grNT{symbol}\;\grOr\;
                                          \textAlethe{(}\; \grNT{symbol}\; \grNT{proof\_term}\,\textAlethe{)} \\
 \grNT{context\_annotation}  &\grRule & \textAlethe{:args}\,\textAlethe{(}\,\grNT{context\_assignment}^{+}\,\textAlethe{)}  \\
 \grNT{context\_assignment}  &\grRule & \grNT{sorted\_var} \\
                             & \grOr  & \textAlethe{(:=}\, \grNT{sorted\_var}\;\grNT{proof\_term}\,\textAlethe{)} \\
 \grNT{positive\_numeral}    &\grRule & \text{any }\grNT{numeral}\text{ except } 0\\
 \grNT{rational}              &\grRule & -^{?}\grNT{numeral}/\grNT{positive\_numeral} \\
 \grNT{nonpositive\_numeral}  &\grRule & -\grNT{numeral} \\
 \grNT{nonpositive\_decimal}  &\grRule & -\grNT{decimal} \\
  \end{array}
  \]
  \caption{The proof grammar.}
  \label{fig:grammar}
\end{figure}

Every Alethe proof is associated with an {\smtlib} problem
that is proved by the Alethe proof.  This can either be a concrete
problem file or, if the incremental solving commands of {\smtlib}
are used, the implicit problem constructed at the invocation of a
\inlineAlethe{get-proof} command.
In this document, we will call this {\smtlib} problem the {\em input problem}.
An Alethe proof inherits the namespace of its {\smtlib} problem.
All symbols declared or defined in the input problem remain declared or
defined, respectively.
Furthermore, the symbolic names introduced
by the \inlineAlethe{:named} annotation also stay valid and can be used
in the script.  For the purpose of the proof rules, terms are treated
as if proxy names introduced by \inlineAlethe{:named} annotations have been
expanded and annotations have been removed.  For example, the term
\inlineAlethe{(or (! (P a) :named baz) (! baz :foo))} and
\inlineAlethe{(or (P a) (P a))} are considered to be syntactically equal.
Here \inlineAlethe{:foo} is an arbitrary {\smtlib} annotation.

Figure~\ref{fig:grammar} shows the grammar of the proof text. It
is based on the {\smtlib} grammar, as defined in the {\smtlib}
standard~\cite[Appendix~B]{SMTLIB}.  The non-terminals
$\grNT{attribute}$,
$\grNT{function\_def}$,
$\grNT{sorted\_var}$, and
$\grNT{term}$
are as defined in the standard.
A special restriction applies to the $\grNT{symbol}$ non-terminal.
Alethe has an extended set of number literals.  Since these can start
with a negation sign, they overlap SMT-LIB's $\grNT{symbol}$ non-terminal.
For example, \inlineAlethe{-1} is a valid $\grNT{symbol}$ in SMT-LIB.
These sequences cannot be used as symbols when using Alethe.
Note that symbols are also used to name user defined constants and functions
in the input problem.  Hence, Alethe cannot express proofs about problems
that use such symbols.

Alethe proofs are a list of commands.
The \inlineAlethe{assume} command introduces a new assumption. While this
command could also be expressed using the \inlineAlethe{step} command with
a special rule, the special semantics of an assumption justifies the
presence of a dedicated command: assumptions are neither tautological nor
derived from premises.  The \inlineAlethe{step} command, on the other hand,
introduces a derived or tautological clause.  Both commands \inlineAlethe{assume}
and \inlineAlethe{step} require an index as the first argument to later
refer back to it. This index is an arbitrary {\smtlib} symbol.
It must be unique for each \inlineAlethe{assume} and \inlineAlethe{step} command.
A special restriction applies to the \inlineAlethe{assume} commands
not within a subproof, which reference assertions in the input {\smtlib} problem.  To simplify
proof checking, the \inlineAlethe{assume} command must use the name assigned
to the asserted formula if there is any.  For example, if the input
problem contains \inlineAlethe{(assert (! (P c) :named foo))}, then
the proof must refer to this assertion (if it is needed in the proof) as
\inlineAlethe{(assume foo (P c))}.  Note that an {\smtlib} problem can
assign a name to a term at any point, not only at its first occurrence.
If a term has more than one name, any can be picked.

The second argument of \inlineAlethe{step} and \inlineAlethe{assume}
is the conclusion of the command.
For a \inlineAlethe{step}, this term is always a clause.
To express disjunctions in {\smtlib} the \inlineAlethe{or} operator is used.
This operator, however, needs at least two arguments and cannot
represent unary or empty clauses. To circumvent this, we introduce a new
\inlineAlethe{cl} operator.  It corresponds to the standard \inlineAlethe{or}
function
extended to one argument, where it is equal to the identity, and zero
arguments, where it is equal to \inlineAlethe{false}.
Every step must use the \inlineAlethe{cl} operator, even if its conclusion
is a unit clause.
The \inlineAlethe{anchor} and \inlineAlethe{define-fun} commands are used for
subproofs and sharing, respectively. The \inlineAlethe{define-fun} command
corresponds exactly to the \inlineAlethe{define-fun} command of the
{\smtlib} language.

Furthermore, the syntax uses annotations as used by {\smtlib}.  The
original {\smtlib} syntax uses the non-terminal $\grNT{attribute}$.
The Alethe syntax uses some predefined annotation.  To simplify parsing,
the order in which those must be printed is strict.
The \inlineAlethe{:premises} annotation denotes the premises and is skipped
if the rule does not require premises. If the rule carries arguments, the
\inlineAlethe{:args} annotation is used to denote them.
Anchors have two annotations: \inlineAlethe{:step} provides the name of the
step that concludes the subproof and \inlineAlethe{:args} provides the context
as sorted variables and assignments.  Note that in this annotation,
the $\grNT{symbol}$ non-terminal is intended to be a variable.
After those pre-defined annotations, the solver can use additional
annotations.  This can be used for debugging, or other purposes.
A consumer of Alethe proofs {\em must} be able to parse proofs
that contain such annotations.

\paragraph{Terms}
The non-terminal $\grNT{proof\_term}$ is an extended version of the {\smtlib}
non-terminal $\grNT{term}$.
First, it has an additional production for the
\inlineAlethe{choice}\index{choice} binder.
%
Second, it has productions to express rationals and negative integers
concisely.
%
A difficulty when parsing {\smtlib} terms is that numerical constants
are not easy to distinguish from general terms.
For example, $-\frac{1}{2}$ is written as
\inlineAlethe{(/ (- 1) 2)}.
The $\grNT{rational}$ non-terminal makes it possible to write this constant
as a single literal: \inlineAlethe{-1/2}.
Furthermore, the non-terminals $\grNT{nonpositive\_numeral}$ and $\grNT{nonpositive\_decimal}$
achieve the same for unary negation.

The sorting rules for $\grNT{proof\_term}$ are as for {\smtlib} terms with
one key difference.
The sort of terms produced by $\grNT{rational}$, $\grNT{decimal}$, and
$\grNT{nonpositive\_decimal}$ is always \inlineAlethe{Real}.
The sort of terms produced by $\grNT{integer}$ and $\grNT{nonpositive\_numeral}$
is always \inlineAlethe{Int}.
For example, in standard {\smtlib}, the term \inlineAlethe{(+ 5 3)}
has the sort \inlineAlethe{Int} in the logic \texttt{QF\_LIA}, but
has the sort \inlineAlethe{Real} in \texttt{QF\_LRA}.
In Alethe, this term always has the sort \inlineAlethe{Int}.

\paragraph{Subproofs}
The abstract notation denotes subproofs by marking them with a vertical
line.  To map this notation to a list of commands, Alethe \index{anchor}uses the
\inlineAlethe{anchor} command.  This command indicates the start of a subproof.
A subproof is concluded by a matching \inlineAlethe{step} command.  This step must use
a \index{rule!concluding}{\em concluding rule}
(such as \proofRule{subproof}, \proofRule{bind}, and so forth).

After the \inlineAlethe{anchor} command, the proof uses
\inlineAlethe{assume} commands to list the assumptions of the subproof.
Subsequently, the subproof is a list of \inlineAlethe{step} commands
that can use prior steps in the subproofs as premises.  It is not allowed
to issue \inlineAlethe{assume} commands after the first \inlineAlethe{step}
command of a subproof has been issued.

In the abstract notation, every step is associated with a context.  The
concrete syntax uses anchors to optimize this.
The \index{context}context is manipulated in a nested way: if a step
pops $c_1,\dots, c_n$ from a context $\Gamma$, there is an earlier step which
pushes precisely $c_1,\dots, c_n$ onto the context.
%
Since contexts can only be manipulated by push and pop, context manipulations
are nested.  The \inlineAlethe{anchor} commands push onto the context and the
corresponding \inlineAlethe{step} commands pop from the context.
%
To indicate these changes to the context, every anchor is associated with a list
of fixed variables and mappings.  The list is provided by the \inlineAlethe{:args}
annotation.  If the list is empty, the \inlineAlethe{:args} annotation is
omitted\footnote{The only rule that allows an empty list is the
\proofRule{subproof} rule.  Since this rule corresponds to implication introduction,
it does not interact with binders.}.
%
Note that, when an \inlineAlethe{anchor} command extends a context $\Gamma$ with
some mappings $x_1 \mapsto t_1, \dots,  x_n \mapsto t_n$,
then the terms $t_i$ are normalized by applying
the substitution $\subst(\Gamma)$ to $t_i$.  This is because the definition
on page~\pageref{page:ctxdef} extends the context by composing the substitutions.

The \inlineAlethe{:step} annotation of the anchor command is used to indicate
the \inlineAlethe{step} command that will end the subproof.  The clause of
this \inlineAlethe{step} command is the conclusion of the subproof.
While it is possible to infer the
step that concludes a subproof from the structure of the proof, the explicit
annotation simplifies proof checking and makes the proof easier to read.
If the subproof uses a
context, the \inlineAlethe{:args} annotation of the \inlineAlethe{anchor} command
indicates the arguments added to the context for this subproof.  The
annotation provides the sort of fixed variables.

In the example proof (Figure~\ref{fig:proof_ex}) a
subproof starts at the \inlineAlethe{anchor} command.
It ends with the \proofRule{bind} steps that finishes the
proof for the renaming of the bound variable \inlineAlethe{z2}
to \inlineAlethe{vr4}.

A further restriction applies: only the conclusion of a subproof can be used
as a premise outside the subproof. Hence, a proof checking tool can
remove the steps of the subproof from memory after checking it.

\begin{example}
\label{ex:ti:ctx-concrete}
This example shows the proof from Example~\ref{ex:ti:ctx-abstract}
expressed as a concrete proof.

\begin{AletheVerb}
(assume h1 (forall ((x S)) (P x)))
(assume h2 (not (forall ((y S)) (P y))))
(anchor :step t5 :args ((y S) (:= (x S) y)))
(step t3 (cl (= x y)) :rule refl)
(step t4 (cl (= (P x) (P y))) :rule cong :premises (t3))
(step t5 (cl (= (forall ((x S)) (P x)) (forall ((y S)) (P y))))
                              :rule bind)
(step t6 (cl (not (= (forall ((x S)) (P x)) (forall ((y S)) (P y))))
             (not (forall ((x S)) (P x)))
             (forall ((y S)) (P y))) :rule equiv_pos2)
(step t7 (cl) :rule resolution :premises (h1 h2 t5 t6))
\end{AletheVerb}
\end{example}

\tikzset{
     solver/.style={draw, thin},
     system/.style={draw, thin, rounded corners},
}

\begin{figure}[h]
\center
\begin{tikzpicture}[node distance=2cm, auto,>=latex', thick,scale=0.8]
    \node[solver] (unsat) {\textsf{Unsat mode}};
    \node[system, right=of unsat] (assume) {\textsf{Assumptions}};
    \path[->] (unsat) edge[bend left] node[font=\scriptsize] {\texttt{get-proof}} (assume);
    \path[->] (assume) edge[loop above] node[align=center,font=\scriptsize] {\inlineAlethe{assume},\\\inlineAlethe{define-fun},\\\inlineAlethe{anchor}} (assume);
    \node[system, right=of assume] (step) {\textsf{Steps}};
    \path[->] (assume) edge[bend left] node[font=\scriptsize] {\inlineAlethe{step}} (step);
    \path[->] (step) edge[loop above] node[align=center,font=\scriptsize] {\inlineAlethe{step},\\ \inlineAlethe{define-fun}} (step);
    \path[->] (step) edge[above, bend left] node[font=\scriptsize] {\inlineAlethe{anchor}} (assume);
    \path[->] (step) edge[above, bend left] node[font=\scriptsize] {\textsf{Last step}} (unsat);
\end{tikzpicture}
\label{fig:proof-states}\caption{Abstract view of the transitions in an Alethe proof.}
\end{figure}


\paragraph{Alethe Proof Printing States}
Figure~\ref{fig:proof-states} shows the states of an Alethe proof
abstractly.  To generate a proof, the SMT solver must be in the
{\em Unsat mode}, i.e., the solver determined that the problem
is unsatisfiable.  The {\smtlib} problem script then requests the proof by
invoking the \inlineAlethe{get-proof} command.  It is possible that this command
fails. For example, if proof production was not activated up front.
If there is no error, the proof is printed and printing starts with
the assumptions.  The solver prints the proof as a list of commands
according to the state.  The states ensure one constraint is maintained:
assumptions can only appear at either the beginning of a proof or right
after a subproof is started by the \inlineAlethe{anchor} command.  They cannot
be mixed with ordinary proof steps.  This simplifies
reconstruction.  Each assumption printed at the beginning of the proof
corresponds to assertions in the input problem, up to symmetry of equality.
Proof printing concludes after the last step is printed and the solver
returns to the Unsat mode and the user can issue further commands.
Usually the last step is an outermost step
(i.e., not within a subproof) that concludes the proof by deriving
the empty clause, but this is not necessary.
The solver is allowed to print some additional, useless,
steps after the proof is concluded.


\paragraph{Sharing and Skolem Terms}
Usually, SMT solvers store terms internally in an efficient manner.  A term data
structure with perfect sharing ensures that every term is stored in memory
precisely once. When printing the proof, this compact storage is unfolded.
This leads to a blowup of the proof size.

Alethe can optionally use sharing\footnote{For {\verit} this can be activated
by the command-line option \Verb{--proof-with-sharing}.} to print common
subterms only once.  This is realized using the standard naming mechanism
of {\smtlib}. A term $t$ is annotated with a name $n$ by writing
\inlineAlethe{(!}\,$t$\,\inlineAlethe{:named}\;$n$\,\inlineAlethe{)}
where $n$ is a symbol. After a term is annotated with a name, the name can
be used in place of the term. This is a purely syntactical replacement.
Alethe continues to use the names already used in the input problem.
Hence, terms that already have a name in the input problem can be replaced
by that name and new names introduced in the proof must not use names
already used in the input problem.

To limit the number of names, an SMT solver can use the following
simple approach used by {\verit}.
Before printing the proof, it iterates over all terms of the proof and
recursively descend into the terms. It marks every unmarked subterm it
visits. If a visited term is already marked, the solver assigns a new name
to this term.  If a term already has a name, it does not descend further
into the term.
%
By doing so, it ensures that only terms that appear as
child of two different parent terms get a name.  Since a named term
is replaced with its name after it first appearance, a term that only
appears as a child of one single term does not need a distinct name.
%
Thanks to the perfect
sharing representation, testing if a term is marked takes constant time
and the overall traversal takes linear time in the proof size.

To simplify reconstruction, Alethe can optionally\footnote{For {\verit} by
using the command-line option \Verb{--proof-define-skolems}.} define
Skolem constants as functions. In this case, the proof contains a list
of \inlineAlethe{define-fun} commands that define shorthand 0-ary functions for
the \inlineAlethe{(choice }\dots\inlineAlethe{)} terms needed. Without this option,
no \inlineAlethe{define-fun} commands are issued, and the constants are expanded.

\paragraph{Implicit Transformations}
Overall, the following aspects are treated implicitly by Alethe.
\begin{itemize}
  \item Symmetry of equalities that are not top-most equalities in steps with
        non-empty context.
  \item The order of literals in the clauses.
  \item The unfolding of names introduced by
     \inlineAlethe{(!}\,$t$\,\inlineAlethe{:named }\;$s$\,\inlineAlethe{)} in the
     original {\smtlib} problem or in the proof.
  \item The removal of other {\smtlib} annotations of the form
     \inlineAlethe{(!}\,$t$\,\dots\,\inlineAlethe{)}.
  \item The unfolding of function symbols introduced by
  \inlineAlethe{define-fun}.\footnote{For {\verit} this is only used when the user
  introduces {\verit} to print Skolem terms as defined functions. User defined
  functions in the input problem are not supported by {\verit} in proof production
  mode.}
  \item If the input problem is in a logic without integers, then constants from
        $\grNT{numeral}$ in the input problem will be printed using
        $\grNT{decimal}$ or $\grNT{rational}$ in the proof.
\end{itemize}

\noindent
Alethe proofs contain steps for other aspects that are commonly left implicit, such
as renaming of bound variables, and the application of substitutions.

\section{Checking Alethe Proofs}
\label{sec:alethe:semantics}

In this section we present an abstract procedure to check if an Alethe
proof is \index{well-formed}well-formed and valid.  An Alethe proof is
well-formed only if its anchors and steps are balanced.  To check that
this is the case, we replace innermost subproofs by holes until there is
no subproof left.  If the resulting reduced proof is free of anchors and steps
that use concluding rules, then the overall proof is well-formed.
To check if a proof is valid we have to check if all steps of a subproof
adhere to the conditions of
their rules before replacing the subproof by a hole.
If all subproofs are valid and all steps in the reduced
proof adhere to the conditions of their rule,
then the entire proof is valid.

Formally, an Alethe proof $P$ is a list $[C_1, \dots, C_n]$ of steps
and anchors.
Since every step uses an unique index, we assume that each step
$C_i$ in $P$ uses $i$ as its index.
The context
only changes at anchors and subproof-concluding steps.
Therefore, the elements of $C_1, \dots, C_n$ that are steps
are not associated with a context.
Instead, the context can be computed
from the prior anchors.
The anchors only ever extend the context.

To check an Alethe proof we can iteratively eliminate the first-innermost
subproof, i.e., the innermost subproof that does not come after a
complete subproof.  The restriction to the first subproofs simplifies
the calculation of the context of the steps in the subproof.

\begin{definition}[First-Innermost Subproof]
  Let $P$\, be the proof $[C_1, \dots, C_n]$ and $1 \leq \mathit{start}
  < \mathit{end} \leq n$ be two indices such that
  \begin{itemize}
    \item $C_{\mathit{start}}$ is an anchor,
    \item $C_{\mathit{end}}$ is a step that uses a concluding rule,
    \item no $C_k$ with $k < \mathit{start}$ uses a concluding rule,
    \item no $C_k$ with $\mathit{start} < k < \mathit{end}$ is an anchor or
     a step that uses a concluding rule.
  \end{itemize}
  Then $[C_{\mathit{start}}, \dots, C_{\mathit{end}}]$ is the
  \index{subproof!first-innermost}first-innermost subproof of $P$.
\end{definition}

\begin{example}
The proof in Example~\ref{ex:ti:ctx-concrete} has only one subproof
and this subproof is also a first-innermost subproof.  It is the subproof
\begin{AletheVerb}
(anchor :step t5 :args ((y S) (:= (x S) y)))
(step t3 (cl (= x y)) :rule refl)
(step t4 (cl (= (P x) (P y))) :rule cong :premises (t3))
(step t5 (cl (= (forall ((x S)) (P x)) (forall ((y S)) (P y))))
                              :rule bind)
\end{AletheVerb}
\end{example}

\noindent
It is easy to calculate the context of the first-innermost subproof.

\begin{definition}[Calculated Context]
  Let $[C_{\mathit{start}}, \dots, C_{\mathit{end}}]$ be
  the first-innermost subproof of $P$.
  Let $A_1, \dots, A_m$ be the anchors among $C_1, \dots, C_{\mathit{start}-1}$.

  The \index{context!calculated}calculated context of $C_i$ is the context
  \[
  c_{1,1}, \dots, c_{1, n_1}, \dots, c_{m,1}, \dots, c_{m, n_m}
  \]
  where $c_{k,1}, \dots, c_{k, n_k}$ is the list of fixed variables
  and mappings associated with $A_k$.
\end{definition}

Note that if $C_i$ is an anchor, its calculated context does not
contain the elements associated with $C_i$.
Therefore, the context of $C_{\mathit{start}}$
is the context of the steps before the subproof.
Furthermore, the step $C_{\mathit{end}}$ is the concluding
step of the subproof and must have the same context as the steps surrounding
the subproof.
Hence, the context of $C_{\mathit{end}}$ is the calculated
context of $C_{\mathit{start}}$.

\begin{example}
The calculated context of the steps \texttt{\color{SmtStepId}t3} and \texttt{\color{SmtStepId}t5} in
Example~\ref{ex:ti:ctx-concrete} is the context $x \mapsto y$.
The calculated context of the concluding step \texttt{\color{SmtStepId}t5} and the anchor is empty.
\end{example}

\noindent
A first-innermost subproof is valid if all its steps adhere to
the conditions of their rule and only use premises that occur before them
in the subproof.  The conditions of each rule are listed in
Section~\ref{apx:rules}.

\begin{definition}[Valid First-Innermost Subproof]
  Let $[C_{\mathit{start}}, \dots, C_{\mathit{end}}]$
  be the first-innermost subproof of $P$.
  The subproof is \index{subproof!valid}{\em valid} if
  \begin{itemize}
    \item all steps $C_i$ with $\mathit{start} < i <
    \mathit{end}$ only use premises $C_j$ with $\mathit{start} <
    j < i$,
    \item all $C_i$ that are steps adhere to the conditions of their
    rule under the calculated context of $C_i$,
    \item the step $C_{\mathit{end}}$
    adheres to the conditions of its
    rule under the calculated context of $C_{\mathit{start}}$.
  \end{itemize}
\end{definition}

The only rule that can discharge assumptions in a subproof is the
\proofRule{subproof} rule.  Therefore, an admissible subproof can only
contain \proofRule{assume} step if $C_{\mathit{end}}$
is the \proofRule{subproof} rule.

To eliminate a subproof we can replace the subproof with a hole that has
at its conclusion the conclusion of the subproof.  This is safe as long
as the subproof that is eliminated is valid (see Section~\ref{sec:alethe:soundness-eh}).

\begin{definition}
The function $E$ eliminates the first-innermost subproof from a proof
if there is one.
Let $P$ be a proof $[C_1, \dots C_n]$.
Then $E(P) = P$ if $P$ has no first-innermost subproof.
Otherwise, $P$ has the first-innermost subproof
$[C_{\mathit{start}}, \dots, C_{\mathit{end}}]$, and
$E(P) = [C_1, \dots, C_{\mathit{start}-1}, C', C_{\mathit{end}+1},
\dots, C_n]$ where $C'$ is a new step that uses the \proofRule{hole} rule
and has the index, conclusion, and premises of $C_{\mathit{end}}$.
\end{definition}

It is important to add the premises of $C_{\mathit{end}}$
to $C'$.  The \proofRule{let} rule can use additional premises
and omitting those premises results in an unsound step.
We can apply $E$ iteratively to a proof $P$ until we reach the least fixed point.
Since $P$ is finite we will always reach a fixed point in finitely many steps.
The result is a list $[P_0, P_1, P_2, \dots, P_{\mathit{last}}]$ where $P_0 = P$,
$P_1 = E(P)$, $P_2 = E(E(P))$ and $P_{\mathit{last}} = E(P_{\mathit{last}})$.

\begin{example}
Applying $E$\, to the proof in
Example~\ref{ex:ti:ctx-concrete} gives us the proof

\begin{AletheVerb}
(assume h1 (forall ((x S)) (P x)))
(assume h2 (not (forall ((y S)) (P y))))
(step t5 (cl (= (forall ((x S)) (P x)) (forall ((y S)) (P y))))
                                      :rule hole)
(step t6 (cl (= (forall ((x S)) (P x)) (forall ((y S)) (P y)))
             (not (forall ((x S)) (P x)))
             (forall ((y S)) (P y)))) :rule equiv_pos2)
(step t7 (cl) :rule resolution :premises (h1 h2 t5 t6))
\end{AletheVerb}

Since this proof contains no subproof, it is also $P_{\mathit{last}}$.
\end{example}


\begin{definition}[Well-Formed Proof]
  \label{def:well_formed_proof}
  The Alethe proof $P$ is \index{proof!well-formed}well-formed
  if every step uses a unique index and $P_{\mathit{last}}$
  contains no anchor or step that uses a concluding rule.
\end{definition}

\begin{definition}[Valid Alethe Proof]
  The proof $P$ is a \index{proof!valid}{\em valid Alethe proof} if
  \begin{itemize}
    \item $P$\, is well-formed,
    \item $P$\, does not contain any step that uses the \proofRule{hole} rule,
    \item $P_{\mathit{last}}$ contains a step that has the empty clause as its conclusion,
    \item the first-innermost subproof of every $P_i$, $i < \mathit{last}$ is valid,
    \item all steps $C_i$ in $P_{\mathit{last}}$ only use premises
          $C_j$ in $P_{\mathit{last}}$ with $1 \leq j < i$,
    \item all steps $C_i$ in $P_{\mathit{last}}$ adhere to the conditions of their
    rule under the empty context.
  \end{itemize}
\end{definition}

The condition that $P$ contains no hole ensures that the original
proof is complete and holes are only introduced by eliminating valid
subproofs.

\begin{example}
The proof in Example~\ref{ex:ti:ctx-concrete} is valid.  The only
subproof is valid, the proof contains no hole, and $P_{\mathit{last}}$
contains the step \textsf{\color{SmtStepId} t7} that concludes with the empty clause.
\end{example}

It is sometimes useful to speak about the steps that are not within a
subproof.  We call such a step an \index{step!outermost}{\em outermost
step}.  In a well-formed proof those are the steps
of $P_{\mathit{last}}$.

\subsection{Contexts and Metaterms}

We now direct our attention to subproofs with \index{context}contexts.
It is useful to give precise semantics
to contexts to have the tools to check that rules that use contexts
are sound.
Contexts are local in the sense that they affect only the
proof step they are applied to.
For the full details on contexts see~\cite{barbosa-2019}.
The presentation here is adapted from this publication, but omits some
details.

To handle subproofs with contexts, we translate the contexts into
λ-terms.
%
This allows us to
leverage the \index{lambda calculus}λ-calculus as an existing well-understood theory of \index{binder}binders.
%
These λ-terms\index{term!lambda} are called \index{term!meta}\index{metaterms}{\em metaterms}.

\begin{definition}[Metaterm]
  Metaterms are expressions constructed by the grammar
  \[
    M \,::=\, \groundbox{$t$}\,\mid\, \lambda x.\,M\,\mid\,(\lambda \bar{x}_n.\,
    M)\,\bar{t}_n
  \]
where $t$ is an ordinary term and $t_i$ and $x_i$ have matching sorts for
all $0 \leq i \leq 1$.
\end{definition}

According to this definition, a metaterm is either a boxed term, a
\index{abstraction!lambda}λ-abstraction, or an application to an uncurried λ-term.
The annotation $\groundbox{$t$}$ delimits terms from the context, a simple
λ-abstraction is used to express fixed variables, and the
application expresses simulations substitution of $n$ terms.\footnote{
  The box annotation used here is unrelated to the boxes
  within the SMT solver discussed in the introduction.}

We use $=_{\alpha\beta}$ to denote syntactic equivalence modulo
α-equivalence and β-reduction.

Proof steps with contexts can be encoded into proof steps with empty
contexts, but with metaterms.  A proof step

\begin{AletheS}
i. & $\Gamma$ & \ctxsep & $t ≈ u$ & $(\ruleType{rule}\; \bar{p}_n)\;[\bar{a}_m]$ \\
\end{AletheS}

\noindent
is encoded into

\begin{AletheS}
i. &  & \ctxsep & $L(\Gamma)[t] ≈ R(\Gamma)[u]$ & $(\ruleType{rule}\; \bar{p}_n)\;[\bar{a}_m]$ \\
\end{AletheS}

\noindent
where
\begin{align*}
 L(\emptyset)[t]    &= \groundbox{t}             &    R(\emptyset)[u] &= \groundbox{u} \\
 L(x, \bar{c}_m)[t] &= \lambda x.\,L(\bar{c}_m)[t] & R(x, \bar{c}_m)[u] &= \lambda x.\,R(\bar{c}_m)[u] \\
 L(\bar{x}_n\mapsto\bar{s}_n, \bar{c}_m)[t] &= (\lambda \bar{x}_n.\,L(\bar{c}_m)[t]) \bar{s}_n
     & R(\bar{x}_n\mapsto\bar{s}_n, \bar{c}_m)[u] &= R(\bar{c}_m)[u] \\
\end{align*}

To achieve the same effect as using the $\subst()$ function described above, we
can translate the terms into metaterms, perform β-reduction, and rename
bound variables if necessary~\cite[Lemma~11]{barbosa-2019}.

\begin{example}
The example on page~\pageref{ex:alethe:substctx} becomes
\begin{flalign*}
\quad & L(x\mapsto 7, x \mapsto g(x))[x] = (\lambda x.\,(\lambda x.\,\groundbox{x})\,(g(x)))\, 7
=_{\alpha\beta} \groundbox{g(7)} &\\
& L(x\mapsto 7, x, x \mapsto g(x))[x] = (\lambda x.\,\lambda x.\,(\lambda x.\,\groundbox{x})\,(g(x)))\, 7
=_{\alpha\beta} \lambda x.\,\groundbox{g(x)}
\end{flalign*}
\end{example}

Most proof rules that operate with contexts can easily be translated into
proof rules using metaterms.  The exception are the tautologous rules,
such as \proofRule{refl} and the $\cdots{}${\ruleType{_simplify}} rules.

Steps that use such rules always encode a judgment
$\vDash \Gamma\,\vartriangleright\,t ≈ u$.  With the encoding described above
we get $L(\Gamma)[t] ≈ R(\Gamma)[u]
=_{\alpha\beta} \lambda \bar{x}_n.\,\groundbox{t'} ≈
\lambda \bar{x}_n.\,\groundbox{u'}$ with some terms $t'$, $u'$.
To handle those terms, we use the $\reify()$ function.
This function is defined as
\[
\reify(\lambda \bar{x}_n.\,\groundbox{t} ≈
\lambda \bar{x}_n.\,\groundbox{u}) = \forall \bar{x}_n.\,(t ≈ u).
\]
Therefore,
all tautological rules with contexts represent a judgment\\
$\vDash \reify(T ≈ U)$
where $T =_{\alpha\beta} L(\Gamma)[t]$ and $U =_{\alpha\beta} R(\Gamma)[u]$.

\begin{example}
  Consider the step

\begin{AletheS}
i. & $y, x \mapsto y$ & \ctxsep
& $x + 0 ≈ y$ & $\proofRule{sum_simplify}$ \\
\end{AletheS}

\noindent
Translating the context into metaterms leads to

\begin{AletheS}
i. & \phantom{$y, x \mapsto y$} & \ctxsep
& $(\lambda y.\,(\lambda x.\,\groundbox{$x + 0$})\, y) ≈
        (\lambda y.\,\groundbox{$y$})$ & $\proofRule{sum_simplify}$ \\
\end{AletheS}

\noindent
Applying β-reduction leads to

\begin{AletheS}
i. & \phantom{$y, x \mapsto y$} & \ctxsep
& $(\lambda y.\,\groundbox{$y + 0$}) ≈
        (\lambda y.\,\groundbox{$y$}) $ & $\proofRule{sum_simplify}$ \\
\end{AletheS}

\noindent
Finally, using $\reify()$ leads to

\begin{AletheS}
i. & \phantom{$y, x \mapsto y$} & \ctxsep
& $\forall y.\,(y + 0 ≈  y)$ & $\proofRule{sum_simplify}$ \\
\end{AletheS}

\noindent
This obviously holds in the theory of linear arithmetic.
\end{example}

\subsection{Soundness}
\label{sec:alethe:soundness-eh}

Any proof calculus should be sound.  In the case of Alethe, most proof
rules are
standard rules, or simple tautologies.  The rules that use context
are unusual, but those proof rules were previously shown to be
sound~\cite{barbosa-2019}.
Alethe does not use any rules that are merely satisfiability preserving.
The skolemization rules replace the bound variables with choice terms
instead of fresh symbols.\footnote{The \inlineAlethe{define-fun} function
can introduce fresh symbols, but we will assume here that those
commands have been eliminated by unfolding the definition.}
All Alethe rules express semantic implications.
Overall, we assume in this document that the proof rules and proofs
written in the abstract notation are sound.

Nevertheless, a modest gap remains.  The concrete, command-based
syntax does not precisely correspond to the abstract notation.
In this section we will address the soundness of concrete Alethe
proofs.

\begin{theorem}[Soundness of Concrete Alethe Proofs]
  \label{thm:sound}
  If there is a valid Alethe proof $P = [C_1, \dots, C_n]$ that has the formulas
  $\varphi_1, \dots, \varphi_m$ as the conclusions of the outermost \proofRule{assume}
  steps, then
  \[
  \varphi_1, \dots, \varphi_m \vDash \bot.
  \]
\end{theorem}

Here, $\vDash$ represents
semantic consequence in the many-sorted first order logic of {\smtlib}
with the theories of uninterpreted functions and linear arithmetic extended
with the choice operator and  clauses.

To show the soundness of a valid Alethe proof $P = [C_1, \dots, C_n]$,
we can use the same approach as for the definition of validity: consider
first-innermost subproof first and then replace them by holes.
%
Since valid proofs do not contain holes, we have to generalize the induction
to allow holes that were introduced by the elimination of subproofs.
%
We start with simple subproofs with empty contexts and without
nested subproofs.

\begin{lemma}
  \label{lem:sound_subproof}
  Let $P$\, be a proof that contains a valid first-innermost subproof where
  $C_{\mathit{end}}$ is a \proofRule{subproof} step.  Let
  $\psi$ be the conclusion of $C_{\mathit{end}}$.
  Then $\vDash \psi$ holds.
\end{lemma}
\begin{proof}
  First, we use induction on the number of steps $n$ after
  $C_{\mathit{start}}$.  Let $\psi_n$ be the conclusion
  of $C_{\mathit{start}+n}$ and $V_n$ the conclusions of
  the \proofRule{assume} steps in $[C_{\mathit{start}}, \dots,
  C_{\mathit{start}+n}]$.
  Assumptions always introduce unit clauses.  We will identify
  unit clauses with their single literal.
  We will
  show $V_n \vDash \psi_n$ if $\mathit{start}+n < \mathit{end}$.

  If $n = 1$, then $C_{\mathit{start}+n} = C_{\mathit{start}+1}$ must
  either be a tautology, or an \proofRule{assume} step.  In the first case,
  $\vDash \psi_{\mathit{start}+1}$ holds, and in the second case
  $\psi_{\mathit{start}+1} \vDash \psi_{\mathit{start}+1}$ holds.

  For subsequent $n$, $C_{\mathit{start}+n}$ is
  either an ordinary step, or an \proofRule{assume} step.  In the second case,
  $\psi_{\mathit{start}+n} \vDash \psi_{\mathit{start}+n}$ which can
  be weakened to $V_n \vDash \psi_{\mathit{start}+n}$.
  In the first case, the step $C_{\mathit{start}+n}$ has a set of
  premises $S$.
  For each step $C_{\mathit{start}+i} \in S$ we have $i < n$ and
  $V_i \vDash \psi_{\mathit{start}+i}$ due to the induction
  hypothesis. Since $i < n$, the premises $V_i$ are a subset of $V_n$ and
  we can weaken $V_i \vDash \psi_{\mathit{start}+i}$
  to $V_n \vDash \psi_{\mathit{start}+i}$.  Since all premises of
  $C_{\mathit{start}+n}$ are the consequence of $V_n$ we get
  $V_n \vDash \psi_n$.

  The step $C_{\mathit{end}-1}$ is the last step of the subproof that
  does not use a concluding rule.
  At this step we have $V_{\mathit{end}-1} \vDash \psi_{\mathit{end}-1}$.
  Since $C_{\mathit{end}}$ is not an \proofRule{assume} step, the
  set $V_{\mathit{end}-1} = \{\varphi_1, \dots, \varphi_m\}$ contains
  all assumptions in the subproof.
  By the deduction theorem we get
  \[
  \vDash \varphi_1 \land \cdots \land \varphi_m → \psi_{\mathit{end}-1}.
  \]
  This can be transformed into the clause
  \[
  \vDash \neg\varphi_1, \cdots, \neg\varphi_m, l_1, \dots, l_o.
  \]
  where $l_1, \dots, l_o$ are the literals of $\psi_{\mathit{end}-1}$.
  This clause is exactly the conclusion of the
  step $C_{\mathit{end}}$
  according to the definition of the \proofRule{subproof} rule.
\end{proof}

\noindent
We can do the same reasoning as for Lemma~\ref{lem:sound_subproof} for
subproofs with contexts.  This is slightly complicated by the
\proofRule{let} rule that can use extra premises.

\begin{lemma}\label{lem:sound_subproof_context}
  Let $P$ be a proof that contains a
  valid first-innermost subproof where
  $C_{\mathit{end}}$ is a step using one of:
  \proofRule{bind}, \proofRule{sko_ex}, \proofRule{sko_forall},
  \proofRule{onepoint}, \proofRule{let}.

  Then $V \vDash \Gamma \vartriangleright \psi$ where $\Gamma$ is the
  calculated context of $C_{\mathit{start}}$ and $\psi$ is the conclusion
  of $C_{\mathit{end}}$.  The set $V$\, is empty if $C_{\mathit{end}}$
  does not use the \proofRule{let} rule.  Otherwise, it contains all
  conclusions of the
  \proofRule{assume} steps among $[C_{δ}, \dots,
  C_{\mathit{start}}]$ where $δ$ is either the largest index
  $δ < {\mathit{start}}$ such
  that $s_{δ}$ is an anchor, or $1$ if no such index exist.
  Hence, there is no anchor between $C_{δ}$ and $C_{\mathit{start}}$.
\end{lemma}
\begin{proof}
  The step $C_{\mathit{start}}$ is an anchor due to the definition
  of innermost-first subproof.
  Let $c_1, \dots, c_n$ be the context introduced by the anchor
  $C_{\mathit{start}}$, and let $\Gamma$ be the calculated context
  of $C_{\mathit{start}}$.
  $\Gamma' = \Gamma, c_1, \dots, c_n$.
  is the calculated context of the steps in the subproof after
  $C_{\mathit{start}}$.

  The step $C_{\mathit{end}}$ is a step

\begin{AletheS}
                  & \spctx{}          &         & $\cdots$    &           \\
$\mathit{end}-1$. & \spctx{$\Gamma'$} & \ctxsep & $\,\psi'$ & $(\dots)$ \\
\spsep
$\mathit{end}$.   & $\Gamma$          & \ctxsep & $\psi$    & $(\ruleType{rule}\;i_1, \dots, i_n)$ \\
\end{AletheS}

  Since we assume the step $C_{\mathit{end}}$ is correctly employed,
  $\vDash \Gamma \vartriangleright \psi$ holds, as long as
  $\vDash \Gamma' \vartriangleright \psi'$ holds.

  We perform the same induction as for
  Lemma~\ref{lem:sound_subproof} over the steps in
  $[C_{\mathit{start}}, \dots, C_{\mathit{end}}]$.
  %
  Since $C_{\mathit{end}}$ does not use the \proofRule{subproof} rule,
  the subproof does not contain any assumptions and
  $V_i$ stays empty.
  %
  Again, we are interested in the step $C_{\mathit{end}-1}$.  At this step we
  get $\vDash \Gamma' \vartriangleright \psi'$.

  Only the \proofRule{let} rule uses additional premises
  $C_{i_1}, \dots, C_{i_n}$.  Hence, for all other rules, the conclusion
  cannot depend on any step outside the subproof and $V$\, is empty.
  % TODO: this is ugly, let is ugly
  Due to the definition of first-innermost subproof, all steps $C_{i_1},
  \dots, C_{i_n}$ are in the same subproof that starts at $C_{δ}$.

  The steps $C_{i_1}, \dots, C_{i_n}$ might
  depend on some \proofRule{assume} steps that appear before them
  in their subproof.  This is the case if the steps are outermost steps, or
  if the subproof that starts at $C_{δ}$ concludes with
  a \proofRule{subproof} step.
  %
  In this case we can, as we saw in the proof of
  Lemma~\ref{lem:sound_subproof}, weaken their judgments to include
  all assumptions in $[C_{δ}, \dots, C_{\mathit{start}}]$.

  If the subproof that starts at $C_{δ}$ concludes with
  any other rule, then there cannot be any assumptions and $V$\, is empty.
\end{proof}

By using Lemma~\ref{lem:sound_subproof} and
Lemma~\ref{lem:sound_subproof_context} we can now show that
a valid, concrete Alethe proof is sound. That is, we can show
Theorem~\ref{thm:sound}.

\begin{proof}
  Since $P =[C_1, \dots, C_n]$ is valid, all steps that do not use the
  \proofRule{hole} rule adhere to their rule.  Since we assume that the
  abstract notation and the rules are sound, we only have to
  worry about the steps using the \proofRule{hole} rule.
  Those should be sound, i.e., for a \proofRule{hole} step with the conclusion
  $\psi$, premises $V$, and context $\Gamma$
  the judgment $V \vDash \Gamma \vartriangleright \psi$ must hold.

  Since $P$\, is a valid proof there is a sequence
  $[P_0, \dots, P_{\mathit{last}}]$ as discussed in Section~\ref{sec:alethe:semantics}.
  For $i < \mathit{last}$, $E(P_i) = P_{i+1}$ replaces the
  first-innermost subproof in $P_i$ by a hole with the conclusion
  $\psi$.  Furthermore, the context of the introduced hole
  corresponds to the context $\Gamma$ of the start of the subproof.
  Since $P$ is a valid proof, the first-innermost subproof eliminated by $E$ is
  always valid.
  Therefore,
  we can apply Lemma~\ref{lem:sound_subproof}
  or Lemma~\ref{lem:sound_subproof_context} to conclude that the hole introduced
  by $E$ is sound.

  Since $P_0$ does not contain any holes, the holes in each proof
  $P_i$ are all introduced by innermost-first subproof elimination.
  Therefore, they are sound. In consequence, all holes in $P_{\mathit{last}}$ are
  sound and we can perform the same
  argument as
  in the proof of Lemma~\ref{lem:sound_subproof} to the proof
  $P_{\mathit{last}}$.

  Let $j$ be the index of the step in $P_{\mathit{last}}$ that concludes
  with the empty clause.
  Let $\mathit{start} = 1$
  and $\mathit{end} = j$ in the argument of Lemma~\ref{lem:sound_subproof}.
  This shows that $V \vDash \bot$, where $V$ is the
  conclusion of the \proofRule{assume} steps in the sublist $[C_1, \dots, C_j]$
  of $P_{\mathit{last}}$.  We can weaken
  this by adding the conclusions of the \proofRule{assume} steps in
  $[C_j, \dots, C_n]$ of $P_{\mathit{last}}$
  to get $\varphi_1, \dots, \varphi_m \vDash \bot$.
\end{proof}


\section{Core Concepts of the Alethe Rules}
\label{sec:alethe:rules-generic}

Together with the language, the Alethe format also includes a set
of proof rule.
Section~\ref{apx:rules}
gives a full list of all
proof rules.  Currently, the proof rules correspond to the rules
that the solver {\verit} can emit.  For the rest of this section, we will discuss some
general concepts related to the rules.

\paragraph{Tautologous Rules and Simple Deduction}
Most rules introduce tautologies. One example is
the \proofRule{and_pos} rule: $\neg (\varphi_1 \land \varphi_2 \land
\dots \land \varphi_n), \varphi_i$.
%
Other rules derive their conclusion from a single premise.
%
Those rules are primarily used to simplify Boolean
connectives during preprocessing. For example, the \proofRule{implies}
rule eliminates an implication: From $\varphi_1 → \varphi_2$,
it deduces $\neg \varphi_1, \varphi_2$.


\paragraph{Resolution.}
{\cdclt}-based SMT solvers, and especially their SAT solvers,
are fundamentally based on resolution of clauses.
Hence, Alethe also has dedicated clauses and a resolution proof
rule.  However, since SMT solvers do not enforce a strict
clausal normal form, ordinary disjunction is also used.
Keeping clauses and disjunctions distinct simplifies rule checking.
For example, in the case of
resolution there is a clear distinction between unit clauses where
the sole literal starts with a disjunction and non-unit clauses.  The
syntax for clauses uses the \inlineAlethe{cl} operator, while disjunctions
use the standard {\smtlib} \inlineAlethe{or} operator. The \proofRule{or}
\emph{rule} is responsible for converting disjunctions into clauses.

The Alethe proofs use a generalized propositional resolution
rule with the name \proofRule{resolution} or \proofRule{th_resolution}.
Both names denote the same rule.  The difference only serves to distinguish
if the rule was introduced by
the SAT solver or by a theory solver. The resolution step is purely
propositional; there is no notion of a unifier.  The resolution
rules also implicitly simplifies repeated negations at the head
of literals.

The premises of a resolution step are clauses, and the conclusion
is a clause that has been derived from the premises by some binary
resolution steps.

\paragraph{Quantifier Instantiation}
To express quantifier instantiation, the rule \proofRule{forall_inst}
is used. It produces a formula of the form $(\neg \forall \bar
x_n.\,\varphi) \lor \varphi[\bar t_n]$, where $\varphi$ is
a term containing the free variables $\bar x_n$, and for each $i$ the
ground term $t_i$ is a new term with the same sort as $x_i$.\footnote{
  For historical reasons, \proofRule{forall_inst} has a unit clause with a disjunction
  as its conclusion and not the clause $(\neg \forall \bar x_n.\,\varphi), \varphi[\bar t_n]$.
}

The arguments of a \proofRule{forall_inst} step are the list $(x_1 , t_1),
\dots, (x_n, t_n)$. While this information can be recovered from the term,
providing it explicitly helps reconstruction because the implicit reordering of
equalities obscures which terms have been used as instances.
Existential quantifiers are handled by skolemization.


\paragraph{Linear Arithmetic}
Proofs for linear arithmetic use a number of straightforward rules,
such as \proofRule{la_totality} ($t_1 \leq t_2 \lor t_2 \leq t_1$)\footnote{
  This rule also has a unit clause with a disjunction as its conclusion.}
and the main rule \proofRule{la_generic}.  The conclusion of an
\proofRule{la_generic} step is a tautology $\neg \varphi_1, \neg
\varphi_2, \dots, \neg \varphi_n$ where the $\varphi_i$ are linear
(in)equalities. Checking the validity of this clause amounts to
checking the unsatisfiability of the system of linear equations
$\varphi_1, \varphi_2, \dots, \varphi_n$.  The annotation of an
\proofRule{la_generic} step contains a coefficient for each (in)equality.
The result of forming the linear combination of the literals with
the coefficients is a trivial inequality between constants.

\begin{example}
The following example is the proof for the unsatisfiability
of $(x+y<1) \lor$ $(3<x)$, $x≈ 2$, and $0≈ y$.

\begin{Alethe}
1.& \ctxsep & $(3 < x) \lor (x + y < 1) $ & $ \proofRule{assume}$ \\
2.& \ctxsep & $x≈ 2                     $ & $ \proofRule{assume}$ \\
3.& \ctxsep & $0≈ y                     $ & $ \proofRule{assume}$ \\
4.& \ctxsep & $(3 < x), (x + y < 1)     $ & $ (\proofRule{or}\,1)$ \\
5.& \ctxsep & $\neg (3<x), \neg (x≈ 2)  $ & $ \proofRule{la_generic}\, [1.0, 1.0]$ \\
6.& \ctxsep & $\neg (3<x)               $ & $ (\proofRule{resolution}\, 2, 5)$ \\
7.& \ctxsep & $x + y < 1                $ & $ (\proofRule{resolution}\, 4, 6)$ \\
8.& \ctxsep & $\neg (x + y < 1), \neg (x≈ 2) \lor \neg (0 ≈ y)$ & $\proofRule{la_generic}\,[1.0, -1.0, 1.0]$ \\
9.& \ctxsep & $\bot $ & $ (\proofRule{resolution}\, 8, 7, 2, 3)$ \\
\end{Alethe}
\end{example}

\paragraph{Skolemization and Other Preprocessing Steps}
One typical example for a rule with context is the \proofRule{sko_ex}
rule that is used to express skolemization of an existentially
quantified variable.
%
The conclusion of a step that uses this rules is an equality.
The left-hand side is a formula starting with an existential quantifier
over some variable $x$.  In the formula on the right-hand side, the
variable is replaced by the appropriate Skolem term.
To provide a proof for the replacement, the \proofRule{sko_ex} step
uses one premise.
The premise
has a context that maps the existentially quantified variable
to the appropriate Skolem term.

\begin{AletheS}
i. & \spctx{$\Gamma, x\mapsto (\varepsilon x.\,\varphi)$} & \ctxsep
& $\varphi ≈ \psi$ & $(\dots)$ \\
\spsep
j. & $\Gamma$ & \ctxsep & $(\exists x.\,\varphi) ≈ \psi$ & $(\proofRule{sko_ex})$ \\
\end{AletheS}

\begin{example}
To illustrate how such a rule is applied, consider the following example
taken from~\cite{barbosa-2019}. Here the term $\neg p(\varepsilon x.\neg
p(x))$ is skolemized. The \proofRule{refl} rule expresses
a simple tautology on the equality (reflexivity in this case), \proofRule{cong}
is functional congruence, and \proofRule{sko_forall} works like
\proofRule{sko_ex}, except that the choice term is $\varepsilon x.\neg\varphi$.

\begin{AletheS}
1. & \spctx{$x\mapsto (\varepsilon x.\,\neg (p\,x))$}
     & \ctxsep
     &  $x  ≈   \varepsilon x.\,\neg (p\,x) $ &
     $\proofRule{refl}$ \\
2. & \spctx{$x\mapsto (\varepsilon x.\,\neg (p\,x))$} & \ctxsep &  $ (p\,x) ≈ p(\varepsilon x.\,\neg (p\,x))$ &  $(\proofRule{cong}\, 1)$ \\
 \spsep
3. &  & \ctxsep &  $(    \forall x.\,(p\,x))≈      (p\,(\varepsilon x.\,\neg (p\,x)))$ &  $(\proofRule{sko_forall}\, 2)$ \\
4. &  & \ctxsep &  $(\neg\forall x.\,(p\,x))≈ \neg (p\,(\varepsilon x.\,\neg (p\,x)))$ &  $(\proofRule{cong}\, 3)$ \\
\end{AletheS}
\end{example}

\subsection{Bitvector Reasoning with Bitblasting}

A standard approach to handle bitvector reasoning in SMT solvers
is bitblasting.  Bitblasting\index{bitblasting} works by translating
bitvector\index{bitvector} functions to propositional formulas that
model the logical circuit of the bitvector function.

To express bitblasting in Alethe proof rules, the the Alethe calculus uses
multiple families of helper functions: $\lsymb{bbT}$, $\lsymb{bitOf}_m$,
$\lsymb{bvsize}$, and $\lsymb{bv}_n^i$.  Functions in the families are
distinguished either by overloading ($\lsymb{bbT}$ and $\lsymb{bvsize}$)
or by explicit indexing ($\lsymb{bitOf}_m$ and $\lsymb{bv}_n^i$).
To avoid name clashes with user defined
functions, $\lsymb{bbT}$ is written as \inlineAlethe{@bbT}, $\lsymb{bitOf}$
as \inlineAlethe{@bitOf}, $\lsymb{bvsize}$ as \inlineAlethe{@bvsize}, and
$\lsymb{bv}$ as \inlineAlethe{@bv}.  The SMT-LIB standard specifies
that simple symbols starting with ``\inlineAlethe{@}'' are reserved for
solver generated functions.

The family $\lsymb{bbT}$ consists of one function for each bitvector sort
$(\lsymb{BitVec}\;n)$.
\[
\lsymb{bbT}\,:\,\underbrace{\lsymb{Bool}\,\dots\,\lsymb{Bool}}_n\;(\lsymb{BitVec}\;n).
\]

\noindent
Intuitively, the function $\lsymb{bbT}$ takes a list of boolean arguments and
packs them into a bitvector.
Let $\langle u_1, \dots, u_n\rangle$ denote a bitvector of sort $(\lsymb{BitVec}\;n)$
where $u_i = \top$ if the bit at position $i$ is $1$, and $u_i = \bot$ otherwise.
The bit $u_n$ is the least significant bit.  Then

\[
\lsymb{bbT}\; v_1 \dots v_n = \langle v_1, \dots, v_n\rangle .
\]

\noindent
The $\lsymb{bbT}$ functions could be defined in terms of standard SMT-LIB functions.
\begin{align*}
\lsymb{bbT}\;v_1 \dots v_n :=\;
 &\lsymb{concat}\,(\lsymb{concat}\,(\dots \\
 &(\lsymb{concat}\,(\lsymb{ite}\,v_1\,\langle\top\rangle\,\langle\bot\rangle)\,(\lsymb{ite}\,v_2\,\langle\top\rangle\,\langle\bot\rangle))\\
 & \dots \\
 & (\lsymb{ite}\,v_{n-1}\,\langle\top\rangle\,\langle\bot\rangle)) \\
 & (\lsymb{ite}\,v_n\,\langle\top\rangle\,\langle\bot\rangle)) \\
\end{align*}

\noindent
The functions $\lsymb{bitOf}_m$ are the inverse of $\lsymb{bbT}$.  They extract
a bit of a bitvector as a boolean.  Just as the built in $\lsymb{extract}$
symbol, $\lsymb{bitOf}_m$ is used as an indexed symbol.  Hence, for $m \leq n$, we
write \inlineAlethe{(_ @bitOf} $m$ \inlineAlethe{)}, to denote functions
\[
\lsymb{bitOf}_m : (\lsymb{BitVec}\;n) \to \lsymb{Bool}.
\]
These functions are defined as 
\[
\lsymb{bitOf}_m \langle u_1, \dots, u_n \rangle := u_m.
\]


\noindent
The functions $\lsymb{bvsize}$ return the size of a bitvector.  Formally, there
is one $\lsymb{bvsize}$ for for each bitvector sort $(\lsymb{BitVec}\;n)$.  Each
$\lsymb{bvsize}$ is a constant function that returns $n$.  Using notation:

\begin{align*}
\lsymb{bvsize}& : (\lsymb{BitVec}\;n) \to \mathbb{N}\\
\lsymb{bvsize}&\;b := n
\end{align*}

\noindent
Finally, $\lsymb{bv}_n^i$ is a family of constants indexed by two parameters:
a bitvector length $n$, and a natural number $i$.  We
write \inlineAlethe{(_ @bv}$n$ $i$ \inlineAlethe{)} for $\lsymb{bv}_n^i$.
The space before $n$ is omitted for historical reasons.
Each $\lsymb{bv}_n^i$ is the bitvector constant that represents the bitvector
of length $n$ that encodes the integer $i$.  Formally, it corresponds
to \inlineAlethe{nat2bv[n](i)}, where \inlineAlethe{nat2bv} is defined
as in the SMT-LIB
standard.\footnote{See \url{https://smt-lib.github.io/theories-FixedSizeBitVectors.shtml}.}

\section{The Alethe Rules}
\label{apx:rules}
This section provides a list of all proof rules supported by
Alethe.  To make this long list more accessible, the section
first lists multiple classes of proof rules.  The classes are not
mutually exclusive: for example, the \proofRule{la_generic} rule is
both a linear arithmetic rule and introduces a tautology.
The number in brackets is the position of the rule in the overall list
of proof rules.
Table~\ref{rule-tab:special} lists rules that serve a special purpose.
Table~\ref{rule-tab:tautologies} lists all rules that introduce
tautologies.  That is, regular rules that do not use premises.

The subsequent section, starting at \ref{sec:alethe:rules-list}, defines
all rules and provides example proofs for complicated rules.
The index of proof rules on page \pageref{sec:alethe:rules-index} can be used
to quickly find the definition of rules.

\subsection{Classifications of the Rules}
\label{sec:alethe:rules-overview}

\begin{xltabular}{\linewidth}{l X}
\caption{Special rules.}
\label{rule-tab:special}\\
  Rule & Description \\
  \hline
  \ruleref{assume}   & Introduction of an assumption. \\
  \ruleref{hole}     & Placeholder for rules not defined here. \\
  \ruleref{subproof} & Concludes a subproof and discharges local assumptions. \\
\end{xltabular}

\begin{xltabular}{\linewidth}{l X}
\caption{Resolution and related rules.}
\label{rule-tab:resolution}\\
  Rule & Description \\
  \hline
  \ruleref{resolution} & Chain resolution of two or more clauses. \\
  \ruleref{th_resolution} & As \proofRule{resolution}, but used by theory solvers. \\
  \ruleref{tautology} & Simplification of tautological clauses to $\top$. \\
  \ruleref{contraction} & Removal of duplicated literals. \\
\end{xltabular}

\begin{xltabular}{\linewidth}{l X}
\caption{Rules introducing tautologies.}
\label{rule-tab:tautologies}\\
  Rule & Description \\
  \hline
\ruleref{true} & $\top$ \\
\ruleref{false} & $\neg\bot$ \\
\ruleref{not_not} & $\neg(\neg\neg\varphi), \varphi$ \\
\ruleref{la_generic} & Tautologous disjunction of linear inequalities. \\
\ruleref{lia_generic} & Tautologous disjunction of linear integer inequalities. \\
\ruleref{la_disequality} & $t_1 ≈ t_2 \lor \neg (t_1 \leq t_2) \lor \neg (t_2 \leq t_1)$ \\
\ruleref{la_totality} & $t_1 \leq t_2 \lor t_2 \leq t_1$ \\
\ruleref{la_tautology} & A trivial linear tautology. \\
\ruleref{la_mult_pos} & Multiplication with a positive factor. \\
\ruleref{la_mult_neg} & Multiplication with a negative factor.\\
\ruleref{forall_inst} & Quantifier instantiation. \\
\ruleref{refl} & Reflexivity after applying the context. \\
\ruleref{eq_reflexive} & $t ≈ t$ without context. \\
\ruleref{eq_transitive} & $\neg (t_1 ≈ t_2) , \dots , \neg (t_{n-1} ≈ t_n) , t_1 ≈ t_n$ \\
\ruleref{eq_congruent} & $\neg (t_1 ≈ u_1) , \dots , \neg (t_n ≈ u_n) , f(t_1, \dots, t_n) ≈ f(u_1, \dots, u_n)$ \\
\ruleref{eq_congruent_pred} & $\neg (t_1 ≈ u_1) , \dots , \neg (t_n ≈ u_n) , P(t_1, \dots, t_n) ≈ P(u_1, \dots, u_n)$ \\
\ruleref{qnt_cnf} & Clausification of a quantified formula. \\
\ruleref{and_pos} & $\neg (\varphi_1 \land \dots \land \varphi_n) , \varphi_k $ \\
\ruleref{and_neg} & $ (\varphi_1 \land \dots \land \varphi_n), \neg\varphi_1  , \dots , \neg\varphi_n $ \\
\ruleref{or_pos} & $ \neg (\varphi_1 \lor \dots \lor \varphi_n) , \varphi_1 , \dots , \varphi_n $ \\
\ruleref{or_neg} & $ (\varphi_1 \lor \dots \lor \varphi_n) , \neg \varphi_k $; with $1\leq k \leq n$ \\
\ruleref{xor_pos1} & $ \neg (\varphi_1 \,\lsymb{xor}\, \varphi_2) , \varphi_1 , \varphi_2 $ \\
\ruleref{xor_pos2} & $ \neg (\varphi_1 \,\lsymb{xor}\, \varphi_2), \neg \varphi_1, \neg \varphi_2 $ \\
\ruleref{xor_neg1} & $ \varphi_1 \,\lsymb{xor}\, \varphi_2, \varphi_1 , \neg \varphi_2 $ \\
\ruleref{xor_neg2} & $ \varphi_1 \,\lsymb{xor}\, \varphi_2, \neg \varphi_1 , \varphi_2 $ \\
\ruleref{implies_pos} & $ \neg (\varphi_1 \rightarrow \varphi_2), \neg \varphi_1 , \varphi_2 $ \\
\ruleref{implies_neg1} & $ \varphi_1 \rightarrow \varphi_2, \varphi_1 $ \\
\ruleref{implies_neg2} & $ \varphi_1 \rightarrow \varphi_2, \neg \varphi_2 $ \\
\ruleref{equiv_pos1} & $\neg (\varphi_1 ≈ \varphi_2) , \varphi_1 , \neg \varphi_2$ \\
\ruleref{equiv_pos2} & $\neg (\varphi_1 ≈ \varphi_2) , \neg \varphi_1 , \varphi_2$ \\
\ruleref{equiv_neg1} & $\varphi_1 ≈ \varphi_2 , \neg \varphi_1 , \neg \varphi_2$ \\
\ruleref{equiv_neg2} & $\varphi_1 ≈ \varphi_2 , \varphi_1 , \varphi_2$ \\
\ruleref{ite_pos1} & $\neg (\lsymb{ite}\ \varphi_1\;\varphi_2\;\varphi_3) , \varphi_1 , \varphi_3$ \\
\ruleref{ite_pos2} & $\neg (\lsymb{ite}\ \varphi_1\;\varphi_2\;\varphi_3) , \neg \varphi_1 , \varphi_2$ \\
\ruleref{ite_neg1} & $(\lsymb{ite}\ \varphi_1\;\varphi_2\;\varphi_3) , \varphi_1 , \neg \varphi_3$ \\
\ruleref{ite_neg2} & $(\lsymb{ite}\ \varphi_1\;\varphi_2\;\varphi_3) , \neg \varphi_1 , \neg \varphi_2$ \\
\ruleref{connective_def} & Definition of some connectives. \\
\ruleref{and_simplify} & Simplification of a conjunction. \\
\ruleref{or_simplify} & Simplification of a disjunction. \\
\ruleref{not_simplify} & Simplification of a Boolean negation. \\
\ruleref{implies_simplify} & Simplification of an implication. \\
\ruleref{equiv_simplify} & Simplification of an equivalence. \\
\ruleref{bool_simplify} & Simplification of combined Boolean connectives. \\
\ruleref{ac_simp} & Flattening and removal of duplicates for $\lor$ or $\land$. \\
\ruleref{ite_simplify} & Simplification of if-then-else. \\
\ruleref{qnt_simplify} & Simplification of constant quantified formulas. \\
\ruleref{qnt_join} & Joining of consecutive quantifiers. \\
\ruleref{qnt_rm_unused} & Removal of unused quantified variables. \\
\ruleref{eq_simplify} & Simplification of equality. \\
\ruleref{div_simplify} & Simplification of division. \\
\ruleref{prod_simplify} & Simplification of products. \\
\ruleref{unary_minus_simplify} & Simplification of the unary minus. \\
\ruleref{minus_simplify} & Simplification of subtractions. \\
\ruleref{sum_simplify} & Simplification of sums. \\
\ruleref{comp_simplify} & Simplification of arithmetic comparisons. \\
\ruleref{distinct_elim} & Elimination of the $\lsymb{distinct}$ operator. \\
\ruleref{la_rw_eq} & $(t ≈ u) ≈ (t \leq u \land u \leq t)$ \\
\ruleref{nary_elim} & Eliminate $n$-ary application of operators via binary applications. \\
\ruleref{eq_symmetric} & Symmetry of equality as equivalence. \\
% TODO
\ruleref{string_length_pos} & Trivial tautology about the length of a String term. \\
\end{xltabular}

\begin{xltabular}{\linewidth}{l X}
\caption{Linear arithmetic rules.}
\label{rule-tab:la-tauts}\\
  Rule & Description \\
  \hline
\ruleref{la_generic} & Tautologous disjunction of linear inequalities. \\
\ruleref{lia_generic} & Tautologous disjunction of linear integer inequalities. \\
\ruleref{la_disequality} & $t_1 ≈ t_2 \lor \neg (t_1 \leq t_2) \lor \neg (t_2 \leq t_1)$ \\
\ruleref{la_totality} & $t_1 \leq t_2 \lor t_2 \leq t_1$ \\
\ruleref{la_tautology} & A trivial linear tautology. \\
\ruleref{la_mult_pos} & Multiplication with a positive factor. \\
\ruleref{la_mult_neg} & Multiplication with a negative factor.\\
\ruleref{la_rw_eq} & $(t ≈ u) ≈ (t \leq u \land u \leq t)$ \\
\ruleref{div_simplify} & Simplification of division. \\
\ruleref{prod_simplify} & Simplification of products. \\
\ruleref{unary_minus_simplify} & Simplification of the unary minus. \\
\ruleref{minus_simplify} & Simplification of subtractions. \\
\ruleref{sum_simplify} & Simplification of sums. \\
\ruleref{comp_simplify} & Simplification of arithmetic comparisons. \\
\end{xltabular}

\begin{xltabular}{\linewidth}{l X}
\caption{Quantifier handling.}
\label{rule-tab:quants}\\
  Rule & Description \\
  \hline
\ruleref{forall_inst} & Instantiation of a universal quantifier. \\
\ruleref{bind} & Renaming of bound variables. \\
\ruleref{sko_ex} & Skolemization of an existential quantifier. \\
\ruleref{sko_forall} & Skolemization of an universal quantifier. \\
\ruleref{qnt_cnf} & Clausification of quantified formulas. \\
\ruleref{qnt_simplify} & Simplification of constant quantified formulas. \\
\ruleref{onepoint} & The one-point rule. \\
\ruleref{qnt_join} & Joining of consecutive quantifiers. \\
\ruleref{qnt_rm_unused} & Removal of unused quantified variables. \\
\end{xltabular}

\begin{xltabular}{\linewidth}{l X}
\caption{Skolemization rules.}
\label{rule-tab:skos}\\
  Rule & Description \\
  \hline
\ruleref{sko_ex} & Skolemization of existential variables. \\
\ruleref{sko_forall} & Skolemization of universal variables. \\
\end{xltabular}

\begin{xltabular}{\linewidth}{l X}
\caption{Clausification rules.  These rules can be used to perform propositional
clausification.}
\label{rule-tab:clausification}\\
  Rule & Description \\
  \hline
\ruleref{and} & And elimination. \\
\ruleref{not_or} & Elimination of a negated disjunction. \\
\ruleref{or} & Disjunction to clause. \\
\ruleref{not_and} & Distribution of negation over a conjunction. \\
\ruleref{xor1} & From $\,(\lsymb{xor}\, \varphi_1\, \varphi_2)$ to $\varphi_1, \varphi_2$. \\
\ruleref{xor2} & From $\,(\lsymb{xor}\, \varphi_1\, \varphi_2)$ to $\neg\varphi_1, \neg\varphi_2$. \\
\ruleref{not_xor1} & From $\neg(\,\lsymb{xor}\, \varphi_1\, \varphi_2)$ to $\varphi_1, \neg\varphi_2$. \\
\ruleref{not_xor2} & From $\neg(\,\lsymb{xor}\, \varphi_1\, \varphi_2)$ to $\neg\varphi_1, \varphi_2$. \\
\ruleref{implies} & From $ \varphi_1\rightarrow\varphi_2 $ to $\neg\varphi_1 , \varphi_2 $. \\
\ruleref{not_implies1} & From $\neg (\varphi_1\rightarrow\varphi_2)$ to $\varphi_1$. \\
\ruleref{not_implies2} & From $\neg (\varphi_1\rightarrow\varphi_2)$ to $\neg\varphi_2$. \\
\ruleref{equiv1} & From $ \varphi_1≈\varphi_2$ to $\neg\varphi_1 , \varphi_2$. \\
\ruleref{equiv2} & From $ \varphi_1≈\varphi_2$ to $\varphi_1 , \neg\varphi_2$. \\
\ruleref{not_equiv1} & From $\neg(\varphi_1≈\varphi_2)$ to $\varphi_1 , \varphi_2$. \\
\ruleref{not_equiv2} & From $\neg(\varphi_1≈\varphi_2)$ to $\neg\varphi_1 , \neg\varphi_2$. \\
\ruleref{and_pos} & $\neg (\varphi_1 \land \dots \land \varphi_n), \varphi_k$\\
\ruleref{and_neg} & $(\varphi_1 \land \dots \land \varphi_n), \neg\varphi_1, \dots , \neg\varphi_n $ \\
\ruleref{or_pos} & $\neg (\varphi_1 \lor \dots \lor \varphi_n) , \varphi_1 , \dots
  , \varphi_n $ \\
\ruleref{or_neg} &
$ (\varphi_1 \lor \dots \lor \varphi_n) , \neg \varphi_k $\\
\ruleref{xor_pos1} &
$ \neg (\varphi_1 \,\lsymb{xor}\, \varphi_2) , \varphi_1 , \varphi_2 $ \\
\ruleref{xor_pos2} &
$ \neg (\varphi_1 \,\lsymb{xor}\, \varphi_2)
, \neg \varphi_1, \neg \varphi_2 $ \\
\ruleref{xor_neg1} &
$ \varphi_1 \,\lsymb{xor}\, \varphi_2, \varphi_1 , \neg \varphi_2 $ \\
\ruleref{xor_neg2} &
$ \varphi_1 \,\lsymb{xor}\, \varphi_2, \neg \varphi_1 , \varphi_2 $ \\
\ruleref{implies_pos} &
$ \neg (\varphi_1 \rightarrow \varphi_2), \neg \varphi_1 , \varphi_2 $ \\
\ruleref{implies_neg1} &
$ \varphi_1 \rightarrow \varphi_2, \varphi_1 $ \\
\ruleref{implies_neg2} &
$ \varphi_1 \rightarrow \varphi_2, \neg \varphi_2 $ \\
\ruleref{equiv_pos1} &
$\neg (\varphi_1 ≈ \varphi_2) , \varphi_1 , \neg \varphi_2$ \\
\ruleref{equiv_pos2} &
$\neg (\varphi_1 ≈ \varphi_2) , \neg \varphi_1 , \varphi_2$ \\
\ruleref{equiv_neg1} &
$\varphi_1 ≈ \varphi_2 , \neg \varphi_1 , \neg \varphi_2$ \\
\ruleref{equiv_neg2} &
$\varphi_1 ≈ \varphi_2 , \varphi_1 , \varphi_2$ \\
\ruleref{let} & Elimination of the $\lsymb{let}$ operator.  \\
\ruleref{distinct_elim} & Elimination of the $\lsymb{distinct}$ operator.  \\
\ruleref{nary_elim} & Elimination of n-ary application of operators. \\
\end{xltabular}

\begin{xltabular}{\linewidth}{l X}
\caption{Simplification rules. These rules represent typical operator-level
simplifications.}
\label{rule-tab:simplification}\\
  Rule & Description \\
  \hline
\ruleref{connective_def} & Definition of the Boolean connectives. \\
\ruleref{and_simplify} & Simplification of a conjunction. \\
\ruleref{or_simplify} & Simplification of a disjunction. \\
\ruleref{not_simplify} & Simplification of a Boolean negation. \\
\ruleref{implies_simplify} & Simplification of an implication. \\
\ruleref{equiv_simplify} & Simplification of an equivalence. \\
\ruleref{bool_simplify} & Simplification of combined Boolean connectives. \\
\ruleref{ac_simp} & Flattening and removal of duplicates for $\lor$ or $\land$. \\
\ruleref{ite_simplify} & Simplification of if-then-else. \\
\ruleref{qnt_simplify} & Simplification of constant quantified formulas. \\
\ruleref{onepoint} & The one-point rule. \\
\ruleref{qnt_join} & Joining of consecutive quantifiers. \\
\ruleref{qnt_rm_unused} & Removal of unused quantified variables. \\
\ruleref{eq_simplify} & Simplification of equalities. \\
\ruleref{div_simplify} & Simplification of division. \\
\ruleref{prod_simplify} & Simplification of products. \\
\ruleref{unary_minus_simplify} & Simplification of the unary minus. \\
\ruleref{minus_simplify} & Simplification of subtractions. \\
\ruleref{sum_simplify} & Simplification of sums. \\
\ruleref{comp_simplify} & Simplification of arithmetic comparisons. \\
\ruleref{qnt_simplify} & Simplification of constant quantified formulas. \\
\end{xltabular}

\begin{xltabular}{\linewidth}{l X}
\caption{Bitvector rules.}
\label{rule-tab:bitblasting}\\
  Rule & Description \\
  \hline
\ruleref{bitblast_extract} & Bitblasting of $\lsymb{extract}$. \\
\ruleref{bitblast_ult} & Bitblasting of $\lsymb{ult}$. \\
\ruleref{bitblast_add} & Bitblasting of $\lsymb{add}$. \\
\end{xltabular}

% TODO
\begin{xltabular}{\linewidth}{l X}
\caption{Strings rules.}
\label{rule-tab:strings}\\
  Rule & Description \\
  \hline
\ruleref{concat_eq} & Rule description. \\
\ruleref{concat_unify} & Rule description. \\
\ruleref{concat_conflict} & Rule description. \\
\ruleref{concat_split_prefix} & Rule description. \\
\ruleref{concat_split_suffix} & Rule description. \\
\ruleref{concat_csplit_prefix} & Rule description. \\
\ruleref{concat_csplit_suffix} & Rule description. \\
\ruleref{concat_lprop_prefix} & Rule description. \\
\ruleref{concat_lprop_suffix} & Rule description. \\
\ruleref{concat_cprop_prefix} & Rule description. \\
\ruleref{concat_cprop_suffix} & Rule description. \\
\ruleref{string_length_pos} & Trivial tautology about the length of a String term. \\
\ruleref{string_length_non_empty} & Rule description. \\
\ruleref{re_inter} & Rule description. \\
\ruleref{re_unfold_neg} & Rule description. \\
\ruleref{re_unfold_neg_concat_fixed_prefix} & Rule description. \\
\ruleref{re_unfold_neg_concat_fixed_suffix} & Rule description. \\
\end{xltabular}

% TODO
\begin{xltabular}{\linewidth}{l X}
\caption{Rules used by cvc5, but not by veriT.}
\label{rule-tab:cvc5}\\
  Rule & Description \\
  \hline
\ruleref{concat_eq} & Rule description. \\
\ruleref{concat_unify} & Rule description. \\
\ruleref{concat_conflict} & Rule description. \\
\ruleref{concat_split_prefix} & Rule description. \\
\ruleref{concat_split_suffix} & Rule description. \\
\ruleref{concat_csplit_prefix} & Rule description. \\
\ruleref{concat_csplit_suffix} & Rule description. \\
\ruleref{concat_lprop_prefix} & Rule description. \\
\ruleref{concat_lprop_suffix} & Rule description. \\
\ruleref{concat_cprop_prefix} & Rule description. \\
\ruleref{concat_cprop_suffix} & Rule description. \\
\ruleref{string_length_pos} & Trivial tautology about the length of a String term. \\
\ruleref{string_length_non_empty} & Rule description. \\
\ruleref{re_inter} & Rule description. \\
\ruleref{re_unfold_neg} & Rule description. \\
\ruleref{re_unfold_neg_concat_fixed_prefix} & Rule description. \\
\ruleref{re_unfold_neg_concat_fixed_suffix} & Rule description. \\
\ruleref{bitblast_extract} & Bitblasting of $\lsymb{extract}$. \\
\ruleref{bitblast_ult} & Bitblasting of $\lsymb{ult}$. \\
\ruleref{bitblast_add} & Bitblasting of $\lsymb{add}$. \\
\ruleref{la_mult_pos} & Multiplication with a positive factor. \\
\ruleref{la_mult_neg} & Multiplication with a negative factor.\\
\ruleref{symm}     & Symmetry of equality. \\
\ruleref{not_symm} & Symmetry of not-equal.\\
\ruleref{reordering} & Reording of the literals in a clause.\\
\end{xltabular}

\begin{xltabular}{\linewidth}{l X}
\caption{Rules used by the Carcara elaborator.}
\label{rule-tab:cvc5}\\
  Rule & Description \\
  \hline
\ruleref{weakening} & Weakening of a clause. \\
\ruleref{eq_symmetric} & Symmetry of equality as equivalence. \\
\end{xltabular}
\subsection{Rule List}
\label{sec:alethe:rules-list}

\begin{RuleDescription}{assume}
\begin{AletheX}
$i$. & \ctxsep & $\varphi$ & \currule \\
\end{AletheX}
  where $\varphi$ corresponds up to the orientation of equalities
  to a formula asserted in the input problem, or $\varphi$ is a local assumption
  in a subproof.
  \ruleparagraph{Remark.}
  This rule can not be used by the
  \inlineAlethe{(step }\dots\inlineAlethe{)} command. Instead it corresponds to the dedicated
  \inlineAlethe{(assume }\dots\inlineAlethe{)} command.
\end{RuleDescription}

\begin{RuleDescription}{hole}

\begin{AletheX}
$i$. & \ctxsep & $\varphi$  & $(\currule\; p_2, \dots, p_n)\, [a_1, \dots, a_m]$ \\
\end{AletheX}
where $\varphi$ is any well-formed formula.

This rule can be used to express holes in the proof.  It can be used by
solvers as a placeholder for proof steps that are not yet expressed
by the proof rules in this document.  A proof checker {\em must not}
accept a proof as valid that contains this rule even if the checker can
somehow check this rule.  However, it is possible for checkers to have
a dedicated status for proofs that contain this rule and are otherwise
valid.  Any other tool can accept or reject proofs that contain this rule.

The premises and arguments are arbitrary, but must follow the syntax
for premises and arguments.
\end{RuleDescription}

\begin{RuleDescription}{true}
\begin{AletheX}
$i$. & \ctxsep & $\top$     & \currule \\
\end{AletheX}
\end{RuleDescription}

\begin{RuleDescription}{false}
\begin{AletheX}
$i$. & \ctxsep & $\neg\bot$     & \currule \\
\end{AletheX}
\end{RuleDescription}

\begin{RuleDescription}{not_not}

\begin{AletheX}
$i$. & \ctxsep & $\neg (\neg \neg \varphi)$, $\varphi$  & \currule \\
\end{AletheX}
\ruleparagraph{Remark.} This rule is useful to remove double negations from a
clause by resolving a clause with the double negation on $\varphi$.
\end{RuleDescription}

\begin{RuleDescription}{th_resolution}
This rule is the resolution of two or more clauses.

\begin{AletheX}
$i_1$. & \ctxsep & $l^1_1$, $\dots$, $l^1_{k^1}$  & ($\ruleType{...}$) \\
\aletheLine
$i_n$. & \ctxsep & $l^n_1$, $\dots$, $l^n_{k^n}$  & ($\ruleType{...}$) \\
  $j$. & \ctxsep & $l^{r_1}_{s_1}$, $\dots$, $l^{r_m}_{s_m}$  & ($\currule\;i_1,\dots, i_n$) \\
\end{AletheX}
where $l^{r_1}_{s_1} , \dots , l^{r_m}_{s_m}$ are from $l^{i}_{j}$ and
are the result of a chain of predicate resolution steps on the clauses of
$i_1$ to $i_n$. It is possible that $m = 0$, i.e. that
the result is the empty clause.  When performing resolution steps, the
rule implicitly merges repeated negations at the start of the formulas
$l^{i}_{j}$.  For example, the formulas $\neg\neg\neg P$ and $\neg\neg P$
can serve as pivots during resolution.  The first formula is interpreted as
$\neg P$ and the second as just $P$ for the purpose of performing resolution
steps.

This rule is only used when the resolution step is not emitted by the SAT solver.
See the equivalent \proofRule{resolution} rule for the rule emitted by the
SAT solver.

\ruleparagraph{Remark.} The definition given here is very general.  The motivation
for this is to ensure the definition covers all possible resolution steps generated
by the existing proof generation code in veriT.  It will be restricted after
a full review of the code.  As a consequence of this checking this rule is
theoretically NP-complete.  In practice, however, the \currule-steps
produced by {\verit} are simple. Experience with reconstructing the step in
{\isabelle} shows that checking can done by naive decision procedures. The
vast majority of \currule-steps are binary resolution steps.
\end{RuleDescription}

\begin{RuleDescription}{resolution}
This rule is equivalent to the \proofRule{th_resolution} rule, but it is
emitted by the SAT solver instead of theory reasoners. The differentiation
serves only informational purpose.

\end{RuleDescription}

\begin{RuleDescription}{tautology}
\begin{AletheX}
$i$. & \ctxsep & $l_1$, \dots, $l_k$, \dots, $l_m$, \dots, $l_n$  & ($\dots$) \\
$j$. & \ctxsep & $\top$  & (\currule\; $i$) \\
\end{AletheX}
  and $l_k$, $l_m$ are such that
  \begin{align*}
  l_k &= \underbrace{\neg \dots \neg}_o \varphi \\
  l_m &= \underbrace{\neg \dots \neg}_p \varphi
  \end{align*}
  and one of $o, p$\/ is odd and the other even.  Either can be $0$.
\end{RuleDescription}

\begin{RuleDescription}{contraction}
\begin{AletheX}
$i$. & \ctxsep  & $l_1$, \dots, $l_n$  & ($\dots$) \\
$j$. & \ctxsep  & $l_{k_1}$, \dots, $l_{k_m}$  & (\currule\; $i$) \\
\end{AletheX}
  where $m \leq n$ and  $k_1 \dots k_m$ is a monotonic map to $1 \dots n$
  such that $l_{k_1} \dots l_{k_m}$ are pairwise distinct and
  $\{l_1, \dots, l_n\} = \{l_{k_1}, \dots, l_{k_m}\}$.
  Hence, this rule removes duplicated literals.
\end{RuleDescription}

\begin{RuleDescription}{subproof}
The \currule{} rule completes a subproof and discharges local
assumptions. Every subproof starts with some \proofRule{assume} steps. The
last step of the subproof is the conclusion.

\begin{AletheX}
$i_1$. & \spctxsep  & $\varphi_1$  & $\proofRule{assume}$ \\
\aletheLine
$i_n$. & \spctxsep  & $\varphi_n$  & $\proofRule{assume}$ \\
\aletheLineB
$j$. & \spctxsep  & $\psi$  & ($\dots$) \\
\spsep
k. & \ctxsep  & $\neg\varphi_1$,\dots, $\neg\varphi_n$, $\psi$   & $\proofRule{subproof}$ \\
\end{AletheX}

\end{RuleDescription}

\begin{RuleDescription}{la_generic}
A step of the \currule{} rule represents a tautological clause of linear
disequalities.  It can be checked by showing that the conjunction of
the negated disequalities is unsatisfiable. After the application of
some strengthening rules, the resulting conjunction is unsatisfiable,
even if integer variables are assumed to be real variables.

A linear inequality is of term of the form
\[
\sum_{i=0}^{n}c_i\times{}t_i +
d_1\bowtie \sum_{i=n+1}^{m} c_i\times{}t_i + d_2
\]
where $\bowtie\;\in \{≈, <,
>, \leq, \geq\}$, where $m\geq n$, $c_i, d_1, d_2$ are either integer or real
constants, and for each $i$ $c_i$ and $t_i$ have the same sort. We will write
$s_1 \bowtie s_2$.

Let $l_1,\dots, l_n$ be linear inequalities and $a_1, \dots, a_n$
rational numbers, then a {\currule} step has the form

\begin{AletheX}
$i$. & \ctxsep & $\varphi_1$, \dots , $\varphi_o$  & \currule\, [$a_1$, \dots, $a_o$] \\
\end{AletheX}
where $\varphi_i$ is either $\neg l_i$ or $l_i$, but never $s_1
≈ s_2$.

The constants $a_i$ are of sort \inlineAlethe{Real} and must be printed
using one of the productions $\grNT{rational}$
$\grNT{decimal}$, $\grNT{nonpositive\_decimal}$.

To check the unsatisfiability of the negation of $\varphi_1\lor \dots
\lor \varphi_o$ one performs the following steps for each literal. For
each $i$, let $\varphi := \varphi_i$ and $a := a_i$.

\begin{enumerate}
    \item If $\varphi = s_1 > s_2$, then let $\varphi := s_1 \leq s_2$.
      If $\varphi = s_1 \geq s_2$, then let $\varphi := s_1 < s_2$.
      If $\varphi = s_1 < s_2$, then let $\varphi := s_1 \geq s_2$.
      If $\varphi = s_1 \leq s_2$, then let $\varphi := s_1 > s_2$. This negates
      the literal.
     \item If $\varphi = \neg (s_1 \bowtie s_2)$, then let $\varphi := s_1 \bowtie s_2$.
    \item Replace $\varphi$ by $\sum_{i=0}^{n}c_i\times{}t_i - \sum_{i=n+1}^{m} c_i\times{}t_i
    \bowtie d$ where $d := d_2 - d_1$.
    \item \label{la_generic:str}Now $\varphi$ has the form $s_1 \bowtie d$. If all
    variables in $s_1$ are integer sorted: replace $\bowtie d$ according to
    the table below.
    \item If $\bowtie$ is $≈$ replace $l$ by
    $\sum_{i=0}^{m}a\times{}c_i\times{}t_i ≈ a\times{}d$, otherwise replace it by
    $\sum_{i=0}^{m}|a|\times{}c_i\times{}t_i ≈ |a|\times{}d$.
\end{enumerate}

The replacements that can be performed by step~\ref{la_generic:str} above
are

\begin{tabular}{r l l}
$\bowtie$  & If $d$ is an integer  & Otherwise \\
$>$        & $\geq d + 1$  & $\geq \lfloor d\rfloor + 1$  \\
$\geq$     & $\geq d$      & $\geq \lfloor d\rfloor + 1$  \\
\end{tabular}

Finally, the sum of the resulting literals is trivially contradictory.
The sum
\[
    \sum_{k=1}^{o}\sum_{i=1}^{m^o}c_i^k*t_i^k \bowtie \sum_{k=1}^{o}d^k
\]
where $c_i^k$ is the constant $c_i$ of literal $l_k$, $t_i^k$ is the term
$t_i$ of $l_k$, and $d^k$ is the constant $d$ of $l_k$. The operator
$\bowtie$ is $≈$ if all operators are $≈$, $>$ if all are
either $≈$ or $>$, and $\geq$ otherwise. The $a_i$ must be such
that the sum on the left-hand side is $0$ and the right-hand side is $>0$ (or
$\geq 0$ if $\bowtie$ is $>$).
\end{RuleDescription}

% Just like in ConTeX we can't have minted in the rule environment for some
% reason.
\begin{RuleExample}
A simple \proofRule{la_generic} step in the logic \textsf{LRA} might look like this:

\begin{AletheVerb}
(step t10 (cl (not (> (f a) (f b))) (not (= (f a) (f b))))
    :rule la_generic :args (1.0 -1.0))
\end{AletheVerb}

To verify this we have to check the insatisfiability of $(f\,a) > (f\,b) \land
(f\,a) ≈ (f\,b)$ (step 2). After step~3 we get $(f\,a) - (f\,b) > 0 \land
(f\,a) - (f\,b) ≈ 0$. Since we don't have an integer sort in this logic step~4 does
not apply. Finally, after step~5 the conjunction is $(f\,a) - (f\,b) > 0 \land
- (f\,a) + (f\,b) ≈ 0$. This sums to $0 > 0$, which is a contradiction.
\end{RuleExample}

\begin{RuleExample}
The following \proofRule{la_generic} step is from a \textsf{QF\_UFLIA} problem:
\begin{AletheVerb}
(step t11 (cl (not (<= f3 0)) (<= (+ 1 (* 4 f3)) 1))
    :rule la_generic :args (1.0 1/4))
\end{AletheVerb}
After normalization we get $-f_3 \geq 0 \land 4\times f_3 > 0$.
This time step~4 applies and we can strengthen this to
$-f_3 \geq 0 \land 4\times f_3 \geq 1$ and after multiplication we get
$-f_3 \geq 0 \land f_3 \geq ¼$. Which sums to the contradiction
$¼ \geq 0$.
\end{RuleExample}

\begin{RuleDescription}{lia_generic}
This rule is a placeholder rule for integer arithmetic solving. It takes the
same form as \proofRule{la_generic}, without the additional arguments.

\begin{AletheX}
$i$. & \ctxsep & $\varphi_1$, \dots , $\varphi_o$  & (\currule) \\
\end{AletheX}
with $\varphi_i$ being linear inequalities. The disjunction
$\varphi_1\lor \dots \lor \varphi_n$ is a tautology in the theory of linear
integer arithmetic.

\ruleparagraph{Remark.} Since this rule can introduce a disjunction of arbitrary
linear integer inequalities without any additional hints, proof checking
can be NP-hard. Hence, this rule should be avoided when possible.

\end{RuleDescription}

\begin{RuleDescription}{la_disequality}
\begin{AletheX}
$i$. & \ctxsep  &
$t_1 ≈ t_2 \lor \neg (t_1 \leq t_2) \lor \neg (t_2 \leq t_1)$
& (\currule) \\
\end{AletheX}
\end{RuleDescription}

\begin{RuleDescription}{la_totality}
\begin{AletheX}
$i$. & \ctxsep & $t_1 \leq t_2 \lor t_2 \leq t_1$ & (\currule) \\
\end{AletheX}
\end{RuleDescription}

\begin{RuleDescription}{la_tautology}
This rule is a linear arithmetic tautology which can be checked without
sophisticated reasoning. It has either the form

\begin{AletheX}
$i$. & \ctxsep & $\varphi$  & (\currule) \\
\end{AletheX}
where $\varphi$ is either a linear inequality $s_1 \bowtie s_2$
or $\neg(s_1 \bowtie s_2)$. After performing step 1 to 3 of the process for
checking the \proofRule{la_generic} the result is trivially unsatisfiable.

The second form handles bounds on linear combinations. It is binary clause:

\begin{AletheX}
$i$. & \ctxsep &
$\varphi_1 \lor \varphi_2$ % Checked: this is a proper disjunction not a cl
 & (\currule) \\
\end{AletheX}

It can be checked by using the procedure for \proofRule{la_generic}
while setting the arguments to $1$. Informally, the rule follows one of several
cases:
\begin{itemize}
    \item $\neg (s_1 \leq d_1) \lor s_1 \leq d_2$ where $d_1 \leq d_2$
    \item $s_1 \leq d_1 \lor \neg (s_1 \leq d_2)$ where $d_1 = d_2$
    \item $\neg (s_1 \geq d_1) \lor s_1 \geq d_2$ where $d_1 \geq d_2$
    \item $s_1 \geq d_1 \lor \neg (s_1 \geq d_2)$ where $d_1 = d_2$
    \item $\neg (s_1 \leq d_1) \lor \neg(s_1 \geq d_2)$ where $d_1 < d_2$
\end{itemize}
The inequalities $s_1 \bowtie d$ are the result of applying normalization
as for the rule \proofRule{la_generic}.
\end{RuleDescription}

\begin{RuleDescription}{la_mult_pos}

Either of the form:

  \begin{AletheX}
    $i$. & \ctxsep  &
     $(t_1 > 0 \wedge t_2 \bowtie t_3) \to t_1 * t_2 \bowtie t_1 * t_3$
    & (\currule) \\
  \end{AletheX}

with $\bowtie \in \{<,>, \le,\ge, ≈\}$.\\

\noindent Or of the form:

\begin{AletheX}
  $i$. & \ctxsep  &
  $(t_1 > 0 \land \neg (t_2 ≈ t_3)) \rightarrow \neg (t_1 * t_2 ≈ t_1 * t_3)$ 
  & (\currule) \\
\end{AletheX}

\end{RuleDescription}

\begin{RuleDescription}{la_mult_neg}

Either of the form:

  \begin{AletheX}
    $i$. & \ctxsep  &
    $(t_1 < 0 \land t_2 \bowtie t_3) \rightarrow t_1 * t_2 \bowtie_{inv} t_1 * t_3$
    & (\currule) \\
  \end{AletheX}

  with $\bowtie \in \{<,>, \le,\ge, ≈\}$ and $\bowtie_{inv}$ being defined according to the following table.

  \begin{center}
  \begin{tabular}{|c | c|}
  \hline
  $\bowtie$ & $\bowtie_{inv}$ is defined as: \\
  \hline
  $ < $ & $ > $ \\
  $ \le $ & $ \ge $ \\
  $ ≈ $ & $ ≈ $ \\
  $ > $ & $ < $ \\
  $ \ge $ & $ \le $ \\
  \hline
  \end{tabular}
  \end{center}

\noindent Or of the form:

  \begin{AletheX}
    $i$. & \ctxsep  &
    $(t_1 < 0 \land \neg (t_2 ≈ t_3)) \rightarrow \neg (t_1 * t_2 ≈ t_1 * t_3)$
    & (\currule) \\
  \end{AletheX}
\end{RuleDescription}

\begin{RuleDescription}{bind}
  The \currule{} rule is used to rename bound variables.

\begin{AletheXS}
\aletheLineSB
$j$. & \spctx{ $\Gamma, y_1,\dots, y_n,  x_1 \mapsto y_1, \dots ,  x_n \mapsto y_n$}
   & \ctxsep & $\varphi ≈ \varphi'$ & ($\dots$) \\
   \spsep
$k$. & $\Gamma$ & \ctxsep &
    $Q x_1, \dots, x_n.\varphi ≈ Q y_1, \dots, y_n. \varphi'$
     & \currule{} \\
\end{AletheXS}

\noindent
where $Q \in \{\forall, \exists\}$,
and the variables $y_1, \dots, y_n$ are neither free in $Q x_1,
\dots, x_n.\varphi$ nor occur in $\Gamma$.
\end{RuleDescription}

\begin{RuleDescription}{sko_ex}
The \currule{} rule skolemizes existential quantifiers.

\begin{AletheXS}
\aletheLineS
$j$. &
\spctx{$\Gamma, x_1 \mapsto \varepsilon_1, \dots ,  x_n \mapsto \varepsilon_n$}
   & \ctxsep &  $\varphi ≈ \psi$ & ($\dots$) \\
   \spsep
$k$. & $\Gamma$ & \ctxsep & $\exists x_1, \dots, x_n.\varphi ≈ \psi$ & \currule{} \\
\end{AletheXS}

\noindent
where $\varepsilon_i$ stands for $\varepsilon x_i. (\exists x_{i+1}, \dots,
x_n. \varphi)$.
\end{RuleDescription}

\begin{RuleDescription}{sko_forall}
The \currule{} rule skolemizes universal quantifiers.

\begin{AletheXS}
\aletheLineS
$j$. &
\spctx{$\Gamma, x_1 \mapsto (\varepsilon x_1.\neg\varphi), \dots,  x_n \mapsto (\varepsilon x_n.\neg\varphi)$}
   & \ctxsep & $\varphi ≈ \psi$ & ($\dots$) \\
 \spsep
$k$. & $\Gamma$ & \ctxsep  & $\forall x_1, \dots, x_n.\varphi ≈ \psi$ & \currule{} \\
\end{AletheXS}

\end{RuleDescription}

\begin{RuleDescription}{forall_inst}
\begin{AletheX}
$i$. & \ctxsep &
$\neg (\forall x_1, \dots, x_n. P) \lor P[x_1\mapsto t_1]\dots[x_n\mapsto t_n]$
 & \currule\, [$t_1$, $\dots$, $t_n$] \\
\end{AletheX}

\noindent
where $x_i$ and $t_i$ have the same sort.
\end{RuleDescription}

\begin{RuleExample}
An application of the \proofRule{forall_inst} rule.
\begin{AletheVerb}
(step t16 (cl (or (not (forall ((x S) (y T)) (P y x    )))
                                             (P b (f a))
      :rule forall_inst :args ((f a) b)
\end{AletheVerb}
\end{RuleExample}


\begin{RuleDescription}{refl}
\begin{AletheXS}
$j$. & \ctxsep  & $\Gamma$  & $t_1 ≈ t_2 $ & \currule \\
\end{AletheXS}

\noindent
where, if $\sigma = \subst(\Gamma)$,
the terms $t_1\sigma$ and $t_2$ are
syntactically equal up to  renaming of
bound variables and the orientation of equalities.

\ruleparagraph{Remark.} This is the only rule that requires the application of
the context.
\end{RuleDescription}

\begin{RuleDescription}{trans}
\begin{AletheXS}
$i_1$. & \ctxsep  & $\Gamma$  & $t_1 ≈ t_2 $ & ($\dots$) \\
$i_2$. & \ctxsep  & $\Gamma$  & $t_2 ≈ t_3 $ & ($\dots$) \\
\aletheLineS
$i_n$. & \ctxsep  & $\Gamma$  & $t_n ≈ t_{n+1} $ & ($\dots$) \\
$j$. & \ctxsep  & $\Gamma$  & $t_1 ≈ t_{n+1}$ & (\currule\; $i_1$, $\dots$, $i_n$) \\
\end{AletheXS}
\end{RuleDescription}

\begin{RuleDescription}{cong}
\begin{AletheXS}
$i_1$. & \ctxsep  & $\Gamma$  & $t_1 ≈ u_1$ & ($\dots$) \\
$i_2$. & \ctxsep  & $\Gamma$  & $t_2 ≈ u_2 $ & ($\dots$) \\
\aletheLineS
$i_n$. & \ctxsep  & $\Gamma$  & $t_n ≈ u_n $ & ($\dots$) \\
$j$. & \ctxsep  & $\Gamma$  & $(f\,t_1\,\cdots\,t_n) ≈ (f\,u_1\,\cdots\,u_n)$  & (\currule\; $i_1$, $\dots$, $i_n$) \\
\end{AletheXS}
where $f$ is any function symbol of appropriate sort.
\end{RuleDescription}

\begin{RuleDescription}{eq_reflexive}
\begin{AletheX}
$i$. & \ctxsep  & $t ≈ t$  & \currule \\
\end{AletheX}
\end{RuleDescription}

\begin{RuleDescription}{eq_transitive}
\begin{AletheX}
$i$. & \ctxsep & $\neg (t_1 ≈ t_2)$, $\dots$ , $\neg (t_{n-1} ≈ t_n)$,
$t_1 ≈ t_n$ & \currule \\
\end{AletheX}
\end{RuleDescription}

\begin{RuleDescription}{eq_congruent}
\begin{AletheX}
$i$. & \ctxsep &
$\neg (t_1 ≈ u_1)$, $\dots$ ,$\neg (t_n ≈ u_n)$,
$(f\,t_1\, \cdots\,t_n) ≈ (f\,u_1\, \cdots\,u_n)$
 & \currule \\
\end{AletheX}
\end{RuleDescription}

\begin{RuleDescription}{eq_congruent_pred}
\begin{AletheX}

$i$. & \ctxsep     &
$\neg (t_1 ≈ u_1)$, \dots ,$\neg (t_n ≈ u_n)$,
$(P\,t_1\, \cdots\,t_n) ≈ (P\,u_1\, \cdots\,u_n)$
 & \currule \\
\end{AletheX}
where $P$\, is a function symbol with co-domain sort $\lsymb{Bool}$.
\end{RuleDescription}

\begin{RuleDescription}{qnt_cnf}
\begin{AletheX}
$i$. & \ctxsep &
$\neg(\forall x_1, \dots, x_n. \varphi) \lor \forall x_{k_1}, \dots, x_{k_m}.\varphi'$
 & \currule \\
\end{AletheX}

This rule expresses clausification of a term under a universal
quantifier. This is used by conflicting instantiation. $\varphi'$ is one of the clause
of the clause normal form of $\varphi$. The variables $x_{k_1}, \dots, x_{k_m}$ are
a permutation of $x_1, \dots, x_n$ plus additional variables added by prenexing
$\varphi$. Normalization is performed in two phases. First, the negative normal form
is formed, then the result is prenexed. The result of the first step is $\Phi(\varphi, 1)$
where:

\begin{align*}
\Phi(\neg \varphi, 1) &:= \Phi(\varphi, 0) \\
\Phi(\neg \varphi, 0) &:= \Phi(\varphi, 1) \\
\Phi(\varphi_1 \lor\dots\lor\varphi_n, 1) &:= \Phi(\varphi_1, 1)\lor\dots\lor\Phi(\varphi_n, 1) \\
\Phi(\varphi_1 \land\dots\land\varphi_n, 1) &:= \Phi(\varphi_1, 1)\land\dots\land\Phi(\varphi_n, 1) \\
\Phi(\varphi_1 \lor\dots\lor\varphi_n, 0) &:= \Phi(\varphi_1, 0)\land\dots\land\Phi(\varphi_n, 0) \\
\Phi(\varphi_1 \land\dots\land\varphi_n, 0) &:= \Phi(\varphi_1, 0)\lor\dots\lor\Phi(\varphi_n, 0) \\
\Phi(\varphi_1 \rightarrow \varphi_2, 1) &:= (\Phi(\varphi_1, 0) \lor \Phi(\varphi_2, 1)) \land
                                                    (\Phi(\varphi_2, 0) \lor \Phi(\varphi_1, 1)) \\
\Phi(\varphi_1 \rightarrow \varphi_2, 0) &:= (\Phi(\varphi_1, 1) \land \Phi(\varphi_2, 0)) \lor
                                                    (\Phi(\varphi_2, 1) \land \Phi(\varphi_1, 0)) \\
\Phi(\lsymb{ite}\,\varphi_1\,\varphi_2\,\varphi_3, 1) &:=
             (\Phi(\varphi_1, 0) \lor \Phi(\varphi_2, 1)) \land (\Phi(\varphi_1, 1) \lor \Phi(\varphi_3, 1)) \\
\Phi(\lsymb{ite}\,\varphi_1\,\varphi_2\,\varphi_3, 0) &:=
             (\Phi(\varphi_1, 1) \land \Phi(\varphi_2, 0)) \lor (\Phi(\varphi_1, 0) \land \Phi(\varphi_3, 0)) \\
\Phi(\forall x_1, \dots, x_n. \varphi, 1) &:= \forall x_1, \dots, x_n. \Phi(\varphi, 1) \\
\Phi(\exists x_1, \dots, x_n. \varphi, 1) &:= \exists x_1, \dots, x_n. \Phi(\varphi, 1) \\
\Phi(\forall x_1, \dots, x_n. \varphi, 0) &:= \exists x_1, \dots, x_n. \Phi(\varphi, 0) \\
\Phi(\exists x_1, \dots, x_n. \varphi, 0) &:= \forall x_1, \dots, x_n. \Phi(\varphi, 0) \\
\Phi(\varphi, 1) &:= \varphi \\
\Phi(\varphi, 0) &:= \neg\varphi
\end{align*}

\ruleparagraph{Remark.} This is a placeholder rule that combines the many steps
done during clausification.
\end{RuleDescription}

\begin{RuleDescription}{and}
\begin{AletheX}
$i$. & \ctxsep & $\varphi_1 \land \cdots \land \varphi_n$ & ($\dots$) \\
$j$. & \ctxsep & $\varphi_k$ & (\currule\;$i$)\, k\\
\end{AletheX}
and $1 \leq k \leq n$.
\end{RuleDescription}

\begin{RuleDescription}{not_or}
\begin{AletheX}
$i$. & \ctxsep & $\neg (\varphi_1 \lor \cdots \lor \varphi_n)$ & ($\dots$) \\
$j$. & \ctxsep & $\neg \varphi_k$  & (\currule\;$i$)\, k\\
\end{AletheX}
and $1 \leq k \leq n$.
\end{RuleDescription}

\begin{RuleDescription}{or}
\begin{AletheX}
$i$. & \ctxsep &
$\varphi_1 \lor \cdots \lor \varphi_n$ & ($\dots$) \\
$j$. & \ctxsep & $\varphi_1, \dots, \varphi_n$ & (\currule\;$i$) \\
\end{AletheX}

\ruleparagraph{Remark.} This rule deconstructs the \inlineAlethe{or} operator
into a clause denoted by \inlineAlethe{cl}.
\end{RuleDescription}

\begin{RuleExample}
An application of the \proofRule{or} rule.
\begin{AletheVerb}
(step t15 (cl (or (= a b) (not (<= a b)) (not (<= b a))))
    :rule la_disequality)
(step t16 (cl     (= a b) (not (<= a b)) (not (<= b a)))
     :rule or :premises (t15))
\end{AletheVerb}
\end{RuleExample}

\begin{RuleDescription}{weakening}
\begin{AletheX}
$i$. & \ctxsep &
$\varphi_1, \cdots, \varphi_n$ & ($\dots$) \\
$j$. & \ctxsep & $\varphi_1, \cdots, \varphi_n, \psi_1, \dots, \psi_m$ & (\currule\;$i$) \\
\end{AletheX}
where $m \geq 1$.
\end{RuleDescription}

\begin{RuleDescription}{reordering}
\begin{AletheX}
$i$. & \ctxsep & $\varphi_1, \cdots, \varphi_n$ & ($\dots$) \\
$j$. & \ctxsep & $\psi_1, \cdots, \psi_n$ & (\currule\;$i$) \\
\end{AletheX}
where the multisets $\{\varphi_1, \cdots, \varphi_n\}$ and $\{\psi_1, \cdots, \psi_n\}$
are the same.  That is, the conclusion of the rule is a reordering of the
literals in the premise.
\end{RuleDescription}

\begin{RuleDescription}{not_and}
\begin{AletheX}
$i$. & \ctxsep & $\neg (\varphi_1 \land \dots \land \varphi_n)$ & ($\dots$) \\
$j$. & \ctxsep & $\neg\varphi_1 , \dots , \neg\varphi_n$ & (\currule\;$i$) \\
\end{AletheX}
\end{RuleDescription}

\begin{RuleDescription}{xor1}
\begin{AletheX}
$i$. & \ctxsep & ($\lsymb{xor}\,\varphi_1\,\varphi_2$)
 & ($\dots$) \\
$j$. & \ctxsep & $\varphi_1, \varphi_2$ & (\currule\;$i$) \\
\end{AletheX}
\end{RuleDescription}

\begin{RuleDescription}{xor2}
\begin{AletheX}
$i$. & \ctxsep & ($\lsymb{xor}\,\varphi_1\,\varphi_2$) & ($\dots$) \\
$j$. & \ctxsep & $\neg\varphi_1 , \neg\varphi_2$ & (\currule\;$i$) \\
\end{AletheX}
\end{RuleDescription}

\begin{RuleDescription}{not_xor1}
\begin{AletheX}
$i$. & \ctxsep & $\neg(\lsymb{xor}\,\varphi_1\,\varphi_2)$ & ($\dots$) \\
$j$. & \ctxsep & $\varphi_1 , \neg\varphi_2$  & (\currule\;$i$) \\
\end{AletheX}
\end{RuleDescription}

\begin{RuleDescription}{not_xor2}
\begin{AletheX}
$i$. & \ctxsep & $\neg(\lsymb{xor}\,\varphi_1\,\varphi_2)$ & ($\dots$) \\
$j$. & \ctxsep & $\neg\varphi_1 , \varphi_2$ & (\currule\;$i$) \\
\end{AletheX}
\end{RuleDescription}

\begin{RuleDescription}{implies}
\begin{AletheX}
$i$. & \ctxsep & $\varphi_1\rightarrow\varphi_2$ & ($\dots$) \\
$j$. & \ctxsep & $\neg\varphi_1, \varphi_2$ & (\currule\;$i$) \\
\end{AletheX}
\end{RuleDescription}

\begin{RuleDescription}{not_implies1}
\begin{AletheX}
$i$. & \ctxsep & $\neg (\varphi_1\rightarrow\varphi_2)$ & ($\dots$) \\
$j$. & \ctxsep & $\varphi_1$ & (\currule\;$i$) \\
\end{AletheX}
\end{RuleDescription}

\begin{RuleDescription}{not_implies2}
\begin{AletheX}
$i$. & \ctxsep & $\neg (\varphi_1\rightarrow\varphi_2)$ & ($\dots$) \\
$j$. & \ctxsep & $\neg\varphi_2$ & (\currule\;$i$) \\
\end{AletheX}
\end{RuleDescription}

\begin{RuleDescription}{equiv1}
\begin{AletheX}
$i$. & \ctxsep & $\varphi_1≈\varphi_2$ & ($\dots$) \\
$j$. & \ctxsep & $\neg\varphi_1, \varphi_2$ & (\currule\;$i$) \\
\end{AletheX}
\end{RuleDescription}

\begin{RuleDescription}{equiv2}
\begin{AletheX}
$i$. & \ctxsep & $\varphi_1≈\varphi_2$ & ($\dots$) \\
$j$. & \ctxsep & $\varphi_1, \neg\varphi_2$ & (\currule\;$i$) \\
\end{AletheX}
\end{RuleDescription}

\begin{RuleDescription}{not_equiv1}
\begin{AletheX}
$i$. & \ctxsep & $\neg(\varphi_1≈\varphi_2)$ & ($\dots$) \\
$j$. & \ctxsep & $\varphi_1 , \varphi_2$ & (\currule\;$i$) \\
\end{AletheX}
\end{RuleDescription}

\begin{RuleDescription}{not_equiv2}
\begin{AletheX}
$i$. & \ctxsep & $\neg(\varphi_1≈\varphi_2)$ & ($\dots$) \\
$j$. & \ctxsep & $\neg\varphi_1, \neg\varphi_2$ & (\currule\;$i$) \\
\end{AletheX}
\end{RuleDescription}

\begin{RuleDescription}{and_pos}
\begin{AletheX}
$i$. & \ctxsep & $\neg (\varphi_1 \land \cdots \land \varphi_n) , \varphi_k$ &
  \currule\, k\\
\end{AletheX}
with $1 \leq k \leq n$.
\end{RuleDescription}

\begin{RuleDescription}{and_neg}
\begin{AletheX}
$i$. & \ctxsep & $(\varphi_1 \land \cdots \land \varphi_n), \neg\varphi_1, \dots , \neg\varphi_n$ & \currule \\
\end{AletheX}
\end{RuleDescription}

\begin{RuleDescription}{or_pos}
\begin{AletheX}
$i$. & \ctxsep & $\neg (\varphi_1 \lor \cdots \lor \varphi_n) , \varphi_1 , \dots, \varphi_n$ & \currule \\
\end{AletheX}
\end{RuleDescription}

\begin{RuleDescription}{or_neg}
\begin{AletheX}
$i$. & \ctxsep & $(\varphi_1 \lor \cdots \lor \varphi_n), \neg \varphi_k$ &
  \currule\, k\\
\end{AletheX}
with $1 \leq k \leq n$.
\end{RuleDescription}

\begin{RuleDescription}{xor_pos1}
\begin{AletheX}
$i$. & \ctxsep & $\neg (\lsymb{xor}\,\varphi_1\,\varphi_2) , \varphi_1 , \varphi_2$ & \currule \\
\end{AletheX}
\end{RuleDescription}

\begin{RuleDescription}{xor_pos2}
\begin{AletheX}
$i$. & \ctxsep & $\neg (\lsymb{xor}\,\varphi_1\,\varphi_2), \neg \varphi_1, \neg \varphi_2$ & \currule \\
\end{AletheX}
\end{RuleDescription}

\begin{RuleDescription}{xor_neg1}
\begin{AletheX}
$i$. & \ctxsep & $(\lsymb{xor}\,\varphi_1\,\varphi_2), \varphi_1 , \neg \varphi_2$ & \currule \\
\end{AletheX}
\end{RuleDescription}

\begin{RuleDescription}{xor_neg2}
\begin{AletheX}
$i$. & \ctxsep & $(\lsymb{xor}\,\varphi_1\,\varphi_2), \neg \varphi_1, \varphi_2$ & \currule \\
\end{AletheX}
\end{RuleDescription}

\begin{RuleDescription}{implies_pos}
\begin{AletheX}
$i$. & \ctxsep & $\neg (\varphi_1 \rightarrow \varphi_2), \neg \varphi_1 , \varphi_2$ & \currule \\
\end{AletheX}
\end{RuleDescription}

\begin{RuleDescription}{implies_neg1}
\begin{AletheX}
$i$. & \ctxsep & $\varphi_1 \rightarrow \varphi_2, \varphi_1$ & \currule \\
\end{AletheX}
\end{RuleDescription}

\begin{RuleDescription}{implies_neg2}
\begin{AletheX}
$i$. & \ctxsep & $\varphi_1 \rightarrow \varphi_2, \neg \varphi_2$ & \currule \\
\end{AletheX}
\end{RuleDescription}

\begin{RuleDescription}{equiv_pos1}
\begin{AletheX}
$i$. & \ctxsep & $\neg (\varphi_1 ≈ \varphi_2), \varphi_1, \neg \varphi_2$ & \currule \\
\end{AletheX}
\end{RuleDescription}

\begin{RuleDescription}{equiv_pos2}
\begin{AletheX}
$i$. & \ctxsep & $\neg (\varphi_1 ≈ \varphi_2), \neg \varphi_1, \varphi_2$ & \currule \\
\end{AletheX}
\end{RuleDescription}

\begin{RuleDescription}{equiv_neg1}
\begin{AletheX}
$i$. & \ctxsep & $\varphi_1 ≈ \varphi_2, \neg \varphi_1, \neg \varphi_2$ & \currule \\
\end{AletheX}
\end{RuleDescription}

\begin{RuleDescription}{equiv_neg2}
\begin{AletheX}
$i$. & \ctxsep & $\varphi_1 ≈ \varphi_2, \varphi_1, \varphi_2$ & \currule \\
\end{AletheX}
\end{RuleDescription}

\begin{RuleDescription}{ite1}
\begin{AletheX}
$i$. & \ctxsep & $(\lsymb{ite}\,\varphi_1\,\varphi_2\,\varphi_3)$ & ($\dots$) \\
$j$. & \ctxsep & $\varphi_1 , \varphi_3$ & (\currule\;$i$) \\
\end{AletheX}
\end{RuleDescription}

\begin{RuleDescription}{ite2}
\begin{AletheX}
$i$. & \ctxsep & $(\lsymb{ite}\,\varphi_1\,\varphi_2\,\varphi_3)$ & ($\dots$) \\
$j$. & \ctxsep & $\neg\varphi_1, \varphi_2$ & (\currule\;$i$) \\
\end{AletheX}
\end{RuleDescription}

\begin{RuleDescription}{ite_pos1}
\begin{AletheX}
$i$. & \ctxsep & $\neg (\lsymb{ite}\,\varphi_1\,\varphi_2\,\varphi_3), \varphi_1, \varphi_3$ & (\currule) \\
\end{AletheX}
\end{RuleDescription}

\begin{RuleDescription}{ite_pos2}
\begin{AletheX}
$i$. & \ctxsep & $\neg (\lsymb{ite}\,\varphi_1\,\varphi_2\,\varphi_3), \neg \varphi_1, \varphi_2$ & (\currule) \\
\end{AletheX}
\end{RuleDescription}

\begin{RuleDescription}{ite_neg1}
\begin{AletheX}
$i$. & \ctxsep & $(\lsymb{ite}\,\varphi_1\,\varphi_2\,\varphi_3, \varphi_1, \neg \varphi_3)$ & (\currule) \\
\end{AletheX}
\end{RuleDescription}

\begin{RuleDescription}{ite_neg2}
\begin{AletheX}
$i$. & \ctxsep & $(\lsymb{ite}\,\varphi_1\,\varphi_2\,\varphi_3, \neg \varphi_1, \neg \varphi_2)$ & (\currule) \\
\end{AletheX}
\end{RuleDescription}

\begin{RuleDescription}{not_ite1}
\begin{AletheX}
$i$. & \ctxsep & $\neg(\lsymb{ite}\,\varphi_1\,\varphi_2\,\varphi_3)$ & ($\dots$) \\
$j$. & \ctxsep & $\varphi_1, \neg\varphi_3$ & (\currule\;$i$) \\
\end{AletheX}
\end{RuleDescription}

\begin{RuleDescription}{not_ite2}
\begin{AletheX}
$i$. & \ctxsep & $\neg(\lsymb{ite}\,\varphi_1\,\varphi_2\,\varphi_3)$ & ($\dots$) \\
$j$. & \ctxsep & $\neg\varphi_1 , \neg\varphi_2$ & (\currule\;$i$) \\
\end{AletheX}
\end{RuleDescription}

\begin{RuleDescription}{connective_def}
  This rule is used to replace connectives by their definition. It can be one
  of the following:
\begin{AletheXS}
$i$. & \ctxsep  & $\Gamma$  &
    $(\lsymb{xor}\,\varphi_1\,\varphi_2) ≈
    ((\neg\varphi_1 \land \varphi_2) \lor (\varphi_1 \land \neg\varphi_2))$ & \currule \\
\end{AletheXS}

\begin{AletheXS}
$i$. & \ctxsep  & $\Gamma$  &
      $(\varphi_1 ≈ \varphi_2) ≈
      ((\varphi_1 \rightarrow \varphi_2) \land (\varphi_2 \rightarrow \varphi_1))$ & \currule \\
\end{AletheXS}

\begin{AletheXS}
$i$. & \ctxsep  & $\Gamma$  &
      $(\lsymb{ite}\,\varphi_1\,\varphi_2\,\varphi_3) ≈
      ((\varphi_1 \rightarrow \varphi_2) \land (\neg\varphi_1 \rightarrow \varphi_3))$ & \currule \\
\end{AletheXS}

\begin{AletheXS}
$i$. & \ctxsep  & $\Gamma$  &
      $(\forall x_1, \dots, x_n.\,\varphi) ≈ \neg(\exists x_1, \dots, x_n.\,
      \neg\varphi)$ & \currule \\
\end{AletheXS}
\end{RuleDescription}

\begin{RuleDescription}{and_simplify}
This rule simplifies an $\land$ term by applying equivalence-preserving
transformations until a fixed point is reached. Hence, the general form is

\begin{AletheXS}
$i$. & \ctxsep  & $\Gamma$  & $\varphi_1\land \cdots\land\varphi_n ≈ \psi$ & \currule \\
\end{AletheXS}
where $\psi$ is the transformed term.

The possible transformations are:
\begin{itemize}
    \item $\top \land \cdots \land \top ⇒ \top$
    \item $\varphi_1 \land \cdots \land \varphi_n ⇒ \varphi_1
    \land \cdots \land \varphi_{n'} $ where the right-hand side has all
    $\top$ literals removed.
    \item $\varphi_1 \land \cdots \land \varphi_n ⇒ \varphi_1
    \land \cdots \land \varphi_{n'} $ where the right-hand side has all
    repeated literals removed.
    \item $\varphi_1 \land\cdots\land \bot\land\cdots \land \varphi_n ⇒ \bot$
    \item $\varphi_1 \land\cdots\land \varphi_i\land \cdots \land \varphi_j\land\cdots \land \varphi_n ⇒ \bot$
  and $\varphi_i$, $\varphi_j$ are such that
  \begin{align*}
  \varphi_i &= \underbrace{\neg \dots \neg}_n \psi \\
  \varphi_j &= \underbrace{\neg \dots \neg}_m \psi
  \end{align*}
  and one of $n, m$ is odd and the other even.  Either can be $0$.
\end{itemize}
\end{RuleDescription}

\begin{RuleDescription}{or_simplify}
This rule simplifies an $\lor$ term by applying equivalence-preserving
transformations until a fixed point is reached. Hence, the general form is

\begin{AletheXS}
$i$. & \ctxsep  & $\Gamma$  & $(\varphi_1\lor \cdots\lor\varphi_n) ≈ \psi$ & \currule \\
\end{AletheXS}
where $\psi$ is the transformed term.

The possible transformations are:
\begin{itemize}
    \item $\bot \lor \cdots \lor \bot ⇒ \bot$
    \item $\varphi_1 \lor \cdots \lor \varphi_n ⇒ \varphi_1
    \lor \cdots \lor \varphi_{n'} $ where the right-hand side has all
    $\bot$ literals removed.
    \item $\varphi_1 \lor \cdots \lor \varphi_n ⇒ \varphi_1
    \lor \cdots \lor \varphi_{n'} $ where the right-hand side has all
    repeated literals removed.
    \item $\varphi_1 \lor\cdots\lor \top\lor\cdots \lor \varphi_n ⇒ \top$
    \item $\varphi_1 \lor\cdots\lor \varphi_i\lor \cdots \lor \varphi_j\lor\cdots \lor \varphi_n ⇒ \top$
  and $\varphi_i$, $\varphi_j$ are such that
  \begin{align*}
  \varphi_i &= \underbrace{\neg \dots \neg}_n \psi \\
  \varphi_j &= \underbrace{\neg \dots \neg}_m \psi
  \end{align*}
  and one of $n, m$ is odd and the other even.  Either can be $0$.
\end{itemize}
\end{RuleDescription}

\begin{RuleDescription}{not_simplify}
This rule simplifies an $\neg$ term by applying equivalence-preserving
transformations until a fixed point is reached. Hence, the general form is

\begin{AletheXS}
$i$. & \ctxsep  & $\Gamma$ & $\neg\varphi ≈ \psi$ & \currule \\
\end{AletheXS}
where $\psi$ is the transformed term.

The possible transformations are:
\begin{itemize}
    \item $\neg (\neg \varphi) ⇒ \varphi$
    \item $\neg \bot ⇒ \top$
    \item $\neg \top ⇒ \bot$
\end{itemize}
\end{RuleDescription}

\begin{RuleDescription}{implies_simplify}
This rule simplifies an $\rightarrow$ term by applying equivalence-preserving
transformations until a fixed point is reached. Hence, the general form is

\begin{AletheXS}
$i$. & \ctxsep  & $\Gamma$ & $\varphi_1\rightarrow \varphi_2 ≈ \psi$ & \currule \\
\end{AletheXS}
where $\psi$ is the transformed term.

The possible transformations are:
\begin{itemize}
    \item $\neg \varphi_1 \rightarrow \neg \varphi_2 ⇒  \varphi_2\rightarrow \varphi_1$
    \item $\bot \rightarrow  \varphi ⇒ \top$
    \item $ \varphi \rightarrow \top ⇒ \top$
    \item $\top \rightarrow  \varphi ⇒  \varphi$
    \item $ \varphi \rightarrow \bot ⇒ \neg \varphi$
    \item $ \varphi \rightarrow  \varphi ⇒ \top$
    \item $\neg \varphi \rightarrow  \varphi ⇒  \varphi$
    \item $ \varphi \rightarrow \neg \varphi ⇒ \neg \varphi$
\end{itemize}
\end{RuleDescription}

\begin{RuleDescription}{equiv_simplify}
This rule simplifies a formula with the head symbol $≈\,: \lsymb{Bool}\,\lsymb{Bool}\,\lsymb{Bool}$
by applying equivalence-preserving
transformations until a fixed point is reached. Hence, the general form is

\begin{AletheXS}
$i$. & \ctxsep  & $\Gamma$ & $(\varphi_1≈ \varphi_2) ≈ \psi$ & \currule \\
\end{AletheXS}
where $\psi$ is the transformed term.

The possible transformations are:
\begin{itemize}
    \item $(\neg \varphi_1 ≈ \neg \varphi_2) ⇒ ( \varphi_1≈ \varphi_2)$
    \item $( \varphi≈ \varphi) ⇒ \top$
    \item $( \varphi≈ \neg \varphi) ⇒ \bot$
    \item $(\neg \varphi≈  \varphi) ⇒ \bot$
    \item $(\top ≈  \varphi) ⇒  \varphi$
    \item $( \varphi ≈ \top) ⇒  \varphi$
    \item $(\bot ≈  \varphi) ⇒ \neg \varphi$
    \item $( \varphi ≈ \bot) ⇒ \neg \varphi$
\end{itemize}
\end{RuleDescription}

\begin{RuleDescription}{bool_simplify}
This rule simplifies a boolean term by applying equivalence-preserving
transformations until a fixed point is reached. Hence, the general form is

\begin{AletheXS}
$i$. & \ctxsep & $\Gamma$ & $\varphi≈ \psi$ & \currule \\
\end{AletheXS}
where $\psi$ is the transformed term.

The possible transformations are:
\begin{itemize}
  \item $\neg(\varphi_1\rightarrow \varphi_2) ⇒ (\varphi_1 \land \neg \varphi_2)$
  \item $\neg(\varphi_1\lor \varphi_2) ⇒ (\neg \varphi_1 \land \neg \varphi_2)$
  \item $\neg(\varphi_1\land \varphi_2) ⇒ (\neg \varphi_1 \lor \neg \varphi_2)$
  \item $(\varphi_1 \rightarrow (\varphi_2\rightarrow \varphi_3)) ⇒ (\varphi_1\land \varphi_2) \rightarrow \varphi_3$
  \item $((\varphi_1\rightarrow \varphi_2)\rightarrow \varphi_2)  ⇒ (\varphi_1\lor \varphi_2)$
  \item $(\varphi_1 \land (\varphi_1\rightarrow \varphi_2)) ⇒ (\varphi_1 \land \varphi_2)$
  \item $((\varphi_1\rightarrow \varphi_2) \land \varphi_1) ⇒ (\varphi_1 \land \varphi_2)$
\end{itemize}
\end{RuleDescription}

\begin{RuleDescription}{ac_simp}
  This rule simplifies nested occurrences of $\lor$ or $\land$:

\begin{AletheXS}
$i$. & \ctxsep & $\Gamma$ & $\psi ≈ \varphi_1 \circ\cdots\circ\varphi_n$ & \currule \\
\end{AletheXS}
  where $\circ \in \{\lor, \land\}$ and $\psi$ is a nested application of $\circ$.
  The literals $\varphi_i$ are literals of the flattening of $\psi$ with duplicates
  removed.
\end{RuleDescription}

\begin{RuleDescription}{ite_simplify}
  This rule simplifies an if-then-else term by applying equivalence-preserving
  transformations until fixed point\footnote{Note however that the order of the
    application is important, since the set of rules is not confluent. For
    example, the term $(\lsymb{ite} \top \; t_1 \; t_2 ≈
    t_1)$ can be simplified into both $p$ and $(\neg (\neg p))$ depending on the
    order of applications.}
  %
It has the form

\begin{AletheXS}
$i$. & \ctxsep & $\Gamma$ & $(\lsymb{ite}\,\varphi\,t_1\,t_2) ≈ u$ & \currule \\
\end{AletheXS}
where $u$ is the transformed term.

The possible transformations are:
\begin{itemize}
    \item $(\lsymb{ite}\, \top      \, t_1 \, t_2) ⇒ t_1$
    \item $(\lsymb{ite}\, \bot      \, t_1 \, t_2) ⇒ t_2$
    \item $(\lsymb{ite}\, \psi      \, t \, t) ⇒ t$
    \item $(\lsymb{ite}\, \neg \varphi \, t_1 \, t_2) ⇒ (\lsymb{ite}\, \varphi \, t_2 \, t_1)$
    \item $(\lsymb{ite}\, \psi \, (\lsymb{ite}\, \psi\,t_1\,t_2)\, t_3) ⇒
      (\lsymb{ite}\, \psi\, t_1\, t_3)$
    \item $(\lsymb{ite}\, \psi \, t_1\, (\lsymb{ite}\, \psi\,t_2\,t_3) ⇒
      (\lsymb{ite}\, \psi\, t_1\, t_3)$
    \item $(\lsymb{ite}\, \psi \, \top\, \bot) ⇒ \psi$
    \item $(\lsymb{ite}\, \psi \, \bot\, \top) ⇒ \neg\psi$
    \item $(\lsymb{ite}\, \psi \, \top \, \varphi) ⇒ \psi\lor\varphi$
    \item $(\lsymb{ite}\, \psi \, \varphi\,\bot) ⇒ \psi\land\varphi$
    \item $(\lsymb{ite}\, \psi \, \bot\, \varphi) ⇒ \neg\psi\land\varphi$
    \item $(\lsymb{ite}\, \psi \, \varphi\,\top) ⇒ \neg\psi\lor\varphi$
\end{itemize}
\end{RuleDescription}

\begin{RuleDescription}{qnt_simplify}
  This rule simplifies a $\forall$-formula with a constant predicate.

\begin{AletheXS}
$i$. & \ctxsep & $\Gamma$ & $(\forall x_1, \dots, x_n. \varphi) ≈ \varphi$ & \currule \\
\end{AletheXS}
  where $\varphi$ is either $\top$ or $\bot$.
\end{RuleDescription}

\begin{RuleDescription}{onepoint}
The {\currule} rule is the ``one-point-rule''. That is: it eliminates quantified
variables that can only have one value.

\begin{AletheXS}
\aletheLineS
$j$. & \spctx{$\Gamma, x_{k_1},\dots, x_{k_m},  x_{j_1} \mapsto t_{j_1}, \dots ,  x_{j_o} \mapsto t_{j_o}$}
   & \ctxsep & $\varphi ≈ \varphi'$ & ($\dots$) \\
 \spsep
$k$. & $\Gamma$ & \ctxsep  & $Q x_1, \dots, x_n.\varphi ≈ Q x_{k_1}, \dots, x_{k_m}. \varphi'$ & \currule{} \\
\end{AletheXS}
where $Q\in\{\forall, \exists\}$,  $n = m + o$,  $k_1, \dots, k_m$ and
$j_1, \dots, j_o$ are monotone
mappings to $1, \dots, n$, and no $x_{k_i}$ appears in $x_{j_1}, \dots, x_{j_o}$.

The terms $t_{j_1}, \dots, t_{j_o}$ are the points of the variables
$x_{j_1}, \dots, x_{j_o}$. Points are defined by equalities $x_i≈ t_i$
with positive polarity in the term $\varphi$.

\ruleparagraph{Remark.} Since an eliminated variable $x_i$ might appear free in a
term $t_j$, it is necessary to replace $x_i$ with $t_i$ inside $t_j$. While
this substitution is performed correctly, the proof for it is currently
missing.
\end{RuleDescription}

\begin{RuleExample}
An application of the \proofRule{onepoint} rule on the term $(\forall x, y.\, x ≈ y
\rightarrow (f\,x)\land (f\,y))$ look like this:

\begin{AletheVerb}
(anchor :step t3 :args ((x S) (:= (y S) x)))
(step t3.t1 (cl (= x y)) :rule refl)
(step t3.t2 (cl (= (= x y) (= x x)))
    :rule cong :premises (t3.t1))
(step t3.t3 (cl (= x y)) :rule refl)
(step t3.t4 (cl (= (f y) (f x)))
    :rule cong :premises (t3.t3))
(step t3.t5 (cl (= (and (f x) (f y)) (and (f x) (f x))))
    :rule cong :premises (t3.t4))
(step t3.t6 (cl (= (=> (= x y) (and (f x) (f y)))
                   (=> (= x x) (and (f x) (f x)))))
    :rule cong :premises (t3.t2 t3.t5))
(step t3 (cl (=
        (forall ((x S) (y S)) (=> (= x y) (and (f x) (f y))))
        (forall ((x S))       (=> (= x x) (and (f x) (f x))))))
    :rule onepoint)
\end{AletheVerb}
\end{RuleExample}

\begin{RuleDescription}{qnt_join}
\begin{AletheXS}
$i$. & \ctxsep & $\Gamma$ & $Q x_1, \dots, x_n.\,(Q x_{n+1}, \dots, x_{m}.\,\varphi)
      ≈ Q x_{k_1}, \dots, x_{k_o}.\,\varphi$ & \currule \\
\end{AletheXS}
  where $m > n$ and $Q\in\{\forall, \exists\}$. Furthermore, $k_1, \dots, k_o$ is a monotonic
  map to $1, \dots, m$ such that $x_{k_1}, \dots, x_{k_o}$ are pairwise
  distinct, and $\{x_1, \dots, x_m\} = \{x_{k_1}, \dots, x_{k_o}\}$.
\end{RuleDescription}

\begin{RuleDescription}{qnt_rm_unused}
\begin{AletheXS}
$i$. & \ctxsep & $\Gamma$ & $Q x_1, \dots, x_n.\,\varphi ≈ Q x_{k_1}, \dots, x_{k_m}.\,\varphi$
 & \currule \\
\end{AletheXS}
  where $m \leq n$ and $Q\in\{\forall, \exists\}$. Furthermore, $k_1, \dots, k_m$ is
  a monotonic map to $1, \dots, n$ and if $x\in \{x_j\; |\; j \in \{1, \dots,
  n\} \land j\in\not \{k_1, \dots, k_m\}\}$ then $x$ is not free in $P$.
\end{RuleDescription}

\begin{RuleDescription}{eq_simplify}
  This rule simplifies an $≈$ term by applying equivalence-preserving
  transformations until a fixed point is reached. Hence, the general form is

\begin{AletheXS}
$i$. & \ctxsep & $\Gamma$ & $(t_1≈ t_2) ≈ \varphi$ & \currule \\
\end{AletheXS}
  where $\psi$ is the transformed term.

  The possible transformations are:
  \begin{itemize}
  \item $t ≈ t ⇒ \top$
  \item $(t_1 ≈ t_2) ⇒ \bot$ if $t_1$ and $t_2$ are different numeric constants.
  \item $\neg (t ≈ t) ⇒ \bot$ if $t$ is a numeric constant.
  \end{itemize}
\end{RuleDescription}

\begin{RuleDescription}{div_simplify}
This rule simplifies a division by applying equivalence-preserving
transformations. The general form is

\begin{AletheXS}
$i$. & \ctxsep & $\Gamma$ & $(t_1\, /\,  t_2) ⇒ t_3$ & \currule \\
\end{AletheXS}
The possible transformations are:
\begin{itemize}
  \item $t\, /\, t ⇒ 1$
  \item $t\, /\, 1 ⇒ t$
  \item $t_1\,  /\,  t_2 ⇒ t_3$
    if $t_1$ and $t_2$ are constants and $t_3$ is $t_1$
    divided by $t_2$ according to the semantics of the current theory.
\end{itemize}
\end{RuleDescription}

\begin{RuleDescription}{prod_simplify}
This rule simplifies a product by applying equivalence-preserving
transformations until a fixed point is reached. The general form is

\begin{AletheXS}
$i$. & \ctxsep & $\Gamma$ & $t_1\times\cdots\times t_n ≈ u$ & \currule \\
\end{AletheXS}
where $u$ is either a constant or a product.

The possible transformations are:
\begin{itemize}
    \item $t_1\times\cdots\times t_n ⇒ u$ where all
    $t_i$ are constants and $u$ is their product.
    \item $t_1\times\cdots\times t_n ⇒ 0$ if any
    $t_i$ is $0$.
    \item $t_1\times\cdots\times t_n ⇒
      c \times t_{k_1}\times\cdots\times t_{k_n}$ where $c$
      is the product of the constants of $t_1, \dots, t_n$ and
      $t_{k_1}, \dots, t_{k_n}$ is $t_1, \dots, t_n$
      with the constants removed.
    \item $t_1\times\cdots\times t_n ⇒
      t_{k_1}\times\cdots\times t_{k_n}$: same as above if $c$ is $1$.
\end{itemize}
\end{RuleDescription}

\begin{RuleDescription}{unary_minus_simplify}
This rule is either

\begin{AletheXS}
$i$. & \ctxsep & $\Gamma$ & $- (-t) ≈ t$ & \currule \\
\end{AletheXS}
or

\begin{AletheXS}
$i$. & \ctxsep & $\Gamma$ & $-t ≈ u$ & \currule \\
\end{AletheXS}
where $u$ is the negated numerical constant $t$.
\end{RuleDescription}

\begin{RuleDescription}{minus_simplify}
This rule simplifies a subtraction by applying equivalence-preserving
transformations. The general form is

\begin{AletheXS}
$i$. & \ctxsep & $\Gamma$ & $t_1- t_2 ≈ u$ & \currule \\
\end{AletheXS}
The possible transformations are:
\begin{itemize}
    \item $t - t ⇒ 0$
    \item $t_1 - t_2 ⇒ t_3$ where $t_1$
    and $t_2$ are numerical constants and $t_3$ is $t_2$ subtracted
    from~$t_1$.
    \item $t - 0 ⇒ t$
    \item $0 - t ⇒ -t$
\end{itemize}
\end{RuleDescription}

\begin{RuleDescription}{sum_simplify}
This rule simplifies a sum by applying equivalence-preserving
transformations until a fixed point is reached. The general form is

\begin{AletheXS}
$i$. & \ctxsep & $\Gamma$ & $t_1+\cdots+t_n ≈ u$ & \currule \\
\end{AletheXS}
where $u$ is either a constant or a product.

The possible transformations are:
\begin{itemize}
    \item $t_1+\cdots+t_n ⇒ c$ where all
    $t_i$ are constants and $c$ is their sum.
    \item $t_1+\cdots+t_n ⇒
      c + t_{k_1}+\cdots+t_{k_n}$ where $c$
      is the sum of the constants of $t_1, \dots, t_n$ and
      $t_{k_1}, \dots, t_{k_n}$ is $t_1, \dots, t_n$
      with the constants removed.
    \item $t_1+\cdots+t_n ⇒
      t_{k_1}+\cdots+t_{k_n}$: same as above if $c$ is
      $0$.
\end{itemize}
\end{RuleDescription}

\begin{RuleDescription}{comp_simplify}
This rule simplifies a comparison by applying equivalence-preserving
transformations until a fixed point is reached. The general form is

\begin{AletheXS}
$i$. & \ctxsep & $\Gamma$ & $t_1 \bowtie t_2 ≈ \psi$ & \currule \\
\end{AletheXS}
where $\bowtie \in \{<, >, \leq, \geq\}$.

The following transformation rules are used to simplify $t_1 \bowtie t_2$:
\begin{itemize}
    \item $s_1 < s_2 ⇒ \top$ if $s_1, s_2$ are numerical constants and
      $s_1$ is strictly less than $s_2$
    \item $s_1 < s_2 ⇒ \bot$ if $s_1, s_2$ are numerical constants and
      $s_1$ is greater or equal than $s_2$

    \item $s_1 \leq s_2 ⇒ \top$ if $s_1, s_2$ are numerical constants and
      $s_1$ is less or equal than $s_2$
    \item $s_1 \leq s_2 ⇒ \bot$ if $s_1, s_2$ are numerical constants and
      $s_1$ is strictly greater than $s_2$

    \item $s < s ⇒ \bot$
    \item $s \leq s ⇒ \top$
    \item $s_1 \geq s_2 ⇒ s_2 \leq s_1$
    \item $s_1 < s_2 ⇒ \neg (s_2 \leq s_1)$
    \item $s_1 > s_2 ⇒ \neg (s_1 \leq s_2)$
\end{itemize}
\end{RuleDescription}

\begin{RuleDescription}{let}
  This rule eliminates $\lsymb{let}$. It has the form

\begin{AletheXS}
$i_1$. & $\Gamma$ & \ctxsep & $t_{1} ≈ s_{1}$ & ($\dots$) \\
\aletheLineS
$i_n$. & $\Gamma$ & \ctxsep & $t_{n} ≈ s_{n}$ & ($\dots$) \\
\aletheLineS
$j$. & \spctx{$\Gamma, x_1 \mapsto s_1, \dots,  x_n \mapsto s_n$}
   & \ctxsep &  $u ≈ u'$ & ($\dots$) \\
\spsep
$k$. & $\Gamma$ & \ctxsep &
     $(\lsymb{let}\,x_1 = t_1,\, \dots,\, x_n = t_n\,\lsymb{in}\, u) ≈ u'$
     & (\currule{}\;$i_1$, \dots, $i_n$) \\
\end{AletheXS}

  The premise $i_1, \dots, i_n$ must be in the same subproof as
  the \currule{} step.  If for $t_i≈ s_i$ the $t_i$ and $s_i$
  are syntactically equal, the premise
  is omitted.
\end{RuleDescription}

\begin{RuleDescription}{bind_let}
  This rule corresponds to the \proofRule{bind} rule for \inlineAlethe{let}.
  It allows the renaming of the variables bound by the \inlineAlethe{let} step,
  the rewriting of the substituted terms, and the rewriting of the body of the 
  \inlineAlethe{let}, resulting in a new \inlineAlethe{let} term.  
  It has the form

\begin{AletheXS}
$i_1$. & $\Gamma$ & \ctxsep & $t_{1} ≈ s_{1}$ & ($\dots$) \\
\aletheLineS
$i_n$. & $\Gamma$ & \ctxsep & $t_{n} ≈ s_{n}$ & ($\dots$) \\
\aletheLineS
$j$. & \spctx{$\Gamma, y_1,\dots, y_n,  x_1 \mapsto y_1, \dots ,  x_n \mapsto y_n$}
   & \ctxsep &  $u ≈ u'$ & ($\dots$) \\
\spsep
$k$. & $\Gamma$ & \kern-5.5em\ctxsep & % The kern hacking here is to make the rule readable
     $ \kern-3em (\lsymb{let}\,x_1~=~t_1,\, \dots,\, x_n~=~t_n \lsymb{in}\, u) ≈
                 (\lsymb{let}\,y_1~=~s_1,\, \dots,\, y_n~=~s_n\,\lsymb{in}\, u')$
     & (\currule{}\;$i_1$, \dots, $i_n$) \\
\end{AletheXS}

  The variables $y_1, \dots, y_n$ are neither free in
  $(\lsymb{let}\,x_1 = t_1,\, \dots,\, x_n = t_n\,\lsymb{in}\, u)$ nor, for each
  $y_i$ different from $x_i$, occur in $\Gamma$.

  The premise $i_1, \dots, i_n$ must be in the same subproof as
  the \currule{} step.  If for $t_i≈ s_i$ the $t_i$ and $s_i$
  are syntactically equal, the premise
  is omitted.
\end{RuleDescription}

\begin{RuleExample}
The following example shows how this rule is used in a proof generated
by Carcara's elaborator.  It elaborates an implicit application of
symmetry of equality.
\begin{AletheVerb}
(step t1 (cl (= (= 0 1) (= 1 0))) :rule eq_symmetric)
(anchor :step t2 :args ((p Bool) (:= (p Bool) p)))
(step t2.t1 (cl (= (= p false) (= false p))) :rule eq_symmetric)
(step t2 (cl (= (let ((p (= 0 1))) (= p false))
                (let ((p (= 1 0))) (= false p))))
         :rule bind_let :premises (t1))
\end{AletheVerb}
 
\end{RuleExample}

\begin{RuleDescription}{distinct_elim}
This rule eliminates the $\lsymb{distinct}$ predicate. If called with one
argument this predicate always holds:

\begin{AletheXS}
$i$. & \ctxsep & $\Gamma$ & $(\lsymb{distinct}\, t) ≈ \top$ & \currule \\
\end{AletheXS}

If applied to terms of type $\lsymb{Bool}$ more than two terms can never be
distinct, hence only two cases are possible:

\begin{AletheXS}
$i$. & \ctxsep & $\Gamma$ &
$(\lsymb{distinct}\,\varphi\,\psi) ≈ \neg (\varphi ≈ \psi)$ & \currule \\
\end{AletheXS}
and

\begin{AletheXS}
$i$. & \ctxsep & $\Gamma$ &
$(\lsymb{distinct}\,\varphi_1\,\varphi_2\,\varphi_3\,\dots) ≈ \bot$ & \currule \\
\end{AletheXS}

The general case is

\begin{AletheXS}
$i$. & \ctxsep & $\Gamma$ &
$(\lsymb{distinct}\,t_1\,\dots\, t_n) ≈
\bigwedge_{i=1}^{n}\bigwedge_{j=i+1}^{n} t_i\;{\not≈}\;t_j$ & \currule \\
\end{AletheXS}
\end{RuleDescription}

\begin{RuleDescription}{la_rw_eq}
\begin{AletheX}
$i$. & \ctxsep & $(t ≈ u) ≈ (t \leq u \land u \leq t)$ & \currule \\
\end{AletheX}
\end{RuleDescription}

\begin{RuleDescription}{nary_elim}
This rule replaces $n$-ary operators with their equivalent
application of the binary operator. It is never applied to $\land$ or $\lor$.

Three cases are possible.
If the operator $\circ$ is left associative, then the rule has the form
\begin{AletheXS}
$i$. & \ctxsep & $\Gamma$ & $\bigcirc_{i=1}^{n} t_i ≈ (\dots( t_1\circ  t_2) \circ  t_3)\circ \cdots  t_n)$
 & \currule \\
\end{AletheXS}

If the operator $\circ$ is right associative, then the rule has the form

\begin{AletheXS}
$i$. & \ctxsep  & $\Gamma$ & $\bigcirc_{i=1}^{n} t_i ≈
( t_1 \circ \cdots \circ ( t_{n-2} \circ ( t_{n-1} \circ  t_n)\dots)$ & \currule \\
\end{AletheXS}

If the operator is {\em chainable}, then it has the form

\begin{AletheXS}
$i$. & \ctxsep & $\Gamma$ & $\bigcirc_{i=1}^{n} t_i ≈
( t_1\circ t_2) \land ( t_2 \circ  t_3) \land \cdots
\land ( t_{n-1}\circ t_n)$ & \currule \\
\end{AletheXS}
\end{RuleDescription}

\begin{RuleDescription}{bfun_elim}
\begin{AletheX}
$i$. & \ctxsep & $\psi$ & ($\dots$) \\
$j$. & \ctxsep & $\varphi$ & (\currule\; $i$) \\
\end{AletheX}

The formula $\varphi$ is $\psi$ after boolean functions have been simplified.
This happens in a two step process. Both steps recursively iterate over $\psi$.
The first step expands quantified variable of type $\lsymb{Bool}$. Hence,
$(\exists x.\,t)$ becomes $t[x\mapsto \bot]\lor t[x\mapsto \top]$ and
$(\forall x.\,t)$ becomes $t[x\mapsto \bot]\land t[x\mapsto \top]$. If $n$ variables of sort
$\lsymb{Bool}$ appear in a quantifier, the disjunction (conjunction) has
$2^n$ terms. Each term replaces the variables in $t$ according
to the bits of a number which is increased by one for each subsequent
term starting from zero. The left-most variable corresponds to the
least significant bit.

The second step expands function argument of boolean types by introducing
appropriate if-then-else terms. For example, consider $(f\,x\, P\, y)$ where
$P$ is some formula. Then we replace this term by $(\lsymb{ite}\, P\,
(f\,x\, \top\ y)\,(f\,x\, \bot\, y))$. If the argument is already the constant $\top$
or $\bot$, it is ignored.
\end{RuleDescription}

\begin{RuleDescription}{ite_intro}
\begin{AletheX}
$i$. & \ctxsep & $t ≈ (t' \land u_1 \land \dots \land u_n)$ & (\currule) \\
\end{AletheX}

The term $t$ (the formula $\varphi$) contains the $\lsymb{ite}$ operator.
Let $s_1, \dots, s_n$ be the terms starting with $\lsymb{ite}$, i.e.
$s_i := \lsymb{ite}\,\psi_i\,r_i\,r'_i$, then $u_i$ has the form
\[
  \lsymb{ite}\,\psi_i\,(s_i ≈ r_i)\,(s_i ≈ r'_i)
\]
The term $t'$ is equal to the term $t$ up to the
reordering of equalities where one argument is an $\lsymb{ite}$
term.

\ruleparagraph{Remark.} This rule stems from the introduction of fresh
constants for if-then-else terms inside veriT. Internally $s_i$ is a new
constant symbol and the $\varphi$ on the right side of the equality is
$\varphi$ with the if-then-else terms replaced by the constants. Those
constants are unfolded during proof printing. Hence, the slightly strange
form and the reordering of equalities.
\end{RuleDescription}

\begin{RuleDescription}{bitblast_extract}
\begin{AletheX}
$i$. & \ctxsep & $((\lsymb{extract}\ j\ i)\ x) ≈ (\lsymb{bbT}\ \varphi_i\ \ldots\ \varphi_j)$ & (\currule) \\
\end{AletheX}

\noindent
where the formulas $\varphi_k$ are $(\lsymb{bitOf}_k\ x)$ for $i \leq k \leq j$.

Alternatively, the rule may also be phrased as a ``short-circuiting'' of the
above when $x$ is a $\lsymb{bbT}$ application:

\medskip
\begin{AletheX}
$i$. & \ctxsep & $((\lsymb{extract}\ j\ i)\ (\lsymb{bbT}\ x_0\ \ldots\
x_i\ \ldots \ x_j\ \ldots\ x_n)) ≈ (\lsymb{bbT}\ x_i\ \ldots\ x_j)$ & (\currule) \\
\end{AletheX}

\noindent
This alternative is based on the validity of the equality
\[
\lsymb{bitOf}_k\ (\lsymb{bbT}\ x_0\ \ldots\ x_i\ \ldots \ x_j\ \ldots\ x_n) ≈ x_k
\]
for any bit-vector $x$ of size $n+1$, where $0\leq k\leq n$.

\end{RuleDescription}

\begin{RuleDescription}{bitblast_ult}
\begin{AletheX}
$i$. & \ctxsep & $(\lsymb{bvult}\ x\ y) ≈ \mathrm{res}_{n-1}$ & (\currule) \\
\end{AletheX}
in which both $x$ and $y$ must have the same type $(\lsymb{BitVec}\ n)$ and, for
$i\geq 0$
\[
  \begin{array}{lcl}
    \mathrm{res}_0&=&\neg (\lsymb{bitOf}_0\ x) \wedge (\lsymb{bitOf}_0\ y)\\
    \mathrm{res}_{i+1}&=&(((\lsymb{bitOf}_{i+1}\ x) ≈ (\lsymb{bitOf}_{i+1}\ y))\wedge \mathrm{res}_i)\vee
                       (\neg (\lsymb{bitOf}_{i+1}\ x) \wedge (\lsymb{bitOf}_{i+1}\ y))
  \end{array}
\]

\noindent
Alternatively, the rule may also be phrased as a ``short-circuiting'' of the
above when $x$ and $y$ are ``$\lsymb{bbT}$'' applications. So given that
\[
  \begin{array}{lcl}
    x&=&(\lsymb{bbT}\ x_0\ \ldots\ x_i\ \ldots \ x_j\ \ldots\ x_n)\\
    y&=&(\lsymb{bbT}\ y_0\ \ldots\ y_i\ \ldots \ y_j\ \ldots\ y_n)\\
  \end{array}
\]
then ``$\mathrm{res}$'' can be defined, for $i \geq 0$, as 
\[
  \begin{array}{lcl}
    \mathrm{res}_0&=&\neg x_0 \wedge y_0\\
    \mathrm{res}_{i+1}&=&((x_{i+1} ≈ y_{i+1})\wedge
                          \mathrm{res}_i)\vee (\neg x_{i+1}\wedge y_{i+1})
  \end{array}
\]

\end{RuleDescription}

\begin{RuleDescription}{bitblast_add}
\begin{AletheX}
$i$. & \ctxsep & $(\lsymb{bvadd}\ x\ y) ≈ A_1$ & (\currule) \\
\end{AletheX}
in which both $x$ and $y$ must have the same type $(\lsymb{BitVec}\ n)$.
The term ``$A_1$'' is
\begin{align*}
(\lsymb{bbT}\;& (((\lsymb{bitOf}_{0}\ x) \,\lsymb{xor}\,(\lsymb{bitOf}_{0}\ y))\,\lsymb{xor}\,\mathrm{carry}_0) \\
              & (((\lsymb{bitOf}_{1}\ x) \,\lsymb{xor}\,(\lsymb{bitOf}_{1}\ y))\,\lsymb{xor}\,\mathrm{carry}_1) \\
              & \ldots \\
              & (((\lsymb{bitOf}_{n-1}\ x) \,\lsymb{xor}\, (\lsymb{bitOf}_{n-1}\ y))\,\lsymb{xor}\,\mathrm{carry}_{n-1}))
\end{align*}
and for $i\geq 0$
\[
  \begin{array}{lcl}
    \mathrm{carry}_0&=&\bot\\
    \mathrm{carry}_{i+1}&=&((\lsymb{bitOf}_{i}\ x) \wedge (\lsymb{bitOf}_{i}\ y))\vee(((\lsymb{bitOf}_{i}\ x)\,\lsymb{xor}\, (\lsymb{bitOf}_{i}\ y))\wedge \mathrm{carry}_i)
  \end{array}
\]

\noindent
Alternatively, the rule may also be phrased as a ``short-circuiting'' of the
above when $x$ and $y$ are ``$\lsymb{bbT}$'' applications. So given that
\[
  \begin{array}{lcl}
    x&=&(\lsymb{bbT}\ x_0\ \ldots\ x_i\ \ldots \ x_j\ \ldots\ x_n)\\
    y&=&(\lsymb{bbT}\ y_0\ \ldots\ y_i\ \ldots \ y_j\ \ldots\ y_n)\\
  \end{array}
\]
then the rule can be alternatively phrased as

\begin{AletheX}
$i$. & \ctxsep & $(\mathrm{bvadd}\ x\ y) ≈ A_2$ & (\currule) \\
\end{AletheX}
with $A_2 := (\lsymb{bbT}\ (x_0 \,\lsymb{xor}\, y_0)\,\lsymb{xor}\,\mathrm{carry}_0\ \ldots\ (x_{n-1}
        \,\lsymb{xor}\, y_{n-1})\,\lsymb{xor}\,\mathrm{carry}_{n-1})$ and
``$\mathrm{carry}$'' being defined, for $i \geq 0$, as 
\[
  \begin{array}{lcl}
    \mathrm{carry}_0&=&\bot\\
    \mathrm{carry}_{i+1}&=&(x_i\wedge y_i)\vee((x_i\,\lsymb{xor}\, y_i)\wedge \mathrm{carry}_i)
  \end{array}
\]

\end{RuleDescription}

% TODO
\begin{RuleDescription}{concat_eq}
\begin{AletheX}
$i$. & \ctxsep & $((\lsymb{extract}\ j\ i)\ x) ≈ (\lsymb{bbT}\ \varphi_i\ \ldots\ \varphi_j)$ & (\currule) \\
\end{AletheX}
\end{RuleDescription}

\begin{RuleDescription}{concat_unify}
\begin{AletheX}
$i$. & \ctxsep & $((\lsymb{extract}\ j\ i)\ x) ≈ (\lsymb{bbT}\ \varphi_i\ \ldots\ \varphi_j)$ & (\currule) \\
\end{AletheX}
\end{RuleDescription}

\begin{RuleDescription}{concat_conflict}
\begin{AletheX}
$i$. & \ctxsep & $((\lsymb{extract}\ j\ i)\ x) ≈ (\lsymb{bbT}\ \varphi_i\ \ldots\ \varphi_j)$ & (\currule) \\
\end{AletheX}
\end{RuleDescription}

\begin{RuleDescription}{concat_split_prefix}
\begin{AletheX}
$i$. & \ctxsep & $((\lsymb{extract}\ j\ i)\ x) ≈ (\lsymb{bbT}\ \varphi_i\ \ldots\ \varphi_j)$ & (\currule) \\
\end{AletheX}
\end{RuleDescription}

\begin{RuleDescription}{concat_split_suffix}
\begin{AletheX}
$i$. & \ctxsep & $((\lsymb{extract}\ j\ i)\ x) ≈ (\lsymb{bbT}\ \varphi_i\ \ldots\ \varphi_j)$ & (\currule) \\
\end{AletheX}
\end{RuleDescription}

\begin{RuleDescription}{concat_csplit_prefix}
\begin{AletheX}
$i$. & \ctxsep & $((\lsymb{extract}\ j\ i)\ x) ≈ (\lsymb{bbT}\ \varphi_i\ \ldots\ \varphi_j)$ & (\currule) \\
\end{AletheX}
\end{RuleDescription}

\begin{RuleDescription}{concat_csplit_suffix}
\begin{AletheX}
$i$. & \ctxsep & $((\lsymb{extract}\ j\ i)\ x) ≈ (\lsymb{bbT}\ \varphi_i\ \ldots\ \varphi_j)$ & (\currule) \\
\end{AletheX}
\end{RuleDescription}

\begin{RuleDescription}{concat_lprop_prefix}
\begin{AletheX}
$i$. & \ctxsep & $((\lsymb{extract}\ j\ i)\ x) ≈ (\lsymb{bbT}\ \varphi_i\ \ldots\ \varphi_j)$ & (\currule) \\
\end{AletheX}
\end{RuleDescription}

\begin{RuleDescription}{concat_lprop_suffix}
\begin{AletheX}
$i$. & \ctxsep & $((\lsymb{extract}\ j\ i)\ x) ≈ (\lsymb{bbT}\ \varphi_i\ \ldots\ \varphi_j)$ & (\currule) \\
\end{AletheX}
\end{RuleDescription}

\begin{RuleDescription}{concat_cprop_prefix}
\begin{AletheX}
$i$. & \ctxsep & $((\lsymb{extract}\ j\ i)\ x) ≈ (\lsymb{bbT}\ \varphi_i\ \ldots\ \varphi_j)$ & (\currule) \\
\end{AletheX}
\end{RuleDescription}

\begin{RuleDescription}{concat_cprop_suffix}
\begin{AletheX}
$i$. & \ctxsep & $((\lsymb{extract}\ j\ i)\ x) ≈ (\lsymb{bbT}\ \varphi_i\ \ldots\ \varphi_j)$ & (\currule) \\
\end{AletheX}
\end{RuleDescription}

\begin{RuleDescription}{string_length_pos}
\begin{AletheX}
$i$. & \ctxsep & $((\lsymb{extract}\ j\ i)\ x) ≈ (\lsymb{bbT}\ \varphi_i\ \ldots\ \varphi_j)$ & (\currule) \\
\end{AletheX}
\end{RuleDescription}

\begin{RuleDescription}{string_length_non_empty}
\begin{AletheX}
$i$. & \ctxsep & $((\lsymb{extract}\ j\ i)\ x) ≈ (\lsymb{bbT}\ \varphi_i\ \ldots\ \varphi_j)$ & (\currule) \\
\end{AletheX}
\end{RuleDescription}

\begin{RuleDescription}{re_inter}
\begin{AletheX}
$i$. & \ctxsep & $((\lsymb{extract}\ j\ i)\ x) ≈ (\lsymb{bbT}\ \varphi_i\ \ldots\ \varphi_j)$ & (\currule) \\
\end{AletheX}
\end{RuleDescription}

\begin{RuleDescription}{re_unfold_neg}
\begin{AletheX}
$i$. & \ctxsep & $((\lsymb{extract}\ j\ i)\ x) ≈ (\lsymb{bbT}\ \varphi_i\ \ldots\ \varphi_j)$ & (\currule) \\
\end{AletheX}
\end{RuleDescription}

\begin{RuleDescription}{re_unfold_neg_concat_fixed_prefix}
\begin{AletheX}
$i$. & \ctxsep & $((\lsymb{extract}\ j\ i)\ x) ≈ (\lsymb{bbT}\ \varphi_i\ \ldots\ \varphi_j)$ & (\currule) \\
\end{AletheX}
\end{RuleDescription}

\begin{RuleDescription}{re_unfold_neg_concat_fixed_suffix}
\begin{AletheX}
$i$. & \ctxsep & $((\lsymb{extract}\ j\ i)\ x) ≈ (\lsymb{bbT}\ \varphi_i\ \ldots\ \varphi_j)$ & (\currule) \\
\end{AletheX}
\end{RuleDescription}

\begin{RuleDescription}{symm}
\begin{AletheX}
$i$. & \ctxsep & $ \varphi ≈ \psi$ & ($\dots$) \\
$j$. & \ctxsep & $ \psi ≈ \varphi$ & (\currule\; $i$) \\
\end{AletheX}

\noindent
If $\varphi \neq \psi$ then the conclusion \emph{must not} be $\varphi ≈ \psi$.

Note that since Alethe allows the implicit reordering of equalities, this
rule is technically superfluous.  However, the rule is useful to indicate
an explicit usage of the symmetry of equality to aid proof reconstruction.

\begin{RuleExample}
The side condition ensures that the following example is not a valid application
of the rule.  Without this condition, this derivation could be obtained by
applying symmetry of equality implicitly to the conclusion.

\begin{AletheX}
$10$. & \ctxsep & $ P(a) ≈ Q(b)$ & ($\dots$) \\
$11$. & \ctxsep & $ P(a) ≈ Q(b)$ & (\currule\; $10$) \\
\end{AletheX}
\end{RuleExample}
\end{RuleDescription}

\begin{RuleDescription}{not_symm}
\begin{AletheX}
$i$. & \ctxsep & $ \neg (\varphi ≈ \psi)$  & ($\dots$) \\
$j$. & \ctxsep & $ \neg (\psi ≈ \varphi)$  & (\currule\; $i$) \\
\end{AletheX}

\noindent
If $\varphi \neq \psi$ then the conclusion \emph{must not} be $\neg (\varphi ≈ \psi)$.

See \proofRule{symm} for an explanation of this rule.

\end{RuleDescription}

\begin{RuleDescription}{eq_symmetric}
\begin{AletheX}
$i$. & \ctxsep & $ (t_1 ≈ t_2) ≈ (t_2 ≈ t_1)$ & (\currule) \\
\end{AletheX}

\noindent
Note that since Alethe allows the implicit reordering of equalities, this
rule is technically superfluous.  However, the rule is useful to state
an explicit usage of symmetry of equality reasoning to aid proof reconstruction.
\end{RuleDescription}

\clearpage
\subsection{Index of Rules}
\label{sec:alethe:rules-index}
\printindex[rules]


%%% Local Variables:
%%% mode: latex
%%% TeX-master: "doc"
%%% End:


\clearpage
\section*{Changelog}
\label{sec:rules}
\addcontentsline{toc}{section}{Changelog}
\subsection*{Unreleased}

Proof rules:
\begin{itemize}
  \item Addition of a section describing bitvector proofs.
  \item Bitblasting rules: \proofRule{bitblast_extract},  \proofRule{bitblast_add},
      \proofRule{bitblast_ult}.
  \item Addition of rules \proofRule{la_mult_pos} and \proofRule{la_mult_neg} to describe multiplication with a positive or negative factor.
  \item Addition of rules \proofRule{symm} and \proofRule{not_symm} to express symmetry of equality explicitly.
  \item Addition of the rule \proofRule{eq_symmetric} to express symmetry
  of equality explicitly but as an equivalence. Note that in principle this
  could be done with the rule \proofRule{symm} above, but would require a long
  and tedious use of \proofRule{subproof} for each direction of the equivalence.
  \item Addition of the rule \proofRule{weakening} to express weakening of a
        clause.
  \item Addition of the \proofRule{reordering} rule to represent reordering of
        the literals in a clause.
  \item Addition of the \proofRule{bind_let} rule.  This rule can be used to
        preprocess \inlineAlethe{let} expressions similar to the \proofRule{bind}
        rule used with ordinary quantifiers.
\end{itemize}

\noindent
Breaking changes:
\begin{itemize}
  \item Allow arbitrary extra annotations in \texttt{assume} commands.
  \item Add the sort to all variables in contexts.  Before, the context
        of a bind could be \texttt{(x S) (:= y x)}.  Now it must
        be \texttt{(x S) (:= (y S) x)}.
  \item The arguments for \proofRule{forall_inst} have been changed to
        no longer take the shape of bindings using \texttt{(:= x c)}.
        Instead, the list of instantiation terms must follow the variable
        order and cover all the respective bound variables.
  \item The rules \proofRule{and_pos}, \proofRule{or_neg}, \proofRule{and},
        \proofRule{not_or} now have one argument that indicates which subterm
        they use.
  \item Add new syntax for decimal and negative numbers and use it
        for \proofRule{la_generic}.
  \item Restrict sorting of numbers such that the sort of a constant is
        only determined by its syntactic category.
  \item Add new syntax for decimal and negative numbers and clarify that
        the set of allowed symbols is restricted to not conflict with this
        new syntax.
  \item Always use decimals for the constants in \proofRule{la_generic}.
  \item Restrict sorting of numbers such that the sort of a constant is
        only determined by its syntactic category.
\end{itemize}

\noindent
Clarifications and corrected errors:
\begin{itemize}
  \item Clarify that the \texttt{:args} annotation in \texttt{anchor} can be
        omitted if the list is empty.
  \item Clarify that \proofRule{bind} can work on existential and universal
        quantifiers.
  \item Fix mistake in proof grammar.  It now uses the \texttt{context\_annotation}
        non-terminal in the rule for the \texttt{anchor} command.
  \item Fix the example of \proofRule{onepoint}.
  \item Add the missing context to the conclusion of \proofRule{bind},
        \proofRule{sko_ex}, \proofRule{sko_forall}, \proofRule{onepoint}.
\end{itemize}

\subsection*{0.3 --- \DTMdisplaydate{2023}{02}{10}{-1}}

This release overhauls the entire document, but introduces only few
changes to the proof format itself.

The standard now specifies that \texttt{assume} commands can only be
issued at the start of the proof or right after an \texttt{anchor}
command.

Beyond many smaller clarifications and typographic improvements, the
following changes are implemented in this release.
\begin{itemize}
  \item Add an abstract proof checking procedure to clarify
  the semantics of the proof format.
  \item Add a description of the semantics of contexts based on λ-terms.
  \item List all transformations that are implicit in Alethe proofs.
  \item Change the notation used for terms from first-order style
   (e.g., $f(x,g(y))$) to higher-order style (e.g.,
   $(f\;x\;(g\;y))$).  This is only a change in notation -- the used logic
   remains many-sorted first-order logic.
  \item Eliminate the distinction between if-and-only-if and
  equality.  Instead, use equality (the symbol $≈$) with appropriate
   sorts.
  \item Add an index that lists all proof rules.
\end{itemize}

Proof rules:
\begin{itemize}
  \item The rule \proofRule{implies_simplify} is no longer allowed to
  perform the simplification $(\varphi_1\rightarrow \varphi_2)\rightarrow \varphi_2
  ⇒  \varphi_1\lor \varphi_2$.  This is now covered by \proofRule{bool_simplify}.
\end{itemize}

\subsection*{0.2 --- \DTMdisplaydate{2022}{12}{19}{-1}}

This is an intermediate release.  It collects all changes to the original
specification document before the major changes that were implemented as
part of Hans-Jörg Schurr's PhD thesis.   These changes will be reflected
in release 0.3.

This release implements major changes to the structure of the document to
clarify the difference between the \emph{language} and the \emph{rules}.
The language has a formal definition and a proof of soundness.  The syntax
describes how proofs are encoded in the text file.

The syntax was extended to allow extra annotations.  Tools consuming Alethe
proofs must be able to ignore such extra annotations.

List of rules:
\begin{itemize}
    \item Improve description of \proofRule{sko_ex}.
    \item Add {\proofRule{hole}} rule to allow holes.  A proof that contains
    steps that use this rule is not valid.
\end{itemize}
\noindent
Corrections:
\begin{itemize}
    \item Grammar: the \texttt{choice} binder can only bind one variable.
\end{itemize}

\noindent
Clarifications:
\begin{itemize}
    \item Clarify functionality of choice in introduction.
    \item Add illustrating example to introduction.
    \item Normalize printing of (variable, term) arguments in the abstract notation.
    \item Fix linear arithmetic example in introduction.
    \item Change syntax of abstract proof steps to be clearer.
\end{itemize}


\subsection*{0.1 -- \DTMdisplaydate{2021}{07}{10}{-1}}

This is the first public release of this document.  It coincides with
the seventh PxTP Workshop.


\clearpage
\section*{Index}
\addcontentsline{toc}{section}{Index}
\printindex

\bibliographystyle{plain}
\bibliography{publications}
\end{document}


%%% Local Variables:
%%% mode: latex
%%% TeX-master: t
%%% End:
